
\begin{center}
\textbf{Abstract}
\end{center}

This paper investigates the degree and determinants of the European stock market's reaction to ordinary news related to the United Nations' 17 Sustainable Development Goals. The framework utilizes an event study method along with the Market Model and the Fama-French Five Factor Model to estimate expected returns. I develop an innovative approach to identify events as spikes in the volume of news articles using a time series sample provided by Matter. 

The empirical analysis is based on more than 12,000 events over a period of 5 years, from 2018-2023 and covering 600 listed firms in Europe. Thus, the study does not only focus on extreme events but also on ordinary events related to the SDGs.    

By employing the event study methodology and the Market Model, the study finds evidence that, while investors seem to penalize negative corporate behavior, they do not reward positive behavior. On average, negative events leads to a short-term drop in market value of -0.72\% over an event window of 21-days. Negative reactions are particularly pronounced for themes within the Planet, Peace, and Prosperity groups.Moreover, by employing the Calendar-Time Portfolio along with the Fama-French model, the study finds that on longer horizons of one and four months, the average penalties are -0.84\% and -0.36\%, respectively. 

Furthermore, the thesis explores the influence of a company's ESG risk on reactions to negative events. 
European low-risk companies experience an average short-term penalty of -1.65\%. In contrast, from a sample of global firms, I show that low-risk companies do not encounter significant market penalties on a short horizon. This discrepancy suggests that European and global companies exhibit varying levels of ESG awareness. All the mentioned abnormal returns are significant on at least a 5\% level. 

Finally, I discuss the practical implications in relation to profit-maximization and market efficiency, along with the differences between European and global perceptions of ESG. 
