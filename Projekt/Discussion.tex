
\subsection{Assessment of the event study methodology:}

The overall results show that the impact of positive and negative news is asymmetric. Investors react more negatively to adverse events than they react positively to good events. In order to generalize shareholder behavior to corporate sustainability events, I need to make certain assumptions about how an event is defined and how to calculate the associated abnormal returns. 

For simplicity, I consider the short term analysis in this section. The ordinary analysis defined an event as a one standard deviation move in average daily news articles. A one standard deviation requirement is possibly too relaxed in order to detect significant events. The sensitivity analysis indicated, for negative events at least, that applying more stricter thresholds increased the abnormal returns. Moreover, although this result is not included in the analysis, a more strict threshold for positive events does not change the outcome of the CAAR, as illustrated in figure \ref{fig:ST_pos_sensi_sd}. Hence, the choice of threshold implicates a trade-off between the relevance and quantity of events included in the sample. We can tighten the threshold, which leaves us with more severe events and likewise investor reactions, or we can relax the threshold, which increases the sample of events but lowers the importance and hence the investor reaction. However, are such assumptions realistic and useful in practice? Applying standard deviations enables one to identify events as spikes in news above the expected levels on any given day, which means I can compare the level of news across various companies independent of size and media exposure. I tested both one, two, and three standard deviations and all values are useful depending on the users end goal. I didn't uncover any other useful method to identify events related to this specific dataset. Applying a simple measure directly dependent on the volume of news articles, without adjusting for company size or average media exposure, would bias the sample toward large and trendy companies. As mentioned, the majority of other research papers analyze a database of hand-picked events (\cite{Blancard_ESG_sentiment}, \cite{kruger2015corporate}). 

In addition, this methodology attempt to identify the most extreme days for a given firm. However, large events does not necessarily take place on a specific date. In contrast, an event could span over several days or weeks if investigations or other public circulations are proceeding, with the amount of news articles being consistently high over a given period. For example, the analysis pointed out that significant events are following smaller spikes in news. One of the limitations of the methodology in its current form is the inability to identify events before investors react. If one is able to identify the early period leading up to a given event, the  abnormal returns would be significant. Hence, a possible way to incorporate this feature is to replace the single day measure of news spikes by a rolling average of news with a look-back period of e.g. five days. Still, the threshold of one standard deviation remains. 

From the perspective of a practitioner, the opportunity to choose between alternative thresholds and calculation of averages provides optionality, as one can employ the SDG data in numerous functions. \\
For example, a large investment corporation with an ESG mandate could exploit the methodology and data as a screening tool to keep track of their portfolio companies. If they hold too many assets to follow each one closely, the approach could identify whether specific firms endure increasingly negative news, thus work as an early warning indicator of possible exposure toward unwanted assets. Currently, many corporations apply ESG ratings, which admittedly are good measures of future events \citep{serafeim2022stock}, to screen companies, however the ratings are revised annually, whereas investment decisions (and public scrutiny about investments) are made more frequent. Another use case, that is more related to generating abnormal returns, would relate to, e.g., hedge funds. The practitioner has three options to generate potential profits. He should either establish a method to detect negative events at an early stage, this could be a moving average of various lengths, or follow the approach from section \ref{sec: long_term_portfolio} in order to chase long run alpha. A third options is to take advantage of the consequences that follows negative or positive investor sentiment. The sentiment scores can possibly assist in predicting which companies' ESG ratings are going to be revised up or down in the following weeks or months. According to \cite{ESG_ratings_change} ESG upgrades and downgrades are leading to average abnormal returns of 0.5\% and -1.2\%, respectively. Detecting such notable developments would require the practitioner to survey long horizons, e.g., a 1-month moving average of news sentiment on individual companies. \\
Overall, the assumptions behind the specific use case of the event study methodology seems appropriate to apply in practice from different perspectives. 

\textbf{Is a 10-day CAAR really effective and does it generate insignificant returns?}

\subsection{The case of long term alpha}

The analysis of the long run relation between SDG news and market reactions concluded that negative news has a significant impact on market value. These results are in contrast with semi-strong market efficiency, which proclaims the market should change to equilibrium levels and adjust expectations as a result of new public market information becoming available. While there may be short term deviations from market efficiency, these anomalies tend to dissipate over time. However, post-event long term return anomalies are generally an indication of reactions from an inefficient market. \\
Does this mean that we can simply discard the efficient market hypothesis? Although the long term results are indicative of a simple "yes", there are modest reasons to be lenient. First, the methodology uses significant spikes in monthly news articles as a proxy of events, implying that a specific event may not provide the complete context of a news story. Post-event continuation of abnormal returns could be a reaction to more information about a specific event becoming available to the market. Second, solely the portfolios with holding periods of one and four months generates significant alpha. While the exact definition of a "long horizon" is arbitrary, I have defined it as holding periods of one month or longer. Nonetheless, it appears that  an event window of one year or longer is generally accepted as being long horizon \cite{kothari}. Hence, the abnormal returns within one and four months are then categorized as short term deviations from market efficiency. As the effect diminishes over eight and twelve months, the critique seems appropriate.
In addition, the increased public focus on ESG affairs through the last decade implicate that the abnormal returns may have been a result of overreaction. Investors may exhibit biases and irrational behavior leading to market inefficiencies. However, these deviations are most likely temporary and tend to be corrected over time \citep{fama1998_events}. Hence, a five year period is not sufficient to make final conclusions, since the period itself may be biased. Third, \citeauthor{fama1998_events} has encountered numerous attempts from long term studies to discard the idea of efficient markets. To accommodate the contradiction, \citeauthor{fama1998_events} argues that "Most important, consistent with the market efficiency prediction that apparent anomalies can be due to methodology, most long-term return anomalies tend to disappear with reasonable changes in technique". Hence, abnormal returns become marginal when different statistical approaches are used to measure them. As a final note in favor of market efficiency, this thesis does not incorporate transactions costs in either short or long term portfolio performance, which, if included, would reduce expected abnormal returns further. 

Overall, this leaves the results as a confirmation of the short term investor behavior in relation to SDG news, whereas no long term significant reactions is found. 



\subsection{Practical implications}

The general results of reactions to negative and positive news are in line with most research conducted in the specific area and over various time periods. For example, \cite{klassen1996impact} studies the impact through 1985-1991, \cite{kruger2015corporate} through 2001-2007, and \cite{Blancard_ESG_sentiment} through 2002-2010, which suggests the impact of ESG information on corporate performance has been incorporated in shareholder decisions for a long period. 

Collectively, these findings undermine the notion that shareholders' sole focus is on profit-maximization and suggest that companies can obtain performance advantages by prioritizing objectives beyond profit maximization. Regardless, I cannot provide evidence to determine whether the market reaction is driven by a genuine focus on ESG or if it relates to expectations of future financial performance. In addition, the identified events associated with ESG may also have financial implications beyond sustainability goals. However, in my analysis, I attempted to address this concern by using regular news articles on Sustainable Development Goals to capture the overall impact of ESG news. Consequently, the significant abnormal returns as a short term response to ordinary SDG news, indicates that the market considers corporate sustainability in addition to profit maximization when making investment decisions. Therefore, companies should leverage this understanding for their benefit.


\textbf{Why are my results different from other research?}

\textbf{Something with Flammer --> he argues that increasing pressure to become green has resulted in decreasing reactions to eco-friendly events over time.}


\textbf{Future work:}
