
\subsection{Assessment of the event study methodology:}

The overall results show that the impact of positive and negative news is asymmetric. Investors react more negatively to adverse events than they react positively to good events. In order to generalize shareholder behavior to corporate sustainability events, I need to make certain assumptions about how an event is defined and how to calculate the associated abnormal returns. 

For simplicity, I consider the short term analysis in this section. The ordinary analysis defined an event as a one standard deviation move in average daily news articles. That is, it includes the days where the amount of articles is higher than expected levels. A one standard deviation requirement is possibly too relaxed in order to detect significant events. The sensitivity analysis indicated, for negative events at least, that applying more stricter thresholds increased the abnormal returns. Moreover, although this result is not included in the analysis, a more strict threshold for positive events does not change the outcome of the CAAR, as illustrated in figure \ref{fig:ST_pos_sensi_sd}. Hence, the choice of threshold implicates a trade-off between the relevance and quantity of events included in the sample. We can tighten the threshold, which leaves us with more severe events and, 

Assumably, tighter threshold leads to increased importance of the identified events, which is expected to be followed by more severe shareholder reactions.

From the viewpoint of a practitioner, the possibility to switch between thresholds is valuable, as one can employ the SDG data in numerous functions. 

For example, a risk management analyst with focus on ESG in a large investment corporation could use the methodology and the data to keep track of the companies they invest in. If a company sees increasing negative news and so on + the increasing bars tell us that it is possible to capture early events. 

Moreover, hedge fund => use it as investment strategy like Qblue. 



\textbf{Are the assumptions behind determining events useful in practice?}

\textbf{Mention the limitations of the methodology, such as the potential exclusion of the initial market response to breaking news due to the identification procedure based on volume.}

\textbf{Is a 10-day CAAR really effective and does it generate insignificant returns?}

\textbf{Discuss the effectiveness of the event study methodology in identifying negative and positive events based on spikes in news.}
Describe the observed investor reactions and the timing of news coverage, indicating that reactions may increase with media attention. Highlight the benefits of the sensitivity analysis in validating the robustness of the initial results and providing insights into the drivers behind the findings.




\subsection{Long-term impact of SDG-related news}
\textbf{Discuss the rejection of the hypothesis that negative news has no long-term impact on market values, highlighting the significant negative alphas observed for negative news portfolios.}

\textbf{Compare the results with the efficient market hypothesis and highlight the potential influence of more information becoming available over time.}
\textbf{Discuss the implications of post-event continuation of abnormal returns and the possibility of overreaction or underreaction in the market.}


\subsection{Possible explanations and practical implications}

Shareholders tend to penalize negative corporate social responsibility (CSR) practices but do not consistently reward positive practices. 

\textbf{What is the interpretation of the results?}

How can I reconcile my findings with shareholders' motivation to react to spikes in news about companies related to the Sustainable Development Goals? 


\textbf{Why are my results different from other research?}

\textbf{Something with Flammer --> he argues that increasing pressure to become green has resulted in decreasing reactions to eco-friendly events over time.}


\subsection{How does this research help practitioners?}
- To be able to actually trade this idea, we need a way of finding the negative events before they happen. 


\textbf{Discuss the implications of the results in terms of efficient markets and the incorporation of new information related to the Social Development Goals (SDGs) into investment decisions.}


\textbf{Future work:}
- Future work could try to use the SDG signals (negative and positive events) as an indication of whether a company is expected to be downgraded, which there is some clear benefits in trading. I.e. refer to some papers about this. 

\subsection{Limitations}


\textbf{Transaction costs}
