
In this chapter I go through some of the limitations of the findings and its practical implications. The chapter will include a discussion of the discrepancies between reactions toward European and global companies. 

\subsection{Practical Implications} \label{sec: discussion_practical}

This study's general results on reactions to events are in line with most re-
search conducted in the specific area and over various time periods. For example, \cite{klassen1996impact} studies the impact in the period 1985-1991, \cite{kruger2015corporate} between 2001-2007, and \cite{Blancard_ESG_sentiment} through 2002-2010, which suggests that the impact of ESG information on corporate performance has been incorporated in shareholder decisions over many years. Collectively, these findings challenge the assumption that shareholders solely prioritize profit-maximization, highlighting the potential performance benefits for companies that prioritize objectives beyond profits. Regardless, it remains unclear whether the market reaction reflects a genuine focus on ethical considerations within ESG or relates to expectations of future financial performance linked to ESG. For instance, events related to the SDGs may have practical financial implications, such as court penalties or other costs, extending beyond sustainability. 

In this paper, I aim to address this issue by using the volume of common news articles as a measure of the overall impact of ESG news. A possible conjecture is that if investors react pessimistic to tangible negative events, such as environmentally harmful actions as in \cite{klassen1996impact}, but do not react to ordinary news, it suggests that investors' incentives is more inclined toward profit-maximization rather than ethics. However, I find significant abnormal returns in response to ordinary SDG news, which suggest that the market considers corporate sustainability from an ethical perspective alongside profit maximization in investment decisions. Therefore, companies can benefit from prioritizing sustainability and aligning their strategies with investor preferences. 

Although there is an evident focus on sustainability, the fundamental consideration driving ESG investments continues to be financial performance. A report by \cite{ESG_survey}, based on a survey of approximately 650 investment professionals, reveals that sustainable investment is predominantly motivated by performance rather than ethical concerns. They state that ESG information continues to be material to investments performance and highlights the importance of future-proofing portfolios against probable risks in areas of environmental, social, and governance. 

To sum up, these arguments imply that conventional ideas on profit-maximization are being adjusted to incorporate ESG. However, the incorporation is based on the possibilities of ESG from a financial perspective and not an ethical one. 

\subsection{Europe vs. The World} \label{sec: discussion_Europe}

With sustainability established to play a considerable role in shareholders' investment decisions, the distinction between reactions to European and global companies is certainly relevant. While the overall market reaction to SDG news is relatively similar for European and global companies, the adjustment for ESG risk reveals a contrasting pattern for low-risk companies. The findings from global companies support the idea that reputation serves a protective role in shielding low-risk firms from market sanctions, as mentioned in section \ref{ESG_reputation}. In contrast, high- and medium-risk firms, which presumably have weaker ESG reputations, face significant market penalties with average declines of -1.5\% and -0.5\%, respectively. As the existing literature focuses on non-European companies, the similarity of our results indicate that reputational effects are present in global equity markets but not in Europe (\cite{godfrey2009relationship}; \cite{Blancard_ESG_sentiment}, \cite{flammer2013corporate}).

One potential driver behind the asymmetric investor reactions is that the perception of ESG, as a firm characteristic, is different in various regions. As discussed in section \ref{ESG_reputation}, the large penalties imposed on European low-risk firms suggest that high levels of sustainability are priced into their market valuations in advance, which must be repriced if expectations are not met. This is consistent with the intuition proposed by \cite{flammer2013corporate}, that increasing external social pressure for responsible behavior has resulted in exaggerated reactions to environmentally harmful actions over times. In regions with strong public awareness of ESG, such as Europe, investors may demand and expect more from firms regarding their ESG performance. On the other hand, in global markets where ESG awareness may be less developed, expectations may be set at accordingly lower levels, implying that firms with strong reputation are less penalized for ordinary negative ESG-related news. 

In support of these postulates, a survey from \cite{van2016esg} reveals that portfolio managers in the United States, on average, do not strongly believe in the existence of a positive relation between responsible investing and performance, whereas such a belief is strong in Europe. However, US managers do acknowledge the expected positive long-term impact of responsible investing in terms of risk reduction. Additionally, the survey indicates that practicing responsible investing in Europe comes with sacrifices in financial performance. These findings suggest that the market penalty from negative news related to low-risk firms may be a reaction to the perceived loss in both financial performance and expected sustainability exposure.    


Overall, these results imply that the awareness on ESG on financial markets is stronger in Europe, which leads to more pronounced responses to negative news when expectations are high. 

\subsection{Assessment of The Event Identification Methodology:} \label{sec: discussion_event_study}

Overall, this study finds asymmetry in the impact of positive and negative news. To generalize shareholder behavior towards corporate sustainability events, I need to make certain assumptions about how an event is defined and how to calculate the associated abnormal return. For simplicity, I consider the short term reactions in this section. The ordinary analysis defined an event as a one standard deviation move in average daily news articles. Such a requirement is possibly too relaxed in order to detect significant events. The sensitivity analysis in section \ref{sec:sensitivity} demonstrates that a stricter threshold leads to a slight reduction in abnormal returns. Ideally, employing a more stringent threshold should identify relatively more extreme events, subsequently triggering more pronounced reactions from investors. As this is not the case, I assume that investors and the general public does not share a perception of which events are most important. 

However, are such assumptions realistic and useful in practice? By applying standard deviations to identify events as significant spikes in news coverage, I compare news levels across different companies irrespective of size and expected media exposure. This approach ensures a fair assessment of events without biasing the sample towards larger or more popular companies. One of the limitations of the methodology, in its current form, is the inability to identify events before investors react. With the current methodology I attempt to identify the most extreme days for a given firm. However, events do not necessarily take place on a specific date. In contrast, an event will often span over several days or weeks if investigations or other public circulations are proceeding. For example, the analysis pointed out that significant events are following smaller spikes in news. Hence, the identification can adopt alternative identification methodologies based on, e.g., consistently high levels or incremental increases in the amount of articles over a given period. If a researcher is able to identify the early periods leading up to a given event, it would be possible to capture the significant abnormal returns. However, returns are not always the main concern. 

From the perspective of a practitioner, the opportunity to choose between alternative thresholds and other features in the event selection process provides useful optionality. For example, a large investment corporation with an ESG mandate could leverage this methodology as a screening tool to monitor their portfolio companies. Particularly, with a vast number of assets to track, this approach could serve as an early warning indicator of potential exposure toward undesirable assets. By identifying firms that experience a growing influx of negative news, the corporation can proactively assess and address any concerns in alignment with their ESG objectives. Currently, many asset managers apply ESG ratings to screen companies, which admittedly are good measures of future events \citep{serafeim2022stock}. However, these ratings are typically revised annually, whereas investment decisions and corporate misdeeds occur with greater frequency. 

Another relevant use case for these methods pertains to, e.g., hedge funds and their pursuit of generating abnormal returns. In this context, the practitioner has various options to harness the advantages of these methodologies in the search of alpha. They should either develop a method to detect negative events at an early stage or follow the approach from section \ref{sec: long_term_portfolio} in order to chase long run alpha. A third options is to leverage the auxiliary consequences that arise from significant negative or positive sentiment. By reliably assessing investor sentiment, the practitioner can potentially utilize such a score to assist in predicting which companies' ESG ratings are likely to be revised up or down in the following weeks or months. According to \cite{ESG_ratings_change} ESG upgrades and downgrades are resulting in average abnormal returns of 0.5\% and -1.2\%, respectively. By incorporating investor sentiment as a supplementary indicator, asset managers may gain additional insights into the potential trajectory of ESG ratings for portfolio companies.

In summary, the assumptions behind the specific use case of the event study methodology seems realistic and appropriate to apply in practice from different perspectives. 


\subsection{The Case of Long Term Alpha} \label{sec: discussion_alpha}

The analysis of the relation between SDG news and long horizon market reactions reveals that negative news has a significant impact on market value. These results are in contrast with semi-strong market efficiency, which proclaims that markets should adjust to equilibrium levels and incorporate new public information immediately. While there may be short term deviations from market efficiency, these anomalies tend to dissipate over time. However, post-event long term return anomalies are generally an indication of reactions from an inefficient market. \\
Does this mean that we can simply discard the efficient market hypothesis? 

While these findings challenge the efficient market hypothesis in the context of SDG news and long-term market reactions, there are modest reasons to be lenient in discarding the hypothesis. First, the methodology identifies significant events as spikes in monthly news articles, which implies that a specific event may not provide the complete context of a news story. Post-event continuation of abnormal returns could be a reaction to more information about a specific event becoming available to the market. Second, only portfolios with holding periods of one and four months generate significant alpha. The definition of a "long horizon" is arbitrary, but for this study, it was considered as holding periods of one month or longer. Nonetheless, it appears that an event window of one year or longer is generally accepted as a long horizon \cite{kothari}. Therefore, abnormal returns within one and four months are classified as short-term deviations from market efficiency. As the impact diminishes over eight and twelve months, the critique seems appropriate.

In addition, the heightened public attention to ESG affairs over the past decade suggest that the observed abnormal returns could be attributed to overreactions by investors. Biases and irrational behavior may lead to market inefficiencies. However, \citep{fama1998_events} states that these deviations are typically temporary and tend to be corrected over time. Therefore, drawing final conclusions based on a five-year period may be inadequate, as the period itself could be influenced by biases. Third, \citeauthor{fama1998_events} has encountered numerous attempts from long term studies to discard the idea of efficient markets. To accommodate the contradiction, \citeauthor{fama1998_events} argues that "Most important, consistent with the market efficiency prediction that apparent anomalies can be due to methodology, most long-term return anomalies tend to disappear with reasonable changes in technique". Hence, abnormal returns become marginal when different statistical approaches are used to measure them. As a final point in favor of market efficiency, these results does not incorporate transactions costs in either short or long term portfolio performance, which, if included, would reduce expected abnormal returns further. 

Overall, in order to counter the notion of efficient markets, we need to find consistent market inefficiencies over longer horizons. As the long run is typically defined as one year or longer, the inference from section \ref{sec: long_term_analysis} is not sufficient evidence to contradict the hypothesis of efficient markets. 

The main points of this chapter include the following. Initially, in section \ref{sec: discussion_practical} I argue that while profit-maximization is still the first and foremost purpose of the company, it should be adjusted to incorporate ESG matters. From the discussion in section \ref{sec: discussion_Europe} it appears that the awareness on ESG is larger on Europe relatively to global markets, which seemingly causes the reputation effect from ESG to diminish. According to section \ref{sec: discussion_practical} the event study in this specific paper is relevant in areas outside of academics, due to the practicality and flexibility of the methodology, e.g, as a screening tool on companies. Finally, section \ref{sec: discussion_alpha} goes through the limitation of the long-term results. The observed long-term alpha is not sufficient in order to challenge hypotheses on efficient markets. 

\pagebreak

\section{Conclusion}

In this thesis I have investigated how investors react to corporate news related to the United Nations' Sustainable Development Goals. I apply an event study methodology in order to investigate the reaction to specific events. These events are based on SDG data from Matter, where an event is identified as a spike in positive or negative news articles related to a specific company. Thus, both ordinary and extreme events are sampled, and the procedure identifies more than 10,000 positive and 1,600 negative daily events across the period 2018-2023. I use the Market Model and the Calendar-Time Portfolio with a Fama-French five Factor model to calculate, respectively, short-term and long-term abnormal returns. 

By analyzing all companies which experienced an event, related to the SDGs in broad terms, the results initially point out that shareholders do in fact react significantly to SDG news. On a short horizon of 21-days around the event, shareholders penalize companies associated with negative events with an average decline in market value of -0.65\%. In contrast, positive events does not affect investor reactions on average. The severity of the events or the methodology of calculating portfolio weights does not have a large effect on the reaction over the full window. Thus, I reject the hypothesis, that negative SDG events has no impact on market value. 

In order to assess the importance of various themes within the development goals, I examine the same short-term relation but with the SDGs categorized into the Five Pillars of SDGs. Shareholders clearly differentiate between SDGs in terms of importance, as the themes Planet, Peace, and Prosperity generate negative abnormal returns in the range of -0.53\% to -0.70\% over the full period, while People and Partnership come out positive, but insignificant. Concerning positive events, these themes provide insignificant outcomes as well. 

The long-term analysis supports the notion that investors tend to react significantly negative to adverse events, while there is no reaction to positive events. The portfolios with holding periods of one and four months after a negative event generate alpha values of -0.84\% and -0.36\%, respectively. Accordingly, I reject the hypothesis that negative SDG news does not have a long-term impact on market values. With that said, the paper discusses whether such results are indicative of market inefficiencies, and concludes that a four month horizon, and especially one month, is generally too narrow to challenge the notion of efficient markets. 

Moreover, I partition the sample of companies based on their ESG risk rating and conduct the analysis from scratch. The main result is that low-risk firms are experiencing a significant short-term market penalty of -1.65\%, as well as a long-term alpha of -0.64\% with a one month holding period, after negative news. Possibly, these reactions are due to high investor expectations on ESG in Europe, in which a negative event introduces a need to recalibrate these expectations.

To test the universality of these results, I conduct the analysis on a large sample of global companies. Overall, global companies react like European ones on average. However, the findings establish that low-risk global companies are not penalized from negative news. The apparent discrepancies between European and global companies may emerge from the level of ESG awareness in specific regions.  



