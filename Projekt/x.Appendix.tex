
\subsection{Empirical results}

\begin{figure} [H]
    \centering
    \caption{Negative news: AAR split on relation to SDGs}
    \includegraphics[scale=0.6]{Projekt/1.Figures analysis/ST_negative_sdg_bar.png}
    \caption*{\footnotesize The figure illustrates the AAR on $t = 0$ from negative news. The error bars represent the 95\% confidence intervals of the AAR.}
    \label{fig:ST_neg_bar}
\end{figure}

\begin{figure} [H]
    \centering
    \caption{AAR per SDG: positive news}
    \includegraphics[scale=0.6]{Projekt/1.Figures analysis/ST_positive_sdg_bar.png}
    \caption*{\footnotesize The figure illustrates the AAR on $t = 0$ from positive news. The error bars represent the 95\% confidence intervals of the AAR.}
    \label{fig:ST_pos_bar}
\end{figure}




\subsection{Sensitivity analysis figures:}
\subsubsection{Short term}


\begin{table}[H]
\centering
\caption{AAR and CAAR (in \%) with new event rule (sd)} 
\begin{tabular}{lccccc}
  \hline  \hline
  & \multicolumn{2}{c}{Positive} &  & \multicolumn{2}{c}{Negative}\\ \cline{2-3} \cline{5-6}  
  & 2 sd & 3 sd & & 2 sd & 3 sd   \\   
 \hline
$AAR_{t=0}$ & $\underset{(0.875)}{0.051}$ & $\underset{(-0.278)}{-0.022}$ & & $\underset{(-3.409)}{-0.522^{***}}$ & $\underset{(-3.024)}{-0.617^{***}}$ \\ 
$CAAR_{[-2;+2]}$  & $\underset{(0.283)}{0.031}$  & $\underset{(0.631)}{0.089}$ & & $\underset{(-3.66)}{-0.714^{***}}$ & $\underset{(-2.648)}{-0.657^{***}}$ \\ 
$CAAR_{[-5;+5]}$  & $\underset{(1.197)}{0.195}$  & $\underset{(1.350)}{0.290}$ & &$\underset{(-3.56)}{-0.902^{***}}$ & $\underset{(-2.782)}{-0.867^{***}}$ \\ 
$CAAR_{[-10;+10]}$  & $\underset{(-0.168)}{-0.035}$  & $\underset{(-0.062)}{-0.018}$ &  & $\underset{(-1.886)}{-0.652^{*}}$ & $\underset{(-1.686)}{-0.697^{*}}$ \\ 
N & 1845 & 1078 & & 648 & 451  \\
   \hline \hline
   \multicolumn{6}{p{12cm}}{ \footnotesize $^* \; p\; <\; 0.1$, $ ^{**} \; p\; <\; 0.05$, $ ^{***} \; p\; <\; 0.01$  } \\
   \multicolumn{6}{p{13cm}}{\footnotesize The table shows the CAAR associated with positive and negative news over an event window of 5, 10, and 21 days surrounding the event date along with the AAR on $t=0$. The models are split on the requirement rule for included events (2 or 3 sd).} \\
   \hline
\end{tabular}
\label{tab:ST_sensitivity}
\end{table}


\begin{table}[H]
\centering
\caption{AAR and CAAR (in \%) with equally weighted returns} 
\begin{tabular}{lcccc}
  \hline  \hline
  & \multicolumn{1}{c}{Positive} &  \multicolumn{1}{c}{Negative}\\  
 \hline
$AAR_{t=0}$ &  $\underset{(0.634)}{0.045}$ & $\underset{(-3.509)}{-0.400^{***}}$ \\ 
$CAAR_{[-2;+2]}$  & $\underset{(0.367)}{0.053}$ & $\underset{(-4.120)}{-0.662^{***}}$ \\ 
$CAAR_{[-5;+5]}$  & $\underset{(-1.233)}{-0.258}$ & $\underset{(-5.457)}{-1.100^{***}}$ \\ 
$CAAR_{[-10;+10]}$    & $\underset{(-0.452)}{-0.128 }$ & $\underset{(-5.702)}{-1.558^{***}}$ \\ 
   \hline \hline
   \multicolumn{3}{p{10cm}}{ \footnotesize $^* \; p\; <\; 0.1$, $ ^{**} \; p\; <\; 0.05$, $ ^{***} \; p\; <\; 0.01$  } \\
   \multicolumn{3}{p{10cm}}{\footnotesize The tables shows the CAAR associated with positive and negative news over an event window of 5, 10, and 21 days surrounding the event date along with the AAR on $t=0$. The models are split on the requirement rule for included events (2 or 3 sd).} \\
   \hline
\end{tabular}
\label{tab:ST_sensitivity_weights}
\end{table}


\begin{figure} [H]
    \centering
    \caption{Positive news: event identification rule}
    \includegraphics[scale=0.6]{Projekt/1.Figures analysis/ST_positive_sensitivity.png}
     \caption*{\footnotesize The figure illustrates the average abnormal return (AAR) and cumulative AAR (CAAR) around the event date (t = 0) of negative news. The figure illustrates the average abnormal return (AAR) and cumulative AAR (CAAR) around the event date (t = 0) of negative news. The various colors represent whether the event identification rule was based on 1, 2, or 3 standard errors. }
    \label{fig:ST_pos_sensi_sd}
\end{figure} 

\begin{figure} [H]
    \centering
    \caption{Positive news: Value vs. Equal weights}
    \includegraphics[scale=0.6]{Projekt/1.Figures analysis/ST_positive_sensitivity_weight.png}
     \caption*{\footnotesize The figure illustrates the average abnormal return (AAR) and cumulative AAR (CAAR) around the event date (t = 0) of positive news. The blue lines are returns calculated from an equally weighted portfolio, while the weights of the red lines are based on market capitalization. }
    \label{fig:ST_pos_sensitivity_weights}
\end{figure} 

\subsubsection{Long term}


\setlength{\tabcolsep}{15pt}
\begin{table}[]
\small
\centering
\caption{Fama-French five-factor model alpha from positive news with equal weights} 
\begin{tabular}{ccccccc}
\hline \hline \\ 
 &     &  &    1 SD  &  2 SD  &  3 SD  &  \\ \cline{4-6} 
& & & \multicolumn{3}{c}{ T = 1} & \\ \cline{2-6}
& Alpha (\%)  &  & $ -0.004$  & $-0.000$  & $-0.006$ &  \\
& t-value &  & -1.55 & -0.27  & -1.48 & \\
& & & \multicolumn{3}{c}{ T = 4} & \\ \cline{2-6}
& Alpha (\%)  &  & $ -0.003^{*}$  & $-0.001^{***}$  &  $-005^{**}$ & \\
& t-value & & -1.77  & -0.75 & -227 & \\
& & & \multicolumn{3}{c}{ T = 8} & \\ \cline{2-6}
& Alpha (\%)  &  & $ -0.003^{***}$   & $-0.003^{**}$  & $-0.005^{**}$ &  \\
& t-value &  & -2.98  & -2.45 & 3.51 & \\
&  & & \multicolumn{3}{c}{ T = 12} & \\ \cline{2-6}
& Alpha (\%)  &  & $ -0.004^{***}$  & $-0.003^{**}$  & $-0.003^{**}$ &  \\
& t-value &  & -3.02  & -2.66 & -1.12 & \\
\hline \hline
 \multicolumn{7}{l}{ \footnotesize $^* \; p\; <\; 0.1$, $ ^{**} \; p\; <\; 0.05$, $ ^{***} \; p\; <\; 0.01$  } \\
 \multicolumn{7}{p{11.5cm}}{ \footnotesize Alpha is the WLS-regression intercept (in \%) of the Fama-French 3-factor model, displayed along with the corresponding t-value. N is the average amount of firms included in the portfolio each month, and T is the portfolio holding period. The threshold for event firms to be included in the portfolio is either 1,2 or 3 "SD" (standard deviations) larger than the mean.}  \\ 
\end{tabular}
\label{tab: FF5_pos_equalW}
\end{table}


\setlength{\tabcolsep}{15pt}
\begin{table}[]
\small
\centering
\caption{Fama-French five-factor model alpha from positive news with RF = 0} 
\begin{tabular}{ccccccc}
\hline \hline \\ 
 &     &  &    1 SD  &  2 SD  &  3 SD  &  \\ \cline{4-6} 
& & & \multicolumn{3}{c}{ T = 1} & \\ \cline{2-6}
& Alpha (\%)  &  & $-0.28$  & $0.02$  & $-0.38$ &  \\
& t-value &  & -1.08 & -0.06  & -0.78 & \\
& & & \multicolumn{3}{c}{ T = 4} & \\ \cline{2-6}
& Alpha (\%)  &  & $ -0.004$  & $-0.02$  &  $-0.08$ & \\
& t-value & & -0.28  & 0.11 & --0.26 & \\
& & & \multicolumn{3}{c}{ T = 8} & \\ \cline{2-6}
& Alpha (\%)  &  & $ -0.00$   & $-0.003^{**}$  & $0.03$ &  \\
& t-value &  & 0.05  & -0.17 & 0.16 & \\
&  & & \multicolumn{3}{c}{ T = 12} & \\ \cline{2-6}
& Alpha (\%)  &  & $ -0.05$  & $0.13$  & $0.23$ &  \\
& t-value &  & 0.77  & 1.06 & 1.32 & \\
\hline \hline
 \multicolumn{7}{l}{ \footnotesize $^* \; p\; <\; 0.1$, $ ^{**} \; p\; <\; 0.05$, $ ^{***} \; p\; <\; 0.01$  } \\
 \multicolumn{7}{p{11.5cm}}{ \footnotesize Alpha is the WLS-regression intercept (in \%) of the Fama-French 3-factor model, displayed along with the corresponding t-value. N is the average amount of firms included in the portfolio each month, and T is the portfolio holding period. The threshold for event firms to be included in the portfolio is either 1,2 or 3 "SD" (standard deviations) larger than the mean.}  \\ 
\end{tabular}
\label{tab: FF5_RF}
\end{table}

