\section{Introduction} \label{sec:intro}

The global financial markets has experienced tremendous growth within sustainable investing. The mega-trend has evolved over the last two decades and emphasises corporate focus on non-financial information categorized as environmental, social, and governance (ESG) in investment decisions. A growing societal focus on capital allocation toward sustainable investments, accountability and transparency, implies that firms cannot override infringement cases such as oil spills, accounting fraud and the use of child labor, which formerly have experienced less public attention. Recent history provides several instances of adverse corporate social behavior causing large financial losses for afflicted companies. For instance, Swedbank faced allegations as being part of a money laundering scheme resulting in a drop in market share of -13\% on the same day news articles spiked. Similarly, the shares of Électicité de France plunged -15\% after a spike in news about faults found in the nuclear reactors. 

Despite the presence of anecdotal evidence, our understanding of the explicit shareholder reactions from everyday ESG news are limited. Recently, \cite{Blancard_ESG_sentiment} as well as \cite{kruger2015corporate} provides interesting insights on how shareholders react negatively to ordinary news about corporate social responsibility (CSR). 

In this study...



Understanding the relationship between SDG-related news and stock returns could provide valuable insights for investors who are looking to incorporate sustainability criteria into their investment decisions. This is particularly important as sustainable investing continues to grow in popularity, with more and more investors seeking to align their portfolios with their values. Additionally, understanding the relationship between SDG-related news and stock returns could also inform the efforts of companies and policymakers to achieve the SDGs by incentivizing companies to prioritize sustainability initiatives and enabling policymakers to design policies that support sustainable development. Therefore, investigating the relationship between SDG-related news and stock returns is not only important for investors, but also for companies, policymakers, and society as a whole.


\subsection{Motivation}


\subsection{Problem statement}

As investors increasingly consider environmental, social, and governance (ESG) factors when making investment decisions, it is important to understand the relationship between ESG-related news and stock market returns.  Specifically, does news related to the United Nations' Sustainable Development Goals (SDGs) have an impact on firms' short-term and long-term market values. The overall research questions that this thesis will investigate is the following: \\

What is the relationship between news related to the United Nations' Sustainable Development Goals and stock market returns?

To investigate this question, I have developed two main hypotheses. First, I hypothesize that news related to SDGs does not have a significant short term impact on firms' market values. The hypothesis is partitioned between positive and negative news through creation of an "a" and a "b" hypothesis.  \\

\textbf{Hypothesis 1:} 

\begin{itemize}
  \item \textbf{a.}  Negative news related to the SDGs does not have a short term impact on firm's market values.
  \item \textbf{b.}  Positive news related to the SDGs does not have a short term impact on firm's market values.
\end{itemize}

Second, I hypothesize that news related to SDGs does not have a significant long-term impact on firms' market values. \\

\textbf{
Hypothesis 2: \textit{Positive and negative news related to the SDGs does not have a long term impact on firms' market values.}}

\begin{itemize}
  \item \textbf{a.}  Negative news related to the SDGs does not have a long term impact on firm's market values.
  \item \textbf{b.}  Positive news related to the SDGs does not have a long term impact on firm's market values.
\end{itemize}


By testing these hypotheses, I hope to contribute to a better understanding of the relationship between sustainability and financial performance. 

However, I also recognize that there may be factors that influence the relationship between SDG-related news and market values. 
To explore these potential moderators, I have developed two additional sub-questions, which ultimately should expand our interpretation of the results from the main hypotheses. 

One potential moderator is the specific SDG that the news is related to. Each of the 17 SDGs represents a different aspect of sustainable development, from eradicating poverty to ensuring access to clean water and sanitation. In this regard, I will review whether the potential impact of SDG-related news on short and long term market values is independent of which specific SDG the news are related to.

Another potential moderator is the firm's level of ESG risk. ESG risk refers to the degree to which a company's operations and business practices are aligned with environmental, social, and governance principles.
By splitting the analysis on the basis of firms' ESG risk profile, I can investigate whether the impact of SDG-related news on short and long term market values is dependent on the level of ESG risk of the company. 

 
By testing these hypotheses and exploring potential moderators, I hope to gain a more nuanced understanding of how SDG-related news may impact firms' market values. Moreover, this knowledge can help investors make more informed decisions around ESG investing and encourage companies to align their business practices with sustainable development principles.


\subsection{Method}

\subsection{Scope and delimitation}

