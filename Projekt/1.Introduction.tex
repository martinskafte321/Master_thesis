\section{Introduction} \label{sec:intro}
The global financial markets has experienced tremendous growth within sustainable investing. The mega-trend has evolved over the last two decades and incorporates an approach that emphasises corporate focus on non-financial information categorized as environmental, social, and governance (ESG) in portfolio selection. A growing societal focus on capital allocation toward sustainable investments, accountability and transparency, implies that firms cannot override infringement cases such as oil spills, accounting fraud and the use of child labor, which formerly have experienced less public attention. Furthermore, the research focus on climate change has enlightened the public and assisted a cultural shift toward a stronger focus on ESG. The escalation in sustainable investing has been institutionalized through the United Nations Principle for Responsible Investments (PRI). The PRI's long term ambition is to develop a more sustainable global financial system by encouraging and supporting signatories in incorporating the ESG factors into investment decisions. By participating in the treaty, asset managers commit to implement the financial and reporting principles for sustainable investments. The treaty has increased its number of signatories from 100 in 2006 with USD 6 trillion in assets under management (AUM) to 5.179 signatories with more than USD 121 trillion AUM \footnote{//www.unpri.org/pri.}. 

ESG objectives are becoming a primary target in asset management, where reallocation of capital has large complications for financial decision making and asset pricing. In line with increasing focus on sustainable investing, the CEO of one of the world largest asset manager, BlackRock, Larry Fink wrote in his 2020 annual letter to CEO's that his firm would exit "investments that present a high sustainability-related risk" \citep{Blackrock}. 

Understanding the relationship between SDG-related news and stock returns could provide valuable insights for investors who are looking to incorporate sustainability criteria into their investment decisions. This is particularly important as sustainable investing continues to grow in popularity, with more and more investors seeking to align their portfolios with their values. Additionally, understanding the relationship between SDG-related news and stock returns could also inform the efforts of companies and policymakers to achieve the SDGs by incentivizing companies to prioritize sustainability initiatives and enabling policymakers to design policies that support sustainable development. Therefore, investigating the relationship between SDG-related news and stock returns is not only important for investors, but also for companies, policymakers, and society as a whole.


- Massive inflow to sustainability funds
- Increased academic focus
- missing clear link
- ESG --> lower expected returns and thus a premium for sin stocks
- Some find ESG --> greater returns


- ESG ratings --> low frequency/granularity --> more difficult to use as a indicator
- ESG rating --> doesn't change very often, so difficult to use as explaining variable for returns


The growing attention towards sustainability in general has increased the focus on company specific ESG-ratings. However, ESG profiles can be difficult to

The main disadvantage of earlier research are the samples behind the results. First, the samples are often small. Second, they consist of a fixed database of news articles, which mean that they cannot be used on a daily or monthly basis to actively manage a portfolio. This thesis utilizes a database that updates on a daily basis, indicating that all new events can be captured on a daily basis.


\subsection{Motivation}

\subsection{Problem statement}

As investors increasingly consider environmental, social, and governance (ESG) factors when making investment decisions, it is important to understand the relationship between ESG-related news and stock market returns.  Specifically, does news related to the United Nations' Sustainable Development Goals (SDGs) have an impact on firms' short-term and long-term market values. The overall research questions that this thesis will investigate is the following: \\

What is the relationship between news related to the United Nations' Sustainable Development Goals and stock market returns?

To investigate this question, I have developed two main hypotheses. First, I hypothesize that news related to SDGs does not have a significant short term impact on firms' market values. The hypothesis is partitioned between positive and negative news through creation of an "a" and a "b" hypothesis.  \\

\textbf{Hypothesis 1:} 

\begin{itemize}
  \item \textbf{a.}  Negative news related to the SDGs does not have a short term impact on firm's market values.
  \item \textbf{b.}  Positive news related to the SDGs does not have a short term impact on firm's market values.
\end{itemize}

Second, I hypothesize that news related to SDGs does not have a significant long-term impact on firms' market values. \\

\textbf{
Hypothesis 2: \textit{Positive and negative news related to the SDGs does not have a long term impact on firms' market values.}}

\begin{itemize}
  \item \textbf{a.}  Negative news related to the SDGs does not have a long term impact on firm's market values.
  \item \textbf{b.}  Positive news related to the SDGs does not have a long term impact on firm's market values.
\end{itemize}


By testing these hypotheses, I hope to contribute to a better understanding of the relationship between sustainability and financial performance. 

However, I also recognize that there may be factors that influence the relationship between SDG-related news and market values. 
To explore these potential moderators, I have developed two additional sub-questions, which ultimately should expand our interpretation of the results from the main hypotheses. 

One potential moderator is the specific SDG that the news is related to. Each of the 17 SDGs represents a different aspect of sustainable development, from eradicating poverty to ensuring access to clean water and sanitation. In this regard, I will review whether the potential impact of SDG-related news on short and long term market values is independent of which specific SDG the news are related to.

Another potential moderator is the firm's level of ESG risk. ESG risk refers to the degree to which a company's operations and business practices are aligned with environmental, social, and governance principles.
By splitting the analysis on the basis of firms' ESG risk profile, I can investigate whether the impact of SDG-related news on short and long term market values is dependent on the level of ESG risk of the company. 

 
By testing these hypotheses and exploring potential moderators, I hope to gain a more nuanced understanding of how SDG-related news may impact firms' market values. Moreover, this knowledge can help investors make more informed decisions around ESG investing and encourage companies to align their business practices with sustainable development principles.


%Hypothesis 1: \textit{Negative news related to the SDGs does not have a short term impact on firm' market value.}
%Hypothesis 1a: \textit{A potential impact is independent on which SDG the news are related to.}  
%Hypothesis 1b: \textit{The potential impact is independent on the firm's level of ESG risk.}

%Hypothesis 2: \textit{Positive news related to the SDGs does not have a short term impact on firm' market value.}
%Hypothesis 2a: \textit{A potential impact is independent on which SDG the news are related to.}  
%Hypothesis 2b: \textit{The potential impact is independent on the firm's level of ESG risk.}

%Hypothesis 3: \textit{The short term impact of news related to SDGs on firms' market value is equivalent for negative news and positive news.}

%Hypothesis 4: \textit{Negative news related to the SDGs does not have a long term effect on firms market value}
%Hypothesis 4a: \textit{The potential impact is independent on the firm's level of ESG risk.}


%Hypothesis 5: \textit{Positive news related to the SDGs does not have a long term effect on firms market value.}
%Hypothesis 5a: \textit{The potential impact is independent on the firm's level of ESG risk.} 

%Hypothesis 6: \textit{The long term impact of news related to SDGs on firms' market value is equivalent for negative news and positive news.}

\subsection{Method}

\subsection{Scope and delimitation}

\subsection{Structure}
