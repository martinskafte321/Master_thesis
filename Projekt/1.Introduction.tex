\section{Introduction} \label{sec:intro}
The global financial markets has experienced tremendous growth within sustainable investing. The mega-trend has evolved over the last two decades and incorporates an approach that emphasises corporate focus on non-financial information categorized as environmental, social, and governance (ESG) in portfolio selection. With a growing societal focus on sustainability, capital allocation toward sustainable investments, as a response to public demand for accountability and transparency. The societal focus on sustainability indicates that firms cannot override infringement cases such as oil spills, accounting fraud and the use of child labor, which formerly have experienced less public attention. Furthermore, the research focus on climate change has enlightened the public and assisted a cultural shift toward a stronger focus on ESG. The escalation in sustainable investing has been institutionalized through the United Nations Principle for Responsible Investments (PRI). The PRI's long term ambition is to develop a more sustainable global financial system by encouraging and supporting signatories in incorporating the ESG factors into investment decisions. By participating in the treaty, asset managers commit to implement the financial and reporting principles for sustainable investments. The treaty has increased its number of signatories from 100 in 2006 with USD 6 trillion in assets under management (AUM) to 5.179 signatories with more than USD 121 trillion AUM \footnote{//www.unpri.org/pri.}. 

The numbers clearly indicate that the ESG objective is becoming a primary target in asset management, where reallocation of capital has large complications for financial decision making and asset pricing. In line with increasing focus on sustainable investing, the CEO of one of the world largest asset manager, BlackRock, Larry Fink wrote in his 2020 annual letter to CEO's that his firm would exit "investments that present a high sustainability-related risk" \citep{Blackrock}. 

- Massive inflow to sustainability funds
- Increased academic focus
- missing clear link
- ESG --> lower expected returns and thus a premium for sin stocks
- Some find ESG --> greater returns


- ESG ratings --> low frequency/granularity --> more difficult to use as a indicator
- ESG rating --> doesn't change very often, so difficult to use as explaining variable for returns


The growing attention towards sustainability in general has increased the focus on company specific ESG-ratings. However, ESG profiles can be difficult to







\subsection{Motivation}

\subsection{Problem statement}

The interest of this paper lies in analyzing whether it’s possible to generate alpha by investing in ESG-intensive companies – i.e.
a strategy that goes long companies with high ESG engagement and short companies with low ESG engagement. 

The research question will be something in relation to; “Does investing in ESG-intensive companies generate significant alpha?”. \\

The question will be followed up by a couple of sub questions/hypothesis revolving specific sectors and time periods. For example, analyzing specific time periods will reveal whether a potential overperformance is generated in periods of high investor attention towards the green transition or simply of inflow into sustainable funds. 

And hypothesis
\subsection{Method}

The analysis will initially rely on portfolio sorts. The first step is to rank all companies by their ESG-related sentiment every month and sort them into deciles, meaning that a group consisting of the 10\% best ESG performers will be created every month (similar for the 10\% worst). Consequently, the strategy is to go long the group with the highest ESG sentiment every month and, possibly, short the worst one. The return of a given portfolio is would thus resemble the return of an ESG factor, and is found by aggregating the individual returns of the specific companies in a the group.  

After calculating the returns of the portfolios I will analyze whether any potential over-performance is explained (in a regression) after adjusting for exposure towards standard factors such as value, size and momentum (and more).
\subsection{Scope and delimitation}

\subsection{Structure}
