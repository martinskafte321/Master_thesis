

\subsection{A short overview of ESG and Corporate Financial Performance}

The initial mentioning of the term "ESG" has been attributed to the study "Who Cares Wins"  by the UN Global Compact in 2004 \citep{WhoCaresWins}, where the succeeding "Who Cares Wins Initiative" from 2004-2008 was built on the idea of a all-win situation for the financial industry, society and the environment. The main purpose was to incentivize the financial industry to incorporate ESG topics into their financial decision-making. The initiative was successful as the aforementioned PRI was established few years later, after which the focus sustainability in the financial world has experienced a dramatic increase. In line with the increased funding and awareness to ESG in the financial industry, the academic attention towards the issues has evolved positively as well.
According to a 1970's essay by Friedman \citeyear{friedman2007} the sole responsibility of a firm is to generate profits for its shareholders. Such a view would imply that any ESG related activity, that is not a part of the core business, should not be taken forward by the manager and neither should investors integrate information on ESG into their financial decision making.

The recent literature seems to disagree whether investing in ESG-leaders generate over-performance (alpha) in terms of returns relative to ESG-laggards. Most literature acknowledge a certain investor response to ESG initiatives, which affects the stock price, but there seems to be no clear consensus on whether the relative performance is positive, negative or simply explainable by other factors. 

\cite{ESG_meta_analysis} made an meta-analysis of more than 2.000 studies looking into the potential of ESG investing. The results of the studies differ depending on which ESG methodology (e.g. source and granularity) were used and which financial metrics were applied (e.g. return or exposure to Fama-French factors). They conclude that the majority of studies find stable positive ESG impact on performance over time. However,  once screened for portfolio performance, such as transaction costs, the studies reporting a positive correlation decreases. 


Although the evidence of negative correlation between ESG-activities and CFP is limited, the theoretical foundation is supported by the aforementioned statement from Friedman and the efficient market hypothesis. If the market is partly driven by socially responsible investors that screen out low-ESG assets, then one would theoretically observe negative correlation between sustainable investments and CFP, and investors would pay a premium for their preference for sustainability. 


Some empirical studies give reason to doubt an optimistic view on a firm's ESG activities, providing evidence for financial costs for companies from being proactive in ESG matters. 

For example \cite{ESG_Frontier} integrates ESG characteristics in the portfolio construction of the efficient frontier instead of as a screening factor. They quantify the cost of ESG preferences as the drop in Sharpe ratio when choosing a portfolio with better ESG characteristics than the efficient portfolio. The authors find that integrating the ESG preferences by a proxy for governance, the maximum Sharpe ratio is achieved for a high level of ESG, implying that ethical goals can be done at little cost. Imposing constraints will always reduce the Sharpe ratio for any given ESG score. The proxies for E, S and overall ESG are weaker predictors of future profits. 

Moreover, the findings from \citep{uk_ESG_stock} show that companies with higher social performance scores exhibit minor returns than those with lower social performance scores. They state that social performance indicators are negatively related to stock returns and that abnormal returns are available from holding a portfolio of the socially least desirable firms. In addition, \cite{Kacperczyk_sin_stocks} advocate that so called "sin stocks"\footnote{Classified as companies involved in alcohol, tobacco, gambling or the weapons industry. } have higher expected returns than stocks of otherwise comparable characteristics, due to them being neglected by constrained investors. Negative screening will lead to these stocks being under-priced in relation to fundamental value.   

For example, using yearly sustainability ratings \cite{Shrunk_ESG_ALPHA} address the question whether non-financial information in ESG scores offers additional performance benefits by construct long-short ESG strategies and asses their value-added to investors. They find that claims of ESG out-performance only hold when considering isolated returns. However, when applying standard risk adjustments, the strategies perform like simple quality strategies constructed from accounting ratios. 


Supporting the evidence from \cite{ESG_meta_analysis}, the paper from \cite{factor_ESG_integration} investigates the impact of ESG integration on different factor-investment strategies. Their results show that a manager of a factor portfolio can increase the portfolio exposure towards ESG intensive companies without impairing the exposure toward the desired factors, such as value and low volatility. For example, the minimum volatility strategy only experienced a 7\% reduction in factor exposure from a 30\% increase in exposure to ESG. The results suggest that investors would be able to improve the ESG characteristics of their portfolio without harming their ability to capture market returns. 

Similarly \cite{ESG_exposure_approach} finds averagely positive ESG return premia between 2009 and 2020. They utilize an exposure-matching approach to isolate and measure the return premium related to ESG by neutralizing the exposure to other factors. The benefit is that they are able to control for risk and sector differences to measure month-on-month ESG returns. They also discuss the advantages of using their approach over a regression approach that sorts stocks into portfolios given ESG attributes, but mainly state that it is of more practical use whereas the latter is more useful within academia.

Additional research acknowledge an investor response to firm's ESG activities which affects the companies performance in different ways (\cite{lins2017social}; \cite{kim2014corporate}; \cite{el2011does}). These studies use yearly observations and explain their findings by concepts of trust-building in sustainable activities, which facilitates the firm to obtain benefits such as lower cost-of-capital, lower crash risk, higher and long-run profitability.

The insights from these findings dictate various long-term benefits of both investors and firms from engaging in ESG intensive investment activities. It is relevant to assume that these benefits are particularly suitable for long term investors in diversified portfolios, whereas short term investment strategies will require information on a more granular basis than yearly ratings, when implementing ESG factors into their investment decision. Applying yearly observations ignores the changing landscape of the market, where news are spreading promptly over the internet. Moreover, the potentially impulsive behavior of a large group of speculators and retail investors, who might base their decisions on instant new information becoming available, as per \cite{black1986noise}. Capitalizing on the possibilities of looking beyond yearly data points may add more insights into whether investors incorporate ESG-related information into their financial decision making \citep{Sustainable_sentiment}. 

Increasing the granularity of observations can be done by relying on public news articles, with a focus on ESG-related matters, as a proxy for ESG engagement of companies. 

- Sustainable investing with ESG rating uncertainty

\subsection{Relevant sentiment analysis literature}


Our approach evolves around a popular core of literature that links public sentiment, through news articles, with financial performance on the stock market. Two of the early comprehensive publications are \cite{tetlock_sentiment} and \cite{baker_sentiment}. \citeauthor{tetlock_sentiment} shows that high media pessimism predicts downward pressure on stock prices followed by a reversion to fundamentals. \citeauthor{baker_sentiment} studies how investor sentiment affects stock returns. Their main empirical finding is that future stock returns are conditional on beginning-of-period sentiment. That is, when sentiment is high, stocks that are attractive to speculators - small stocks, unprofitable stocks, high-volatility stocks and more - tend to earn relatively low subsequent returns, while if sentiment is low these stocks earn relatively high returns.   

\cite{Blancard_ESG_sentiment} utilize the ideas from the aforementioned authors to investigate the relationship between positive and negative ESG news and changes in firm market value. They conduct an event study and find that investors on average react negatively to negative events, while positive events has no impact. 

We want to elaborate on the relation between sustainability of firms and financial performance, by combining the ideas from the previous sections on ESG and sentiment analysis. We will focus on news articles that contain company specific information in relation to the United Nations Sustainable Development Goals\footnote{https://www.un.org/development/desa/disabilities/envision2030.html} (SDGs), and research how company valuation responds to such events. 





\subsection{Market theory}




\subsection{Identification of gap in the literature}

