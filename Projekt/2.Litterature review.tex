


The relation between ESG and corporate financial performances (CFP) has received a lot of attention within the academic literature in the last couple of decades. A comprehensive meta-analysis conducted by \cite{ESG_meta_analysis} investigates more than 2200 individual studies on the relation between corporate social responsibility (CSR) and CFP in the period 1970-2014. The results of the studies differ depending on which ESG methodology - source and granularity - were used and which financial metrics were applied. They find that the orientation toward long-term responsible investing should be important for all rational investors, as the majority of studies find stable and positive ESG impact on corporate financial performance over time. 

A large part of the studies in the meta analysis applies ESG ratings, or similar measures, to account for CSR. Such analyses are great when it comes to determining a causal relation of how shareholders evaluate corporate sustainability and performance in their investment decisions over a long horizon. However, we do not gain understanding of how investors react to sudden changes in beliefs. Several research papers have contributed to the empirical understanding of the shareholder reaction to various extreme positive and negative events related to CSR. The negative financial repercussions of oil spills, toxic releases, child labor, etc., are well established. The literature is lacking research on how shareholders react to ordinary events. 

The initial research on the shareholder reaction from ESG news was ventured by \cite{klassen1996impact} in a cross-sectional event study analysis of environmental management. They examined 22 environmental negative events and 140 positive ones in the period spanning 1985-1991 for companies listed in the U.S. On average, the companies experienced abnormal returns of -1.5\% from negative events, while positive events generated abnormal returns of 0.82\%. 

\cite{kruger2015corporate} extends the approach from \citeauthor{klassen1996impact} and is the first paper to examine the relation between CSR and corporate performance on all CSR dimensions and a large database of events. They consider 2.116 negative and positive ESG events from 745 distinct firms between 2001 and 2007. The results confirms the initial findings from \citeauthor{klassen1996impact}, as negative events are pursued by a decline in market share, whereas the effect from good news is only positive if stakeholder relations are well managed. One of their key contributions is the content analysis which shows that shareholders react more strongly to ESG news involving strong economic or legal information. 

\citeauthor{Blancard_ESG_sentiment} investigate the same relationship over the period 2002-1010, however they utilize a much broader database of around 33.000 positive or negative ESG events, which allows to examine both extreme and ordinary events. They conduct an event study and measure abnormal performance through the Market Model. The average investor reacts negatively to bad events, while positive events has no impact. As a new feature, their regression approach indicate a mitigation of loss from sector ESG reputation, and that cultural distance play an important role. 

These results are both intriguing and well documented. However, the primary limitation of these studies resides in the contents of the utilized samples. The researchers heavily depend on hand collected and validated events sourced from well established databases. This method will in some cases restricts the sample size and, more importantly, inherently constrain the scope of research to events that have desirable characteristics and outcomes ex-post, while inadvertently excluding non-events that might have held significance for shareholders at the time of occurrence. The approaches mentioned so far make use of methodologies that are are a great fit for academic work and empirical relations. However, for the outcome of the research to be applicable in practice, one needs to measure the relation between knowledge available to market participants at the time of occurrence. 





\subsection{Identification of gap in the literature}


I want to elaborate on the relation between sustainability of firms and financial performance, by combining the ideas from the previous sections on ESG and sentiment analysis. We will focus on news articles that contain company specific information in relation to the United Nations Sustainable Development Goals\footnote{https://www.un.org/development/desa/disabilities/envision2030.html} (SDGs), and research how company valuation responds to such events. 


\textbf{I use a data source that also works from a practical perspective. Most other data sources use events which are identified as actual events and handpicked by analysts. However, the data I use from Matter is based 100\% on artificial intelligence and is updated on a daily basis, which means that a potential portfolio manager can generate replicate the analysis that I make.} 