


The relation between ESG and corporate financial performances (CFP) has received a lot of attention within the academic literature in the last couple of decades with varying conclusions. A comprehensive meta-analysis conducted by \cite{ESG_meta_analysis} investigates more than 2200 individual studies on the relation between corporate social responsibility (CSR) and CFP in the period 1970-2014. The results of the studies differ depending on which ESG methodology, source, and data granularity were used and which financial metrics were applied. They find that the orientation toward long-term responsible investing should be important for all rational investors, as the majority of studies find a positive impact of ESG on corporate financial performance over time, with less than 10\% of studies finding a negative relations. A large part of the studies in the meta analysis applies ESG ratings, or similar measures, to account for CSR. Such analyses are beneficial when determining a causal long horizon relation between shareholder investment preferences and corporate sustainability. However, such methodologies does not examine how investors react to sudden changes in beliefs. The granularity of the ESG ratings are simply too infrequent. 

Several research papers have contributed to the empirical understanding of the shareholder reaction to various positive and negative events related to CSR. The negative financial repercussions of oil spills, toxic releases, child labor, and other extreme events are well established. The initial research on the shareholder reaction from ESG news was ventured by \cite{klassen1996impact} in a cross-sectional event study analysis of environmental management. They examined 22 environmental negative events and 140 positive ones in the period spanning 1985-1991 for companies listed in the U.S. On average, the companies experienced abnormal returns of -1.5\% from negative events, while positive events generated abnormal returns of 0.82\%. 

Recent studies examine both positive and negative occurrences. \cite{flammer2013corporate} extends the approach from \citeauthor{klassen1996impact} and finds evidence of positive responses in the United States to the disclosure of environmentally friendly initiatives and negative reactions to announcements of environmentally harmful actions. Flammer further notes that the intensity of positive responses has diminished over time, while negative responses have become more pronounced. These findings are attributed to the growing pressure on companies to adopt environmentally sustainable practices. \cite{kruger2015corporate} conducts a similar study, and is the first paper to examine the relation between CSR and corporate performance on all dimensions of ESG and with a large database of events. They consider 2.116 negative and positive ESG events from 745 distinct firms between 2001 and 2007. The results confirms the findings from the other researchers. One of their key contributions is a content analysis, which shows that shareholders react more strongly to ESG news involving strong economic or legal information. 

\cite{Blancard_ESG_sentiment} construct the most comprehensive study on the subject so far with an investigation over the period 2002-2010. However they utilize a much broader database of around 33.000 positive or negative ESG events, which allows to examine both extreme and ordinary events with more in-depth inference. They conduct an event study and measure abnormal performance through the Market Model. The average investor reacts negatively to bad events, while positive events has no impact. As a new feature, their regression approach along with the richness of the database allows the authors to investigate a broad set of hypotheses. The results indicate that the market value loss from negative events are mitigated from the sector's ESG reputation. Moreover, market participants only react to information disclosed by the media, meaning that investors are not fooled by window dressing in company announcements. Finally, it appears that the loss is larger when there is a cultural proximity between shareholders and the event. 

These results are both intriguing and well documented. However, the primary limitation of these studies resides in the contents of the utilized samples. The researchers heavily depend on hand collected and validated events sourced from well established databases. This method will in some cases restricts the sample size and, more importantly, inherently constrain the scope of research to events that have desirable characteristics and outcomes ex-post, while inadvertently excluding non-events that might have held significance for shareholders at the time of occurrence. The approaches mentioned so far make use of methodologies that are a great fit for academic work and empirical relations. However, for the outcome of the research to be applicable in practice, one needs to measure the relation between knowledge available to market participants at the time of occurrence. 

Following this logic, I apply the volume of news articles as a proxy of the severity of negative and positive events. Newspapers have a significant role in distributing information to a wide range of readers, particularly individual investors. Our approach evolves around a popular core of literature that links public sentiment, through news articles, with financial performance on the stock market. One of the early comprehensive publications are \cite{tetlock_sentiment}, who measure the interactions by daily contents from a \textit{Wall Street Journal} column. The paper shows that high media pessimism predicts downward pressure on stock prices followed by a reversion to fundamentals. Accordingly, the media content from a single newspaper may influence the sentiment but is not a proxy for fundamental asset values. However, \cite{fang2009media} argues that the mass media can affect security pricing by reaching a broad population of investors. They find that stocks with no media coverage earn higher abnormal returns than their popular peers. Hence, the broadness of information dissemination affects stock returns. 

These results imply that I can measure the reactions from shareholders to events by applying spikes in news articles as a proxy for events. 
