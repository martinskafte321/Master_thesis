
\subsection{Data description}


\subsection{Estimation methodology}



\subsection{Short term abnormal returns in the Market Model}

To test hypothesis #1 and #2 of whether negative and positive SDG related events impacts firm value on the short term, I separate negative and positive events and assess the aggregated development in abnormal returns 10 days before and 10 days after an event has occurred. Moreover, I isolate the effect of the individual SDGs to test hypothesis #4 of whether events on specific sustainability goals are more relevant for investors than other. The Market Model is applied to measure abnormal returns around an event.   

\subsubsection{Negative news}

The average abnormal returns and development in cumulative average abnormal returns, inferred from the Market Model, are illustrated in figure \ref{fig:ST_neg_news}. To support the analysis, the AAR and CAAR are portrayed along with their respective 95\% confidence intervals and the amount of events on a given day relative to the event day $(t = 0)$ (right axis) illustrated by the barplot in the background. 

The left y-axis depicts the abnormal return and the x-axis is the number of days before and after an event has occurred. The average effect on returns from negative events is represented by the blue line in the graph. 
The impact from negative events is approximately zero until $t = -5$ upon which the AAR decreases steadily until the event date, where it bottoms at $-0.36\%$. After a negative event has occurred at $t=0$ the AAR surges towards neutrality until $t=2$ and remains there for the rest of the window. The CAAR stays negative during the full event window and bottoms on $t=2$ after a large decline of approximately $1.2\%$ from $t=-5$. Moreover, the AAR is significantly negative between $t=-2$ and $t=0$ at $95\%$. The same goes for the CAAR after $t=-2$ and the remaining window. Generally, negative news seems to be priced in leading up to and including the identified event day, after which they have limited impact. \\ 
The bars offers insights into the extent of public attention leading up to an event. The bars begin to increase at $t=-5$ at which the AAR begins to decline, which imply a relation between attention towards negative news and pessimistic investor behavior even before the event date. Furthermore, this is an indication that the specified event time may be lagging the incident of the *true* event, which is a disadvantage of this model setup.  

\begin{figure} [H]
    \centering
    \caption{Negative news: $CAAR_{t=10}$}
    \includegraphics[scale=0.6]{Projekt/1.Figures analysis/ST_negative_all_CI.png}
     \caption*{\footnotesize The figure illustrates the average abnormal return (AAR) and cumulative AAR (CAAR) around the event date (t = 0) of negative news. The lines (left axis) represent the average and the ribbons represent the 95th confidence intervals. The bars (right axis) represent the amount of events on a given day relative to t = 0. }
    \label{fig:ST_neg_news}
\end{figure} 

Examining the average effect from, respectively, positive and negative events provides insights into the overall relational tendency between shareholder sentiment and corporate sustainability. By investigating the abnormal returns resulting from events specific to the individual SDGs one can gain a deeper understanding of which themes within corporate sustainability that investors place most emphasis with. Figure \ref{fig:ST_neg_bar} illustrates the AAR on $t=0$ from events related to specific SDGs along with corresponding confidence intervals. 

\begin{figure} [H]
    \centering
    \caption{AAR per SDG: negative news}
    \includegraphics[scale=0.6]{Projekt/1.Figures analysis/ST_negative_sdg_bar.png}
    \caption*{\footnotesize The figure illustrates the AAR on $t = 0$ from negative news. The error bars represent the 95\% confidence intervals of the AAR.}
    \label{fig:ST_neg_bar}
\end{figure}

Splitting the events on SDGs generally illustrate that negative events in most groups are associated with negative abnormal returns. However, the statistical uncertainty in the measurements means that one cannot generate a lot of insights from the aggregated values. Only SDG 10, 15, and 16 provides abnormal returns significantly different from zero at a \% level. The statistical uncertainty originates from the relatively low amount of observations associated with the individual groups after dividing the events into SDGs. For example, SDG 4 and 9 has only, respectively, 16 and 83 observations, which results in excessively wide confidence interval, whereas SDG 12 and 16 has, respectively, 576 and 1364 observations resulting in higher statistical certainty.   

The significance of the results in relation to negative and positive events are summarised with a z-test in table \ref{tab: ST_significace}, which presents the z-scores and significance complementary to the AAR on $t=0$, along with the CAAR on respectively, two, five, and 10 days around the event.  

The selection method for negative events was proposed to gather cases of severe negative abnormal returns. The CAAR of the full and subset windows from negative events are significantly different from zero at $p \; < 0.01$, which further validates the selection method, and that negative abnormal returns are attainable to negative events. However, the method may be late to pick up the signal of negative events or new information on the market.  

\begin{table}[ht]
\centering
\caption{AAR and CAAR over event window (in percentage)} 
\begin{tabular}{llllll}
   \hline  \hline
  & $AAR_{t=0}$ & $CAAR_{[-2;+2]}$ & $CAAR_{[-5;+5]}$ & $CAAR_{[-10;+10]}$ & N \\
 \hline
Positive news & $\underset{(1.973)^{**}}{0.073}$  & $\underset{(-0.211)}{-0.016}$    &  $\underset{(-0.379)}{-0.043}$ & $\underset{(-1.091)}{-0.164}$ & 3.500 \\ 
Negative news & $\underset{(-3.457)^{***}}{-0.360}$  & $\underset{(-5.749)^{***}}{-0.917}$    & $\underset{(-4.761)^{***}}{-0.955}$ & $\underset{(-3.288)^{***}}{-0.888}$ & 1.500\\ 
   \hline
   \multicolumn{5}{p{12cm}}{\footnotesize  $^* \; p\; <\; 0.1$, $ ^{**} \; p\; <\; 0.05$, $ ^{***} \; p\; <\; 0.01$  } \\ 
\end{tabular}
\label{tab: ST_significace}
\end{table}

\subsubsection{Positive news}

According to figure \ref{fig:ST_pos_news} the identified positive events doesn't seem to be associated with any significant reaction from the shareholders on average, as the AAR doesn't fluctuate significantly from zero during the event window besides a marginally positive deviation at the event date. Following, the statistical uncertainty of the CAAR becomes more widespread through the event window, as indicated by the widening red ribbon. The positive tendency of the CAAR is insignificant since the measure is based on insignificant AARs, meaning that one cannot for certain state that the cumulative return is different from zero. Nonetheless, a positive reaction is inferred at $t=0$ indicating that investor do react instantly to positive news on average, however the gains seems to be depleted in the following weeks.  

\begin{figure} [H] 
    \centering
    \caption{Short term positive news: AAR and CAAR}
    \includegraphics[scale=0.6]{Projekt/1.Figures analysis/ST_positive_all_CI.png}
    \caption*{\footnotesize The figure illustrates the average abnormal return (AAR) and cumulative AAR (CAAR) around the event date (t = 0) of positive news. The lines (left axis) represent the average and the ribbons represent the 95th confidence intervals. The bars (right axis) represent the amount of events on a given day relative to t = 0. }
    \label{fig:ST_pos_news}
\end{figure}


The tendency of insignificance related to positive events is further reinforced by the results from the individual SDGs, as most returns are statistically indistinguishable from zero. In contrast to the case of negative events, where the main issue was the amount of observations, the underlying issue with uncertainty comes with the seemingly random shareholder reaction to positive events, as indicated by the aggregate overview in figure \ref{fig:ST_pos_news}. Only positive news associated with SDG 7, that deals with clean energy, displays a significant, positive abnormal return. 
 
\begin{figure} [H]
    \centering
    \caption{AAR per SDG: positive news}
    \includegraphics[scale=0.6]{Projekt/1.Figures analysis/ST_positive_sdg_bar.png}
    \caption*{\footnotesize The figure illustrates the AAR on $t = 0$ from positive news. The error bars represent the 95\% confidence intervals of the AAR.}
    \label{fig:ST_pos_bar}
\end{figure}

The insignificance of the abnormal returns associated with positive events are manifested in table \ref{tab: ST_significace}. The the CAARs on, respectively, two, five and 10 days around the event are all insignificantly different from zero. However, the AAR does exhibit positive performance at $t = 0$ at 5\% significance.  

Generally, the short term empirical evidence demonstrate a non-verifiable relation between positive events, related to both broad and specific SDGs, and abnormal returns. Positive news is however associated with an instant reaction from investors on the event date. The evidence point towards an overall negative relationship, but this is not statistically valid.  



\subsection{Long term abnormal returns: The Calendar Time Portfolio}

In contrast to the short term models, where the returns are measured relative to an event date, the long term portfolios are rebalanced each month and includes all firms, that have experienced an event within the T = 1/4/8/12 preceding months. The strategy resembles that T becomes the holding period for event firms. The portfolio returns are estimated over the full sample period of five years, and are evaluated by a regression on the market excess return and the factors from the Fama-French \citeyear{Fama_french_3fac} three-factor model. Full regression statistics including coefficient estimates, r-squared, and standard errors are available in Appendix X. The factor models do generally have high power in explaining the excess return risk exposure, with a r-square coefficient between 73\% to 98\% throughout the various portfolios. The portfolios which hold the least amount of firms on average generally has the lowest r-squared, which generally is a result of increased variance with less observations.  


\subsubsection{Negative news}

The abnormal returns of the portfolios are reflected through the regression intercept from the Fama-French model, presented as the Alpha in table \ref{tab: FF3-neg} along with the corresponding t-value. The table include the mean monthly number of firms in each portfolio as "N". Moreover, the event preconditions are partly disclosed, as the 1, 2 and 3 "sd" gives the required number of standard deviations larger than the mean for an event firm to be identified as consequential and thus selected to the portfolio. Above each horizontal line, the "$T = x$" express the portfolio holding period after an event has occurred. 

In the matter of negative events the alphas are negative for all portfolio specifications. However, they are solely significant in three instances. With a one SD requirement the alpha is significant for holding periods of $T = 1$ and $T = 4$ months at, respectively, -0.82\% at $1\%$ and -0.41\% at $5\%$. For a two SD requirement the alpha with $T=4$ month holding period is significant with an alpha of -0.5\% at $5\%$. The alphas for holding periods larger than $T = 4$ are closer to zero, which is in support of market efficiency over the long term. An implication of the strategy is that portfolios with longer holding periods will accordingly include more firms in each rebalancing, which ultimately will drive the portfolio return closer to the market return. For example with 1 SD, we see the inverse relation between the increases in the average N and the correspondingly decreases in alpha. In contrast, a higher SD requirement is equivalent to a more strict selection procedure designating that fewer firms will enter the portfolio. A consequence of the strict requirements is the size of the portfolios. A portfolio consisting of 5 or 10 firms will inevitably be exposed to idiosyncratic risks from the individual companies, which means that the portfolio outcome may be driven by specific risks rather than the effect from negative news. A large portfolio consisting of, e.g., 60 firms will not encounter the same risks due to diversification benefits, which enable one to approximately isolate the effect from negative news. 

\setlength{\tabcolsep}{15pt}
\begin{table}[h]
\small
\centering
\caption{Fama-French five-factor model alpha from negative news } 
\begin{tabular}{llllllc}
\hline \hline \\ 
 &     &  &    1 SD  &  2 SD  &  3 SD  &  \\ \cline{4-6} 
& & & \multicolumn{3}{c}{ T = 1} & \\ \cline{2-6}
& Alpha (\%)  &  & -0.84^{***}  & -0.47  & -0.78 &  \\ 
& t-value &  & -2.97 & -1.33  & -1.63 &  \\
&  N       &  &  19    & 10  & 5  & \\
& & & \multicolumn{3}{c}{ T = 4} & \\ \cline{2-6}
& Alpha (\%)  &  & -0.36^{**}  & -0.45^{**}  &  -0.27 & \\
& t-value &  & -2.23 & -2.12  & -1.04 & \\
&  N       &  & 60      & 35  & 18  & \\
& & & \multicolumn{3}{c}{ T = 8} & \\ \cline{2-6}
& Alpha (\%)   &  & -0.15   & -0.14  & -0.12 &  \\
& t-value &  & -1.21  & -0.89 & -0.68 &  \\
& N       &  & 100 & 63   & 34 & \\
& & & \multicolumn{3}{c}{ T = 12} & \\ \cline{2-6}
& Alpha (\%)   &  & -0.08  & -0.16  & -0.17 &  \\
& t-value &  & -0.86  & -1.16 & -0.97 & \\
& N       &  & 130    & 86  & 48 \\\hline \hline
 \multicolumn{7}{l}{ \footnotesize $^* \; p\; <\; 0.1$, $ ^{**} \; p\; <\; 0.05$, $ ^{***} \; p\; <\; 0.01$  } \\
 \multicolumn{7}{p{11.5cm}}{ \footnotesize Alpha is the WLS-regression intercept (in \%) of the Fama-French 3-factor model, displayed along with the corresponding t-value. N is the average amount of firms included in the portfolio each month, and T is the portfolio holding period. The threshold for event firms to be included in the portfolio is either 1,2 or 3 "SD" (standard deviations) larger than the mean.} \\ 
\end{tabular}
\label{tab: FF3-neg}
\end{table}

Overall, the relation between negative news and returns are most severe within 1-4 months after an event has occurred and with a "loose" selection criteria of 1 SD for event firms to be included in the portfolio. Although most of the alphas are insignificant it is important to note that all are negative, which support the relation between negative news and pessimistic investor behavior.  
??. 


\subsubsection{Positive news}

The impact from positive news on financial performance in the long run is illustrated in table \ref{tab: FF3-pos}. The table applies the same format as table \ref{tab: FF3-neg}. Positive events have no significant impact on returns over any of the portfolio and event constructions. Nevertheless, the long run effect is predominantly negative, which appears in contrast to the short term effect, that indicate a positive relation, nonetheless insignificant. Surprisingly, the performance is most negative when applying a brief holding period of $T=1$ and a strict rule of events with $SD = 3$, which generates alpha of -0.42\%. With $T=1$ and $SD=1$ the portfolio achieves an alpha of -0.3\%.  

Applying holding periods longer than $T=1$ generates exclusively alphas close to zero, which resembles a well-diversified portfolio return close to the market portfolio, which is an expected result if positive news are recognized as irrelevant. 

The 

\setlength{\tabcolsep}{15pt}
\begin{table}[]
\small
\centering
\caption{Fama-French 3 factor model alpha from positive news } 
\begin{tabular}{ccccccc}
\hline \hline \\ 
 &     &  &    1 SD  &  2 SD  &  3 SD  &  \\ \cline{4-6} 
& & & \multicolumn{3}{c}{ T = 1} & \\ \cline{2-6}
& Alpha (\%)  &  & -0.30  & -0.05  & -0.42 &  \\
& t-value &  & -1.19 & -0.14  & -0.89 & \\
&  N       &  &  34    & 15 & 7 &\\
& & & \multicolumn{3}{c}{ T = 4} & \\ \cline{2-6}
& Alpha (\%)  &  & -0.07  & -0.02  &  -0.09 & \\
& t-value &  & -0.49 & -0.08  & -0.32 & \\
& N       &  & 105     & 52  & 25 & \\
& & & \multicolumn{3}{c}{ T = 8} & \\ \cline{2-6}
& Alpha (\%)  &  & -0.04   & -0.01  & -0.06 &  \\
& t-value &  & -0.41  & -0.57 & -0.29 & \\
& N       &  & 164 & 63   & 47 & \\
&  & & \multicolumn{3}{c}{ T = 12} & \\ \cline{2-6}
& Alpha (\%)  &  & -0.01  & 0.04  & 0.08 &  \\
& t-value &  & -1.03  & 0.33 & 0.47 & \\
& N       &  & 240    & 122  & 67 & \\ \hline \hline
 \multicolumn{7}{l}{ \footnotesize $^* \; p\; <\; 0.1$, $ ^{**} \; p\; <\; 0.05$, $ ^{***} \; p\; <\; 0.01$  } \\
 \multicolumn{7}{p{11.5cm}}{ \footnotesize Alpha is the WLS-regression intercept (in \%) of the Fama-French 3-factor model, displayed along with the corresponding t-value. N is the average amount of firms included in the portfolio each month, and T is the portfolio holding period. The threshold for event firms to be included in the portfolio is either 1,2 or 3 "SD" (standard deviations) larger than the mean.}  \\ 
\end{tabular}
\label{tab: FF3-pos}
\end{table}

