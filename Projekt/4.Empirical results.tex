
\subsection{Data description}


\subsection{Estimation methodology}

\subsection{Abnormal returns in the market adjusted model}


\subsection{Short term abnormal returns in the Market Model}

To test hypothesis #1 and #2 of whether negative and positive SDG related events impacts firm value on the short term, we set apart negative and positive events and assess the aggregated development in abnormal returns 10 days before and 10 days after an event. Moreover, we isolate the effect of the individual SDGs to test hypothesis #4 of whether events on some sustainability goals are more relevant for investors than other. We apply the Market Model to measure abnormal returns around an event.   

\subsubsection{Positive news}

 
The average abnormal returns and cumulative average abnormal returns retrieved from the Market Model are illustrated in table \ref{ST_tab}. The development in AAR and CAAR, along with its corresponding standard error bands with a confidence interval of 95\%, is portrayed in figure \ref{fig:ST_pos_news} to support the analysis from a visual perspective. The y-axis depicts the abnormal return and the x-axis before and after an event. The effect of positive events on average stock behavior is presented by the blue line in the graph, and is mostly positive leading up to an event, which indicates a leakage effect. The positive AAR is especially pronounced between $t =-8$ and $t=-4$ before an event, and with a spike on the event date. However, when adjusting for the expected errors with assistance from the error bands, the AAR is only significantly positive and different from zero on $t= 0$ with an average abnormal return of 0.07\%. 

\begin{figure} [H] 
    \centering
    \includegraphics[scale=0.6]{Projekt/1.Figures analysis/ST_positive_all_CI.png}
    \caption{Short term positive news: AAR and CAAR}
    \label{fig:ST_pos_news}
\end{figure}

The positive AAR provides basis for a significantly positive and increasing trend in the CAAR (red line) leading up to and including the event date. However, at $t = 1$ and all subsequent days the AARs are slightly negative leading to a downward-sloping trend in CAAR, which eventually ends in significantly negative territory on $t = 10$ at $-0.16\%$.

Examining the average effect from, respectively, positive and negative events provides insights into the overall tendency of the relation between shareholder sentiment and corporate sustainability. By investigating the abnormal returns resulting from events specific to the individual SDGs one can gain a deeper understanding of which themes in corporate sustainability that investors places most emphasis with. Figure \ref{fig:ST_pos_bar} illustrates the aggregated CAAR over the full event window (from $t=-10$ to $t=10$) from events related to specific SDGs along with the amount of observed events. In general, positive news in relation to the SDGs seems to be priced positively preceding an event, whereas the subsequent reaction is negative. 

The tendency of positive events from above is further reinforced by the results from the individual SDGs, as most are offering negative CAAR over the event window. The SDGs with the most negative abnormal returns from positive events are goals 1 (no hunger), 14 (life below water), 15 (life on land), and 16 (peace, justice, and strong institutions). However, three of the groups are also among the SDGs subject to the highest statistical uncertainty related to the estimate, as indicated by the confidence intervals (alpha = 0.05) on the bars. Only SDG 16 has a large negative CAAR with low uncertainty. The relatively wide range of many of the intervals highlight a potential issues with splitting the events on SDGs as some of the groups will have high uncertainty due to a low amount of observations. For example, the large negative CAARs of SDG 1 and 14 are based on only 190 and 206 events. In contrast, the groups with a CAAR close to neutrality seems to be with low uncertainty.  
Only positive events involving SDG 8 (decent work and economic growth) are associated with significant short term positive abnormal returns. 
 
\begin{figure} [H]
    \centering
    \includegraphics[scale=0.6]{Projekt/1.Figures analysis/ST_positive_sdg_bar.png}
    \caption{$CAAR_{t=10}$: short term positive events}
    \label{fig:ST_pos_bar}
\end{figure}


% latex table generated in R 4.2.2 by xtable 1.8-4 package
% Sat Mar 11 14:44:17 2023
\begin{table}[ht]
\centering
\begin{tabular}{lrrrr}
  \hline
   & \multicolumn{2}{c}{Positive news} & \multicolumn{2}{c}{Negative news}  \\
  Time & AAR & CAAR & AAR & CAAR \\
 \hline
-10 & -0.0368 & -0.0368 & -0.0539 & -0.0539 \\ 
  -9 & -0.0465 & -0.0833 & 0.0196 & -0.0343 \\ 
  -8 & 0.0454 & -0.0380 & 0.0267 & -0.0077 \\ 
  -7 & 0.0101 & -0.0279 & -0.0042 & -0.0118 \\ 
  -6 & 0.0368 & 0.0089 & -0.0082 & -0.0200 \\ 
  -5 & -0.0037 & 0.0052 & 0.0052 & -0.0148 \\ 
  -4 & 0.0300 & 0.0351 & -0.0910 & -0.1059 \\ 
  -3 & 0.0085 & 0.0437 & -0.0930 & -0.1989 \\ 
  -2 & -0.0204 & 0.0233 & -0.2364 & -0.4353 \\ 
  -1 & 0.0113 & 0.0345 & -0.2184 & -0.6536 \\ 
  0 & 0.0735 & 0.1081 & -0.3604 & -1.0141 \\ 
  +1 & -0.0296 & 0.0785 & -0.0833 & -1.0974 \\ 
  +2 & -0.0496 & 0.0288 & -0.0232 & -1.1207 \\ 
  +3 & -0.0440 & -0.0151 & 0.1148 & -1.0059 \\ 
  +4 & -0.0174 & -0.0325 & 0.0313 & -0.9745 \\ 
  +5 & -0.0002 & -0.0328 & -0.0087 & -0.9833 \\ 
  +6 & -0.0466 & -0.0793 & 0.0385 & -0.9448 \\ 
  +7 & -0.0492 & -0.1285 & 0.0200 & -0.9248 \\ 
  +8 & 0.0000 & -0.1285 & 0.0261 & -0.8987 \\ 
  +9 & -0.0127 & -0.1412 & -0.0191 & -0.9178 \\ 
  +10 & -0.0241 & -0.1654 & 0.0185 & -0.8993 \\ 
   \hline
\end{tabular}
\caption{AAR and CAAR over event window (in percentage)} 
\end{table}

 \label{ST_tab}

\subsubsection{Negative news}

The effect from negative events is clearly more influential on shareholder sentiment than positive events. Figure \ref{fig:ST_pos_news} illustrates the development in AAR and CAAR. The impact from negative events is approximately zero until $t = -6$ upon which the AAR decreases steadily until $t=0$, where it bottoms at $-0.36\%$. Subsequently, after the negative event at $t=0$ the AAR increases towards neutrality at zero and remains there for the rest of the window. The CAAR stays negative during the full event window and bottoms at $t=2$ after a large decline of approximately $1.2\%$ from  $t=-5$.
Moreover, the AAR is significantly negative between $t=-2$ and $t=0$ at $95\%$. The same goes for the CAAR after $t=-2$ and the remaining window. Generally, negative news seems to be priced in leading up to and including the identified event day, after which they have limited impact. 

\begin{figure} [H]
    \centering
    \includegraphics[scale=0.6]{Projekt/1.Figures analysis/ST_negative_all_CI.png}
    \caption{Short term negative news: AAR and CAAR}
    \label{fig:ST_neg_news}
\end{figure}

Splitting the events on SDGs, as presented in figure \ref{fig:ST_neg_bar}, illustrate that negative events for most groups leads to significantly negative abnormal returns, as expected. Specifically, SDG 6, 9, 14, and 15 deliver a CAAR of at least $-1\%$. The statistical uncertainty on CAAR from negative events are larger than for positive events due to fewer observation in the measure.   


\begin{figure} [H]
    \centering
    \includegraphics[scale=0.6]{Projekt/1.Figures analysis/ST_negative_sdg_bar.png}
    \caption{CAAR short term negative news}
    \label{fig:ST_neg_bar}
\end{figure}




