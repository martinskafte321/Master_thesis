
This chapter is divided into three sections. First, in section \ref{sec: short_term_analysis} I provide a definite answer to the hypotheses on short- and long-term performance from events related to the overall Social Development Goals, and to demonstrate the benefits of the sensitivity analysis. The foundation is the reviewed results from sections \ref{sec:results} and \ref{sec:sensitivity}. 
Subsequently, the analysis shifts its focus towards building intuition and understanding regarding the sub-questions. Thud, in section \ref{sec: short_term_analysis_SDG}, I aim to determine whether investors place more emphasis on specific themes within sustainability. Finally, section \ref{ESG_reputation} delves into the relevance of firms' ESG risk characteristics when investors react to events. Throughout the chapter, I aim to first describe what happens empirically, then explain the reason it happens, and finally what other papers has found. 

\subsection{The Impact of General SDG News on Stock Prices} \label{sec: short_term_analysis}
To answer the hypotheses 1 and 2, I assess the statistical significance from the short- and long term results. After concluding remarks on the hypotheses, each subsection will present some intuitive interpretation and explanations from related research to the specific outcomes.

\subsubsection{Short Term Hypothesis} 

The null hypothesis assumes that there is no market reaction to news related to the SDGs within a short horizon. Any significant deviation from the expected returns suggest that sustainability-related news has a notable influence on firm's market values.  

The significance of the results obtained from the Market Model is evaluated with a z-test. The z-statistics and corresponding significance levels, complementary to the AAR on $t=0$ and the CAAR 21 days surrounding the event, are presented in table \ref{tab: ST_neg_significance} for negative events and table \ref{tab: ST_pos_significance} for positive ones. The values presented in the tables are equivalent to the graphics in figures \ref{fig:ST_neg_news} and \ref{fig:ST_pos_news}. 

According to table \ref{tab: ST_neg_significance} negative SDG disclosures impact firms' market values significantly adverse not only on the event date but also over a 21-day event window. In support of hypothesis 1.a the cumulative abnormal return is $-0.72\%$ throughout the 21-day event window. The observed impact of negative events is substantial and significantly below zero at the 1\% level of significance. As a result, I reject the hypothesis of no abnormal returns from negative events, and accept the alternative that negative events are associated with negative abnormal returns on average. 

Regarding positive news, the CAAR around a 21-day event window is slightly negative and insignificant with a value of -0.02\%, as illustrated in the "Overall" column in table \ref{tab: ST_pos_significance}. Hence, there is insufficient evidence to reject the null hypothesis, and positive news related to the SDGs does not have a short term impact on firm's market values. Nonetheless, the AAR on the event date is significantly positive at the 1\% level with a value of 0.05\%, which suggest that a spike in positive news is instantly rewarded with a small increase in market value. However, this reaction alone is not sufficient to establish general intuition.

\noindent Accordingly, shareholders seem to penalize negative corporate social responsibility, while they do not really reward positive practices. While this is not a novel finding, the results of this paper provide evidence that shareholders do indeed integrate general information related to the Sustainable Development Goals in their investment decisions. 

\begin{table}[H]
\centering
\caption{Negative news: AAR and CAAR on overall and ESG risk level} 
\begin{tabular}{ccccc}
  \hline  \hline
  & \multicolumn{1}{c}{Overall} &  \multicolumn{1}{c}{Low} & \multicolumn{1}{c}{Medium} & \multicolumn{1}{c}{High}\\  
 \hline
$AAR_{t=0}$ &   $\underset{(-2.25)}{-0.17^{**}}$ &   $\underset{(-2.67)}{-0.22^{***}}$ &   $\underset{(-1.18)}{-0.13}$ &   $\underset{(-0.20}{-0.67 }$ \\
$CAAR_{[-10;+10]}$    & $\underset{(-3.21)}{-0.72^{***}}$ &   $\underset{(-4.49)}{-1.65^{***}}$ &   $\underset{(-1.80)}{-0.54^{**}}$ &   $\underset{(-0.38)}{-0.25}$ \\ 
   \hline \hline
   \multicolumn{5}{p{12cm}}{ \footnotesize $^* \; p\; <\; 0.1$, $ ^{**} \; p\; <\; 0.05$, $ ^{***} \; p\; <\; 0.01$  } \\
   \multicolumn{5}{p{13cm}}{\footnotesize The tables shows the CAAR and associated t-value related to positive and negative events over an event window of 10, and 21 days surrounding the event date along with the AAR on $t=0$. Negative events consists of 1046 observations. } \\
   \hline
\end{tabular}
\label{tab: ST_neg_significance}
\end{table}

The findings are in line with most of the previous literature. \cite{Blancard_ESG_sentiment} and \citep{kruger2015corporate} finds similar evidence of a short term pessimistic market reaction from negative news by applying the event study methodology and the Market Model as well. \citeauthor{Blancard_ESG_sentiment} identify slightly positive, but insignificant, abnormal return for positive events as well, whereas \citeauthor{kruger2015corporate} finds an ambivalent association that depends on the quality of the relation between the firms and their stakeholders. 

\noindent \textbf{A general note on the methods.}\\
The Market Model appears to be a good choice for measuring abnormal returns in relation to negative news. In figure \ref{fig:ST_neg_news} the AAR is hovering around 0\% in the first four days of the window and again after the event has occurred at $t = 0$, implying that the expected returns are effective in reflecting the realized returns. The lack of a clear pattern in the reactions to positive news may be more indicative of the relative negligible impact of the news factor rather than criticism of the model.    

\begin{table}[H]
\centering
\caption{Positive news: AAR and CAAR (in \%) on overall and ESG risk level} 
\begin{tabular}{ccccc}
  \hline  \hline
  & \multicolumn{1}{c}{Overall} &  \multicolumn{1}{c}{Low} & \multicolumn{1}{c}{Medium} & \multicolumn{1}{c}{High}\\  
 \hline
$AAR_{t=0}$ &  $\underset{(3.22)}{0.05^{***}}$ & $\underset{(0.37)}{0.03}$ & $\underset{(1.05)}{0.08}$ &  $\underset{(0.20)}{0.16}$ \\ 
$CAAR_{[-10;+10]}$    & $\underset{(-0.23)}{-0.02}$ &  $\underset{(-2.25)}{-0.23^{***}}$ &  $\underset{(1.86)}{0.24^{**}}$ &  $\underset{(-1.04)}{-0.34}$ \\ 
    \hline \hline
   \multicolumn{5}{p{12.5cm}}{ \footnotesize $^* \; p\; <\; 0.1$, $ ^{**} \; p\; <\; 0.05$, $ ^{***} \; p\; <\; 0.01$  } \\
   \multicolumn{5}{p{13cm}}{\footnotesize The tables shows the CAAR and associated t-value related to positive and negative events over an event window of 5, 10, and 21 days surrounding the event date along with the AAR on $t=0$. Positive and negative events consists of, respectively, 3564 and 1046 observations. } \\
   \hline
\end{tabular}
\label{tab: ST_pos_significance}
\end{table}

Likewise, the event study methodology works adequately at identifying negative events from spikes in news, as an evident investor reaction is recognized through the abnormal returns. However, the investor reaction materializes prior to the actual identification of events ($t=0$), which is a drawback of the event selection procedure. This issue arises since the underlying data set contains the frequency of firms being mentioned in news articles globally, rather than the precise timing of specific events. Consequently, the event identification procedure relies on detecting the largest events based on volume. As a results, the methodology may not fully capture the initial market response to "breaking" news, which is typically reflected in early investors reactions. These early reactions, characterized by significant drawdowns, are apparent in figure \ref{fig:ST_neg_news} between $t = -6$ and $t = 0$. 

The incremental increases in the bars, from $t=-6$ preceding an event, indicates that, on average, the initial information regarding a particular event tends to be published approximately six business days prior to the event gaining significant attention from the outstanding media and experiences a spike in news coverage. During the period between $t=-6$ and $t=0$ investors gradually become aware of the event and react accordingly based on the information they acquire. 

To quickly sum up, the evidence demonstrates that negative news is penalized with an adverse shareholder reaction in the short term, whereas positive news has no impact.  


\subsubsection{Long Term Hypothesis} \label{sec: long_term_analysis}

The hypotheses 2.a and 2.b aim to answer whether SDG-related news are associated with changes in firms' market values on long horizons. I assess the statistical significance of the alpha generated from portfolios, which hold firms that have experienced a positive or negative event within T months. 

The regressions reveal significant monthly alphas of -0.84\% and -0.36\% from negative news portfolios with holding periods of one and four months, respectively. With these results I reject hypothesis 2.a, which implicate that negative news has a long term adverse impact, of at least four months, on the average firm's market value. 

The portfolios based on positive events do not generate significant alpha across any of the holding periods. Hence, I fail to reject hypothesis 2.b that positive news related to the SDGs does not have a long term impact on market values.


\subsubsection{Threshold Values}

The benefits from the sensitivity analysis are two-fold. First, the results validate the robustness of the initial findings. Second, it can broaden our knowledge on the drivers behind the results. 

Section \ref{sec: sens_st_sd} establish the impact of tightening the event threshold on the short-term abnormal returns. A more strict threshold directly affects which events get identified as important. Thus, the observed decrease in abnormal returns in figure \ref{fig:ST_neg_sensitivity} is not in line with expectations. However, there are no theoretical underpinnings of the result and it appears to be mostly an empirical issue in the sample. It appears that the events receiving the most public attention may not necessarily coincide with the events that investors value the most. 
The implications are similar for the long-term abnormal returns as demonstrated in table \ref{tab: FF5_sensitivity}. However, all the estimated alpha value becomes insignificant when tightening the threshold. 

On a side note, the monthly average amount of firms in the portfolios decrease by roughly half as the threshold is tightened by one standard deviation. For instance, with a holding period of one month, the portfolio with a threshold of one standard deviation consists of 18 firms on average, while tightening the threshold to two standard deviations reduces the number to 9. While the portfolios are sorted on negative events in order to capture a consequent "news-factor", portfolios with fewer firms are inevitably exposed to higher levels of idiosyncratic risks from individual companies. Therefore, the likelihood, that the portfolio alpha is driven by specific company risks rather than a response to negative news, increases. The analysis conducted in this paper relies on diversification benefits achieved through rebalancing a random selection of stocks on a monthly basis. This approach allows to approximately isolate the effect of negative news on stock returns. Thus, reducing the number of firms will introduce certain drawbacks. 

Anyway, based on the sign and magnitude of abnormal returns from the altered portfolio constructions, it appears that the original results are robust and reflect the relation between negative events and investor reactions. 

\subsubsection{Changing Portfolio Weights}

Section \ref{sec: sens_st_weights} demonstrates that using equal weights, which applies relatively more weight to smaller firms, increase the short-term abnormal returns through the period $t = -10$ to $t = 0$. That outcome may seem unintuitive in this relation. 

On the one hand, larger firms are anticipated to attract more media attention. Presumably, with greater news coverage, the shareholder reaction is expected to be more severe due to two factors. First, the news may reach a larger potential investor base. Second, investors receive comprehensive and more detailed information about the specific companies. 

On the other hand, the severe reaction among small stocks may be attributed to an overreaction when new information becomes available. This explanation seems plausible as the market value loss reverts, to the same level as the value weighted portfolio, in the days following the event, where more information becomes available. Anyhow, these explanations are not mutually exclusive. 

The specific results discussed in this section are backed by the early research on media sentiment from \cite{tetlock_sentiment}. The author finds a larger price impact on small stocks relative to larger stocks from ordinary negative news. Our methodologies are not alike, however. My analysis computes returns from companies based on the relative volume of firm-specific news, which may suppress small firms when applying value weights. \citeauthor{tetlock_sentiment}, on the other hand, calculates the overall sentiment from a \textit{Wall Street Journal} column and measures the effect on a cross section of stocks, where volatility to general sentiment plays a large role for small stocks. 

The findings from the long-term portfolios support this postulate, as relatively higher weight toward small stocks takes part in driving the alphas of the equal weighted portfolio lower for all holding periods. Regarding the long-term portfolios, the inherently higher volatility of small stocks will probably play a similar role \citep{Fama_french_3fac}. 

Moreover, the discrepancies between the outcomes from using different portfolio weights may be a feature of the Fama-French regressions. Tables \ref{tab: summary_neg_1} and \ref{tab: summary_neg_EW} are present in the appendix, and show the sensitivities to the risk factors of the value and equal weighted portfolios, respectively, with a holding period of one month. Both portfolios have a positive and significant beta-coefficients (betas) to the market excess return in the range of 1.15-1.2 and to high-minus-low (value) of 0.31 and 0.22 for value and equal weights, respectively, while the remaining factors are insignificant. Thus, in these aspects, the portfolios are quite similar. However, when shifting from value weights to equal weights, and thus allocates more weight to smaller stocks, the beta coefficient of the small-minus-big size factor becomes significant with a value of 0.51. In contrary, the value weighted portfolio has approximately zero sensitivity to the size factor, with a value of -0.04. 

Although the r-squared is high at 0.89, it appears that the factor regression has issues with explaining the returns of the value weighted portfolios, as most of the variation in returns is explained by the market excess return and the value factor. The same issue does not pertain to the equal weighted portfolio, where the size factor explains a large part of the variations. I have tested the portfolios with a Fama-French 3 Factor Model, and they show roughly similar sensitivities to the size and value factors.  



Putting it all together, this section concludes that negative news related to the SDGs are penalized significantly by shareholders at both short horizons of 21 days and on longer horizons of up to four months. In contrast, positive news has no impact on short or long horizons. However, the strategy cannot capitalize on the short term abnormal returns, since the methodology does not capture the breaking news reactions. Moreover,
higher volatility of small stocks is a possible explanation to the short-term investor reactions.

\subsection{News Impact from SDG Pillars} \label{sec: short_term_analysis_SDG}

The question in focus is whether specific dimensions of corporate social responsibility are more important to shareholders. This section goes through the various investor reactions to corporate events related to specific themes within SDGs. The purpose is mainly to provide intuition and explanations from existing research to the results I have obtained. In this section, each paragraph will focus on one of the Five Pillars of SDGs introduced in section \ref{sec:results}.    

The figures \ref{fig:ST_neg_bar} and \ref{fig:ST_pos_bar} illustrate the investor reactions to negative and positive events, respectively, over a 21-day window. Negative news concerning the SGD Pillars Peace, Planet, and Prosperity leads to pessimistic market reactions on average. On the other hand, no Pillar is associated with significant returns with positive news. Given these initial results along with the predominant focus on negative events in existing research, this section will place greater emphasis these topics. 

\textbf{Prosperity.} Negative news related to the Prosperity Pillar demonstrates a CAAR of -0.53\%, whereas positive news is not followed by a significant impact. The Prosperity Pillar is mostly driven by SDG 7 (Affordable and Clean Energy) and 8 (Decent work and Economic Growth) which account for 71\% of the volume in negative events. SDG 7 relates to the global transition to sustainable energy sources, in line with the objectives defined in the Paris Agreement, which describes climate risk as a systematic risk. While SDG 7 is associated with many events, both positive and negative, the corresponding investor reaction remains close to zero. This suggests that while there is a significant public attention to clean energy, these events may be perceived as relatively unimportant from a corporate perspective. A new study from \cite{being_green} shows that neither low-emission or high-emission companies with higher environmental scores perform better financially. According to the the article, companies that do not live up to environmental standards are not necessarily penalized from a profit-maximization perspective.
On a side note, although former research indicates that high-emission, or in other ways low ESG, companies are not penalized financially, they bear the risk of being black-listed by institutional investors with high ESG portfolio requirements, which as an isolated event is expected to increase divestments by other investors further \cite{dell2021norwegian}. 

\textbf{Peace.} The Peace Pillar encompass SDG 16 (Peace, Justice and Strong Institutions) solely. The keywords, peace and strong institutions, are not typically associated with important corporate events and financial performance. Justice, on the other hand, is crucial for both, as the keyword related to various negative corporate events like accounting fraud and market manipulation. Consequently, SDG 16 is linked to more than 1400 negative events, surpassing all other SDGs by a decent margin. Events tied to the SDG also exhibit a large and significant average drop in market value, with a CAAR of approximately -0.57\% throughout the entire window. A short term event study conducted by \cite{bauer2010misdeeds} examining alleged corporate misconduct involving governance-related offenses reveal that investors react negatively to corporate filings before any verdict is published. The study reports an average CAAR of -11.6\% based on a Fama-French 3 factors model over a 21-day window. Given the significant declines observed from actual filings, the investor reaction from spikes in news related to similar events seem appropriate. As mentioned in the literature review in section \ref{lit_rev}, the disparity in methodologies between identifying events based on media activity (as in this case) and hand-picking specific events, as done in \citeauthor{bauer2010misdeeds}) results in discrepancies in results - in this case an average difference of more than 11\%-points. 

\textbf{Planet.} The short term reaction to negative news concerning the Pillar, Planet, generates the highest abnormal return at -0.7\%. The overall theme incorporates development goals related to tangible subjects as climate action, responsible production, and the earth as a whole, which has received much public attention in the 21st century.

The Pillar "Planet" obtains the highest negative abnormal return of -0.7\% from negative events and a relatively high abnormal return of 0.1\% for positive events in the short term. This can be attributed to the pillar's focus on development goals related to tangible subjects such as climate action and responsible production. These issues have received significant public attention in the 21st century, possibly resulting in a heightened investor attention as well. Extensive research has been conducted in this area. For instance, \cite{karpoff2005reputational} find that the average legal penalty for environmental violations is around 2.26\%, and the market penalty is roughly of similar magnitude. Similarly, \cite{capelle2010does} discovers an average market penalty of -1.3\% in terms of market value from industrial disasters. Figure \ref{fig:event_distribution_SDG} shows that, within this theme, most of the identified events are related to responsible production (SDG 12) and climate action (SDG 13). Hence, the SDGs draw a lot of public attention, however investor are mostly reacting to the former when considering negative events. 

\textbf{Partnership.} Among the Five Pillars, People and Partnership stands out with positive, although insignificant, CAAR values of 0.48\% and 0.37\%, respectively, concerning negative events.
The Partnership Pillar focuses solely on SDG 17 (Partnerships for the Goals). The SDG in itself is relatively vague and supposedly difficult to quantify, as the main message is to strengthen the implementation of global partnerships. The underlying catalyst for these events are driven by some firms falsely proclaiming to be sustainable across various aspects. Such strategies, commonly referred to as 'green-washing,' are heavily disapproved of by activist groups and also penalized in financial markets, if discovered. News concerning green-washing is related to SDG 17. Although the average value of the abnormal returns across companies is positive, the wide confidence bands indicate that the average comes with high uncertainty, and thus a large deal of negative reactions.  \cite{Blancard_ESG_sentiment} backs up the argument, that investors punish such misconduct. However, she also notes that successful green-washing can help mitigate the financial penalties from adverse ESG events. Overall, the Partnership Pillar is associated with positive, but uncertain, average abnormal returns. 

\textbf{People.}The People Pillar, which encompasses development goals related to poverty reduction, quality education, gender equality, and more, presents challenges when it comes to quantifying social corporate behavior. As the SDGs within this pillar are more related to supporting developing countries, it is uncommon for regular firms to take a large part, which in part is reflected through the relatively low amount of events for these SDGs, as per figure \ref{fig:event_distribution_SDG}. Thus, immoral acts in this context may be better reflected through long term rather than short term investor reactions. For instance, SDG 3 (Good Health and Well Being) has the second highest number of negative and positive events among the SDGs, however the investor reaction is minimal. The volume of events indicates the media considers well-being of people as a important theme, but investors do not view it as essential from a corporate or profit-maximization perspective.   

Overall, the question of whether SDG news are important for shareholders is clearly dependent on which theme the news is related to. Negative news related to the Pillars Planet, Peace, and Prosperity generate negative abnormal returns at -0.70\%, -0.57\%, and -0.53, respectively, with the two former being significant at 5\%. On the other hand, People and Partnership generates positive abnormal returns of 0.48\% and 0.37\%, respectively. The characteristics of the former groups are more tangible than the latter, which in the end helps in generating more significant investor reactions. Overall, there is clear evidence that investors value certain characteristics within the Social Development Goals higher than others. 

\subsection{Impact of ESG Risk Reputation} \label{ESG_reputation}

Throughout this section I explore whether a firms' sustainability level influences the magnitude of the investor reaction to corporate events. For example, whether a company with low ESG-risk experiences distinct investor reactions to that of a high-risk company in the case of negative news. Following, I present some possible explanations to the observed phenomenons.     

\textbf{Short term.} Generalizing stock returns based on SDG news offer valuable empirical insights into how investors consider corporate sustainability in their decision-making process. However, companies face diverse ESG-related issues and risks, and the external investor pressure they encounter might vary accordingly. From section \ref{sec: st_negative} we learned that European companies with medium-risk experience no short-term reaction to negative event, whereas low ESG risk are penalized heavily when associated with negative events.  

The existing literature is inconclusive of how investor incorporate the impact of ESG reputation. On the one hand, some argue that companies with a positive reputation may experience  more severe investor reactions if they fail to live up to their esteemed status. \cite{noNewsgoodnews} support this claim from a media perspective, as they find that accidents involving companies with an admirable record of corporate sustainability are more likely to be reported in the media. Conversely, other studies like \cite{flammer2013corporate} and \cite{godfrey2009relationship}, argue that companies with a strong history of corporate responsibility are better equipped to mitigate the social pressure from adverse events, resulting in a less dramatic impact on market value. Additionally, \cite{Blancard_ESG_sentiment} point out that a firm's sector ESG reputation can help alleviate the market value loss resulting from negative events.  

The findings, that low-risk companies generate the largest drop in market value from negative events, from section \ref{sec:results} contradict the theories that suggest ESG reputation acts as a protective shield. The specific results on the short-term reaction are summarised with significance levels in table \ref{tab: ST_neg_significance}.

As a natural virtue of the rating, firms with low ESG-risks may be priced with a premium, as these companies are expected to avoid extreme negative events related to corporate sustainability. For instance, a low-risk company may have a relatively higher market valuation due to its current positive sustainability profile. However, if such a company disappoint on ESG-related topics, investors need to adjust their expectations regarding the likelihood of future events and the corresponding impact on the firm's sustainability profile. This adjustment in expectations can possibly lead to a change in market value for the company. 

The disparity between theory and empirical results could potentially be attributed to the media effect, as discussed by \cite{noNewsgoodnews}, with the argument that low-risk firms may receive greater media exposure in the event of negative news, which could amplify the market reactions. With that said, I cannot prove whether the distinctions in market reaction is a consequence of the samples and time periods applied. 

Firms with high ESG risk are not encountering any significant change in market value in response to negative news. If investors already anticipate a higher likelihood of negative events occurring for these high-risk firms, then such risks may be reflected in their market valuations market value apriori. In other words, the market value of these firms may already account for the possibility of adverse ESG-related incidents. Medium-risk firms are experiencing negative abnormal returns, although to a lesser extent compared to low-risk firms. 

The reaction to positive news is less clear cut. Medium-risk firms experience abnormal returns from positive events, as one may expect, while low-risk firms encounter market value losses. I find no theoretical underpinnings for the distinction in abnormal returns between the two groups in the literature. Although, \cite{flammer2013corporate} does report that positive market reactions to eco-friendly events are smaller for companies with low environmental risk. In addition, \cite{fisher2011voluntary} finds that positive ESG announcements have been found to be followed by negative abnormal returns, since these decisions often appear to be in contrast to profit-maximization. The argument is that certain investors perceive interactions with SDGs and ESG as creating additional and unnecessary costs for the firm, which they believe may impact profitability negatively. 

Overall, these findings suggest that short-term market reactions to negative news vary across different ESG-risk categories. High-risk firms have already factored in potential negative events, medium-risk firms experiencing a moderate reaction, and low-risk firms face more significant market value declines.

\textbf{Long term.} The distinction between low and medium ESG risk profiles diminishes over longer horizons. As shown in figure \ref{tab:FF5_neg_ESG}, when applying a one-month holding period to these portfolios, both low and medium risk portfolios generate similar negative abnormal returns of -0.64\% and -0.61\%, respectively. The similarities of the portfolio returns persist over time even as the alphas reduce. While the portfolios may generate approximately equivalent alphas, their characteristics differ significantly in terms of their sensitivities to the Fama-French factors, as illustrated in tables \ref{tab: summary_neg_ESG_L} and \ref{tab: summary_neg_ESG_M} in the appendix. 

Specifically, the difference in sensitivity to the size factor indicate that the reaction to negative events may remain even across firm characteristics. 

The low-risk portfolio has a negative correlation with the size factor, with a beta of -0.28, significant at a 10\% level, whereas the medium risk portfolio has a coefficient of 0.41, significant 5\%. Hence, it appears that the loss in market value from negative events are not strongly affected by the portfolio composition. In other words, the firms loose market value from negative events independent of their size, like we saw in section \ref{sec: sens_st_weights} on portfolio weights.  

The observed short-term pattern for high-risk firms does not generalize to longer horizons, as the sign of abnormal returns is complete opposite for these firms. However, the large degree of uncertainty associated with this sample seems to be general across horizons. The high-risk portfolio has limited data with only 81 observed events and 18 distinct firms throughout the full period, which contributes to significant idiosyncratic risks, and, consequently, very fluctuating alpha values. 

Overall, the main take-away from this section is the phenomena of short term reputational effects from ESG integration in companies. Parts of the existing research describes ESG as a protective shield. However, the results from this paper introduces a clear discrepancy since low-risk companies are penalized the most from negative news.  


In summary, this chapter explores the various perspectives on investor reactions to news related to the SDGs. Initially, in section \ref{sec: short_term_analysis} I conclude that investors react significantly negative to adverse news on general SDGs, both on a short horizon as well as longer ones, which allows to reject hypothesis 1a and 2a. In addition, section \ref{sec: short_term_analysis_SDG} demonstrates how investors place more emphasis with certain themes within the SDGs. Specifically, negative news related to Planet, Peace, and Prosperity exhibit sizable negative abnormal returns. Finally, I discuss how the findings from this paper is in contrast with relevant research in relation to reputational effects within ESG. I highlight that the penalty to low-risk companies from negative news is a natural market reaction based on adjusting future expectations. In order to collect and challenge some of the findings from this chapter, I discuss parts of the assumptions in the methodology and its practical implications in the following chapter. 





