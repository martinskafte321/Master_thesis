\subsection{Short term investor reaction} \label{sec: short_term_analysis}
To answer hypothesis 1 and provide intuition to the sub-questions, I assess the statistical significance of the cumulative average abnormal returns over a window of 21 days around all identified events of negative or positive news in section \ref{ST_results}. The hypothesis regarding short term performance address the immediate investor reaction to events with estimated abnormal returns from the Market Model. \\

The significance of the results from the Market Model are summarised with a z-test. The test statistic is computed and compared to its assumed distribution under the null hypothesis that the abnormal return is zero. The z-scores and significance complementary to AAR on $t=0$ and CAAR on 10 and 21 days around the event are presented in table \ref{tab: ST_neg_significance} for negative news and table \ref{tab: ST_pos_significance} for positive news.  

\subsubsection{The impact of SDG news on stock prices }

The null hypothesis assumes that the market does not react to news related to SDGs in the short horizon. Any significant deviation from the expected returns indicates that news related to sustainability has an observable effect on firm's market values. Rejecting the null means accepting the alternative hypothesis that the short term abnormal return is different from zero. \\
 
Table \ref{tab: ST_neg_significance} shows that negative SDG disclosures impact firms' market values adversely both on the event date, and with a 10 and a 21 day event window. In support of hypothesis 1.a the cumulative abnormal return is $-1.56\%$ throughout the 21-day event window. The magnitude of the impact is high and significantly lower than zero at the 1\% level. Hence, I reject the hypothesis stating that negative events have no impact on abnormal returns, and accept the alternative that negative events are concurrent with negative abnormal returns on average. 

For positive news the cumulative abnormal average change in firm's market value around a 10 and 21-day event window is positive, however highly insignificant. With a CAAR of 0.07\% throughout the full event window, I fail to reject the null hypothesis. Hence, positive news related to the SDGs does not have a short term impact on firm's market values. However, the AAR on the event date is 0.08\% which is significant on the 5\% level, although the single observation is not sufficient to establish general intuition. Thus, positive news are rewarded when news spikes happen, but there is no general pattern in the days around an event.  

Accordingly, shareholders seem to penalize negative corporate social responsibility, while they do not really reward positive practices. 

\begin{table}[H]
\centering
\caption{Negative news: AAR and CAAR on overall and ESG risk level} 
\begin{tabular}{ccccc}
  \hline  \hline
  & \multicolumn{1}{c}{Overall} &  \multicolumn{1}{c}{Low} & \multicolumn{1}{c}{Medium} & \multicolumn{1}{c}{High}\\  
 \hline
$AAR_{t=0}$ &   $\underset{(-3.51)}{-0.40^{***}}$ &   $\underset{(-4.48)}{-0.65^{***}}$ &   $\underset{(-3.41)}{-0.39^{***}}$ &   $\underset{(-0.29)}{-0.14 }$ \\

$CAAR_{[-5;+5]}$  &  $\underset{(-5.46)}{-1.10^{***}}$ &   $\underset{(-6.72)}{-2.25^{***}}$ &   $\underset{(-3.70)}{-1.02^{***}}$ &   $\underset{(0.06)}{0.04}$ \\ 

$CAAR_{[-10;+10]}$    & $\underset{(-5.70)}{-1.56^{***}}$ &   $\underset{(-8.46)}{-4.01^{***}}$ &   $\underset{(-3.14)}{-1.16^{***}}$ &   $\underset{(0.46)}{0.35}$ \\ 
   \hline \hline
   \multicolumn{5}{p{12cm}}{ \footnotesize $^* \; p\; <\; 0.1$, $ ^{**} \; p\; <\; 0.05$, $ ^{***} \; p\; <\; 0.01$  } \\
   \multicolumn{5}{p{13cm}}{\footnotesize The tables shows the CAAR and associated t-value related to positive and negative events over an event window of 10, and 21 days surrounding the event date along with the AAR on $t=0$. Negative events consists of 1046 observations. } \\
   \hline
\end{tabular}
\label{tab: ST_neg_significance}
\end{table}

The findings are in line with most of the previous literature. \cite{Blancard_ESG_sentiment} and \citep{kruger2015corporate} finds similar evidence of a short term pessimistic market reaction from negative news. \citeauthor{Blancard_ESG_sentiment} identify slightly positive, but insignificant, abnormal return for positive events as well, whereas \citeauthor{kruger2015corporate} finds an ambivalent association that depends on the quality of the relation between the firms and their stakeholders. \cite{fisher2011voluntary} even find that companies which make ESG-positive announcement are penalized by shareholders since such actions are in conflict with firm value maximization. The distinction between the results may stem from the event methodology, as I gather all types of news related to SDGs, which may in fact be value maximizing. 

The Market Model resemble a good measure of abnormal returns in relation to negative news. In figure \ref{fig:ST_neg_news} the AAR is hovering around the level of 0\% in the first five days of the window and again after the event has occurred at $t = 0$, implying that the realized returns are approximately as the expected returns. No clear pattern is present regarding positive news, which may be more indicative of the negligibility of the news factor rather than criticism of the models applied.   

\begin{table}[H]
\centering
\caption{Positive news: AAR and CAAR (in \%) on overall and ESG risk level} 
\begin{tabular}{ccccc}
  \hline  \hline
  & \multicolumn{1}{c}{Overall} &  \multicolumn{1}{c}{Low} & \multicolumn{1}{c}{Medium} & \multicolumn{1}{c}{High}\\  
 \hline
$AAR_{t=0}$ &  $\underset{(2.12)}{0.08^{**}}$ & $\underset{(0.16)}{0.00}$ & $\underset{(2.77)}{0.15^{***}}$ &  $\underset{(-0.36)}{-0.05}$ \\ 
$CAAR_{[-5;+5]}$  & $\underset{(0.75)}{0.09}$ &  $\underset{(-1.66)}{-0.26^{*}}$ &  $\underset{(2.28)}{0.39^{**}}$ &  $\underset{(-1.25)}{-0.44}$ \\ 
$CAAR_{[-10;+10]}$    & $\underset{(0.46)}{0.07}$ &  $\underset{(-2.29)}{-0.48}$ &  $\underset{(2.90)}{0.47^{***}}$ &  $\underset{(-0.80)}{-0.34}$ \\ 
    \hline \hline
   \multicolumn{5}{p{10cm}}{ \footnotesize $^* \; p\; <\; 0.1$, $ ^{**} \; p\; <\; 0.05$, $ ^{***} \; p\; <\; 0.01$  } \\
   \multicolumn{5}{p{10cm}}{\footnotesize The tables shows the CAAR and associated t-value related to positive and negative events over an event window of 5, 10, and 21 days surrounding the event date along with the AAR on $t=0$. Positive and negative events consists of, respectively, 3564 and 1046 observations. } \\
   \hline
\end{tabular}
\label{tab: ST_pos_significance}
\end{table}

Likewise, the event study methodology works adequately at identifying negative events from spikes in news, as an evident investor reaction is recognized through the abnormal returns. However, the reaction materializes prior to the identification of events, which is an unfavorable feature of the event selection procedure. This occurs since the underlying data set contains the frequency of firms being mentioned in news articles globally - not the actual time of a specific event - upon which the identification procedure is based on detecting the largest events based on volume. As a results, the methodology may not capture the initial market response to "breaking" news, which is typically reflected in early investors reactions. Such early reactions are apparent in figure \ref{fig:ST_neg_news}, demonstrated by the large drawdowns between $t = -6$ and $t = 0$. The incremental increases in the bars from $t=-6$ preceding an event indicates that the initial news story may be published approximately six business days on average before the outstanding medias pick up the story and the spike in news actually happens, which is harmonious with the observed investor reaction. Intuitively, investor reactions seems to increase along with media attention.   \\

The benefits from the sensitivity analysis are two-fold. First, the results validate the robustness of the initial results. Second, it can broaden our knowledge on the drivers behind the results. \\

Section \ref{sec: sens_st_weights} demonstrates that applying more weight to smaller firms, through an equal-weighted portfolio, decrease the abnormal returns in absolute terms. The discrepancy originates in part from larger firms receiving more media attention with in-depth news on corporate behavior. With greater news coverage the stock price effect may be more severe due to 1; reaching a larger potential investor base and 2; investors receiving comprehensive and more detailed information. Such results are in contrast to \cite{tetlock_sentiment}, who finds a larger price impact on small stocks from negative sentiment due to the behavior of individual investors. The disparity between the results possibly emerge from the event selection methodology. My results are dependent on the volume of firm-specific news, which may suppress small firms, while \citeauthor{tetlock_sentiment} calculates the overall sentiment from a \textit{Wall Street Journal} column and measures the effect on small stocks, where the the volatility of these play a large role.  

Moreover section \ref{sec: sens_st_sd} established how restraining the identification rule from one to two and three standard deviations for events decreased the abnormal returns. There are no theoretical underpinnings of the result and it appears to be mostly an empirical issue. Intuitively, abnormal returns are expected to increase to the downside when the severity of negative events escalates. Apparently, a fair part of the events investors find important are ignored when focusing on more excessive news. 

\subsubsection{News related to specific SDGs}

The question in focus is whether specific dimensions of corporate social responsibility are more important to shareholders. As figure \ref{fig:ST_neg_bar} and \ref{fig:ST_pos_bar} illustrates, negative news related to Prosperity, Peace, and Partnership are associated with significantly pessimistic market reactions, while no single Pillar for positive news is associated with significant abnormal returns. With such results and that the focus of most other research has been with negative events, I will place more emphasis to the effect from adverse news. \\

People is the only Pillar with a positive, although insignificant, CAAR (0.34\%) in relation to negative events. Social corporate behavior can be difficult to quantify as the theme is related to SDGs 1 (No Poverty), 4 (Quality Education) and 5 (Gender Equality) among others. Hence, immoral acts in this relation may be more visible through consumer actions rather than short term investor reactions. The positive investor reaction from negative news is contradictory compared to other studies. For example \cite{chen2001layoffs} find an abnormal return of -1.2\% from social interactions of layoff announcements. The reaction from positive news is close to non-existing. 

The short term reaction from negative news on "Planet" is less severe than one may expect, given it incorporates themes such as clean water and climate action, which has received much public attention in the 21st century. \cite{karpoff2005reputational} finds that the average legal penalty for environmental violations is 2.26\% with the market penalty being approximately of the same magnitude. With a CAAR from negative news of less than 0.2\% a clear distinction is present. 

Negative news related to Prosperity generates a significant, negative CAAR of 0.6\%, whereas for positive news it is insignificant with an value of around 0.1\%. The Prosperity Pillar is mostly driven by SDG 7 (Affordable and Clean Energy) and 8 (Decent work and Economic Growth) which account for 70\% of the volume in events. The data does not provide information on the contents of the news, however events within these themes may focus on sustainable development and whether firms keep up with the Paris Agreement. The necessary global transition to sustainable energy sources is accompanied with increased investor attention on climate risk as a systematic risk. Companies that doesn't live up to climate goals are not necessarily penalized from a profit-maximization perspective, however they bear the risk of being black-listed by institutional investors with high ESG portfolio requirements, which as an isolated event is expected to increase divestments by other investors further \cite{dell2021norwegian}. 

\textbf{Peace, Justice and ??} - captures corporate fraud and most cases of trails and so on. 


The Partnership Pillar consists of SDG 17 (Partnerships for the Goals) alone. The SDG in itself is relatively vague, as the main message is to strengthen the implementation of global partnerships. However, it generates a CAAR of -1.1\%. The catalyst behind such events and significant returns lies in the discovery that some firms, proclaiming to be sustainable in diverse aspects, are revealed to be operating immorally. Such strategies, commonly referred to as 'green-washing,' are disapproved of by activist groups and seemingly also penalized in financial markets, if discovered. News about green-washing is related to SDG 17, which is expected to be the main driver of the negative abnormal returns. \cite{Blancard_ESG_sentiment} backs up the argument, but also notes that successful green-washing can help mediate the financial penalties from adverse ESG events.  


Whether specific SDGs are more important for shareholders is clearly dependent on the category of the news. Positive news related to the 5 P's does not generate abnormal returns in any instance, although Planet and Prosperity comes close to being significant. Hence, I conclude that the average impact from positive news is similar across the 5 Pillars of SDGs and not different from zero. On the other hand, abnormal returns related to negative news are varying to a higher degree. For Prosperity, Peace and Partnership the returns are significant, hence it appears shareholders attach more importance to sustainability goals within these themes than People and Planet.    

\subsubsection{Impact of ESG risk reputation}

Generalizing how stock returns change to general and specific SDG news provide a good empirical understanding of how investors integrate corporate sustainability in their decision making. \\
However, companies are facing different issues and have different risks toward ESG, hence external pressure might vary accordingly. The literature on the effect of reputation is inconclusive.
On the one hand, some argue that firms with a history of strong corporate responsibility will face more severe investor reactions from failing to live up to their status. Others, like \cite{flammer2013corporate}, argue high ESG grades will serve as a protective shield in gloomy times, meaning that firms with good a reputation experience less decrease of market value from negative ESG news. 

The illustrations from figure \ref{fig:ST_neg_ESG} are in contrast with the latter claims. The results from the graph are concretized in table \ref{tab: ST_neg_significance} along with significance levels of 10 and 21 day CAAR. 

Firms with low ESG risk are penalized relatively harder after negative events compared to medium and high risk-firms. As a natural virtue of the rating, firms with low ESG risk are expected to avoid extreme negative events related to corporate sustainability. Hence, if they do appear in such circumstances the penalty will be relatively harder.

For example, a low-risk company may have a relatively high market valuation based on their current rating. If they disappoint on ESG-related topics, investors need to adjust their expectations about the market value they place with the firm's sustainability profile. \cite{baron2009positive} explains the phenomena as a social pressure from citizen preferences with credible threats of divestment if firms do not live up to their reputation. \\ 
Moreover, firms with high ESG-risk are encountering no change in market value from negative news. A fair probability of negative events occurring for high-risk firms seems to be priced into the market value for the average firm as there is no initial market reaction. Medium ESG-risk firms are experiencing negative abnormal returns, however less negative than low-risk firms, which seems on par with an intermediate investor reaction.   

The reaction to positive news is less clear cut. Medium risk firms experience abnormal returns from positive events, as expected, while low risk firms bear negative abnormal returns. I find no theoretical underpinnings for the distinction in abnormal returns between the two groups in the literature. However, as mentioned, positive ESG announcements are found to be followed by negative abnormal returns \cite{fisher2011voluntary}, which is a reasonable explanation for the development in the low risk group, although the distinction to medium risk-firms remains unexplained.  

Overall, I find a clear distinction between ESG-risk levels of companies and their abnormal returns from negative news related to the SDGs. 



\subsection{Long term}

The hypotheses \#2 a and b aims to answer whether SDG-related news are associated with changes in firms' market values on a long horizon. I apply the Calendar Time Portfolio approach and use the Fama-French 5 factor model to examine the existence of abnormal returns. The empirical analysis is made with a focus on general news and with a partition on the level of ESG-risk a company faces.  

\textbf{a.}  Negative news related to the SDGs does not have a long term impact on firm's market values.
\textbf{b.}  Positive news related to the SDGs does not have a long term impact on firm's market values.

\subsection{Broad sustainability news and alpha}

I assess the statistical significance of the alpha generated from portfolios holding firms which have experienced a positive or negative event within T months. With significant monthly alphas of -0.84\% and -0.36\% from negative news-portfolios with holding periods of, respectively, 1 and 4 months, I reject hypothesis 2.a as negative news has a long term adverse impact on the average firms' market values.   

These results are in contrast with market efficiency, which proclaims the market should react instantly and adjust expectations to new information becoming available. Hence, such perspectives are in line with the short term results from section \ref{sec: short_term_analysis}. Post-event long term return anomalies are generally an indication of reactions from an inefficient market. However, the methodology in this study applies significant spikes in news articles as a proxy of events, indicating that a specific event date may not necessarily reveal the full context of a news story. Thus, post-event continuation of abnormal returns could be a reaction to more information about the specific events becoming available to the market. Although, such circumstances can result in overreaction as well as underreaction to the initial response, which is an argument from \cite{fama1998_events} to support the efficient market hypothesis.   

The portfolios based on positive events do not generate significant alpha across any of the holding periods. Hence, I fail to reject the hypothesis that positive news related to the SDGs does not have a long term impact on market values. 

The sensitivity analysis altered the original models by, separately, applying equal weights and by changing the event rule from one to two and three standard deviations with outcomes displayed in table \ref{tab: FF5_sensitivity}. Applying equal weights in portfolio allocation results in an adverse alpha for all holding periods compared to value weights.

Relatively more weight toward small stocks takes part in driving portfolio alphas lower, as the returns of small stocks tend to be more volatile \citep{Fama_french_3fac}. This becomes especially apparent as I am isolating for firms involved in negative events, which may induce volatility further. 

Tables \ref{tab: summary_neg_1} and \ref{tab: summary_neg_EW} show the sensitivities of the value and equal weighted portfolios to the small-minus-big factor from the Fama-French regressions. The value weighted portfolio has approximately zero sensitivity with a coefficient of -0.04, whereas the equal weighted is 0.51. Hence, higher sensitivity to the returns of small stocks is seemingly a part of the explanation to lower alphas. The change in sensitivity from changing the weighting method further indicates that the portfolio does include a variety of small firms, but their influence is suppressed when applying value weights. \\ 
The intuition acts as a counterargument to what I postulated in the former subsection about applying different weights. However, as mentioned in section \ref{sec:method}, the time horizon plays an important role in adjusting for risk, as it is of less importance in the short term whereas  with long horizons even small errors in risk-adjustments can increase discrepancies in abnormal returns as the errors compound exponentially over time. \\

Changing the relative amount of articles required for a spike in news to be classified as a significant event from one standard deviation to two and three doesn't change the direction of the abnormal returns. However, all alphas become insignificant. On a side note, the monthly average amount of firms in the portfolios decrease by roughly half as the requirements are tightened by one standard deviation. For example, with T = 1 the portfolio with SD = 1 consists of 18 firms, while SD = 2 has 9. Portfolios with less firms will inevitably be exposed to more idiosyncratic risks arising from the individual companies. Hence, the likelihood, that the portfolio alpha is driven by specific company risks rather than the response to negative news, increases. The analysis performed in this paper relies on diversification benefits from rebalancing random firms on a monthly basis as it allows to approximately isolate the effect of negative news on stock returns. Hence, reducing the amount of firms comes with disadvantages.  Anyway, due to the sign and magnitude of abnormal returns of the altered portfolios, it seems the original results are representative of how the strategy performs. 

\subsubsection{Impact of ESG risk reputation}

The analysis of the short term impact of negative news suggests that firms with low ESG-risks are penalized harder than firms with medium and high risk. The gap between low and medium risk companies does not persist on the long horizon as low and medium risk firms generate an alpha of -0.64\% and -0.61\%, respectively, as illustrated in table \ref{tab:FF5_neg_ESG}. 
Furthermore, the pattern cannot be fully generalized to inferences on longer horizons, due to the complete turnaround of the sign on alphas on high risk firms. However, the high risk portfolio has imminent issues with idiosyncratic risks as it consists of only 81 events and holds only 18 distinct firms throughout the full period, which inevitably plays a sizeable factor in the remarkably negative and fluctuating alpha outcome. Ignoring the outcome of the high risk portfolio, the remaining results do to some degree indicate a pattern similar to that of the short term with negative abnormal returns.

As the low and medium risk portfolios generate negative alphas of approximately the same magnitude the long term abnormal returns are assumed to be independent of the level of ESG-risk of firms.  


Looking into the risk factors from ESG portfolios may better our understanding of how firm characteristics 

