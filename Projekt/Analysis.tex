

\subsection{The impact of general SDG news on stock prices} \label{sec: short_term_analysis}

To answer hypothesis 1 and provide intuition to the sub-questions, I assess the statistical significance of the cumulative average abnormal returns over a window of 21 days around identified events of negative and positive news in section \ref{ST_results}.  
The significance of the results from the Market Model are summarised with a z-test. The test statistic is computed and compared to its assumed distribution under the null hypothesis that the abnormal returns are zero. The z-scores and significance complementary to AAR on $t=0$ and CAAR on 10 and 21 days around the event are presented in table \ref{tab: ST_neg_significance} for negative events and table \ref{tab: ST_pos_significance} for positive ones.  

\subsubsection{Short term hypothesis}

The null hypothesis assumes that the market does not react to news related to SDGs in the short horizon. Any significant deviation from the expected returns indicates that news related to sustainability has considerable influence on firm's market values.  \\
 
According to \ref{tab: ST_neg_significance} negative SDG disclosures impact firms' market values significantly adverse both on the event date and with a 10 and a 21 day event window. In support of hypothesis 1.a the cumulative abnormal return is $-1.56\%$ throughout the 21-day event window. The magnitude of the impact is high and significantly lower than zero at the 1\% level. Hence, I reject the hypothesis of no abnormal returns from negative events, and accept the alternative that negative events are concurrent with negative abnormal returns on average. 

For positive news the cumulative abnormal average change in firm's market value around a 10 and 21-day event window is positive, however highly insignificant as per figure\ref{tab: ST_pos_significance}. With a CAAR of 0.07\% throughout the full event window, I fail to reject the null hypothesis. Hence, positive news related to the SDGs does not have a short term impact on firm's market values. However, the AAR is significantly positive at 5\% level with a value of 0.08\% on the event day, which indicate that a spike in positive news is rewarded instantaneously. However, the simple reaction is not sufficient to establish general intuition.

Accordingly, shareholders seem to penalize negative corporate social responsibility, while they do not really reward positive practices. 

\begin{table}[H]
\centering
\caption{Negative news: AAR and CAAR on overall and ESG risk level} 
\begin{tabular}{ccccc}
  \hline  \hline
  & \multicolumn{1}{c}{Overall} &  \multicolumn{1}{c}{Low} & \multicolumn{1}{c}{Medium} & \multicolumn{1}{c}{High}\\  
 \hline
$AAR_{t=0}$ &   $\underset{(-3.51)}{-0.40^{***}}$ &   $\underset{(-4.48)}{-0.65^{***}}$ &   $\underset{(-3.41)}{-0.39^{***}}$ &   $\underset{(-0.29)}{-0.14 }$ \\

$CAAR_{[-5;+5]}$  &  $\underset{(-5.46)}{-1.10^{***}}$ &   $\underset{(-6.72)}{-2.25^{***}}$ &   $\underset{(-3.70)}{-1.02^{***}}$ &   $\underset{(0.06)}{0.04}$ \\ 

$CAAR_{[-10;+10]}$    & $\underset{(-5.70)}{-1.56^{***}}$ &   $\underset{(-8.46)}{-4.01^{***}}$ &   $\underset{(-3.14)}{-1.16^{***}}$ &   $\underset{(0.46)}{0.35}$ \\ 
   \hline \hline
   \multicolumn{5}{p{12cm}}{ \footnotesize $^* \; p\; <\; 0.1$, $ ^{**} \; p\; <\; 0.05$, $ ^{***} \; p\; <\; 0.01$  } \\
   \multicolumn{5}{p{13cm}}{\footnotesize The tables shows the CAAR and associated t-value related to positive and negative events over an event window of 10, and 21 days surrounding the event date along with the AAR on $t=0$. Negative events consists of 1046 observations. } \\
   \hline
\end{tabular}
\label{tab: ST_neg_significance}
\end{table}

The findings are in line with most of the previous literature. \cite{Blancard_ESG_sentiment} and \citep{kruger2015corporate} finds similar evidence of a short term pessimistic market reaction from negative news. \citeauthor{Blancard_ESG_sentiment} identify slightly positive, but insignificant, abnormal return for positive events as well, whereas \citeauthor{kruger2015corporate} finds an ambivalent association that depends on the quality of the relation between the firms and their stakeholders. Short term shareholder reactions are in line with theories on efficient markets \citep{fama1969_EMH}, as the development in AAR is equivalent to shareholders incorporating new information becoming available to the market. There is nothing new here. However, the results from this paper indicates that the shareholders indeed incorporates new information related to the Social Development Goals into their investment decisions. 

\cite{fisher2011voluntary} even find that companies which make ESG-positive announcement are penalized by shareholders since such actions are in conflict with firm value maximization. The distinction between the results may stem from the event methodology, as I gather all types of news related to SDGs, which may in fact be value maximizing. 

The Market Model resemble a good measure of abnormal returns in relation to negative news. In figure \ref{fig:ST_neg_news} the AAR is hovering around the level of 0\% in the first five days of the window and again after the event has occurred at $t = 0$, implying that the realized returns are approximately as the expected returns. No clear pattern is present regarding positive news, which may be more indicative of the negligibility of the news factor rather than criticism of the models applied.   

\begin{table}[H]
\centering
\caption{Positive news: AAR and CAAR (in \%) on overall and ESG risk level} 
\begin{tabular}{ccccc}
  \hline  \hline
  & \multicolumn{1}{c}{Overall} &  \multicolumn{1}{c}{Low} & \multicolumn{1}{c}{Medium} & \multicolumn{1}{c}{High}\\  
 \hline
$AAR_{t=0}$ &  $\underset{(2.12)}{0.08^{**}}$ & $\underset{(0.16)}{0.00}$ & $\underset{(2.77)}{0.15^{***}}$ &  $\underset{(-0.36)}{-0.05}$ \\ 
$CAAR_{[-5;+5]}$  & $\underset{(0.75)}{0.09}$ &  $\underset{(-1.66)}{-0.26^{*}}$ &  $\underset{(2.28)}{0.39^{**}}$ &  $\underset{(-1.25)}{-0.44}$ \\ 
$CAAR_{[-10;+10]}$    & $\underset{(0.46)}{0.07}$ &  $\underset{(-2.29)}{-0.48}$ &  $\underset{(2.90)}{0.47^{***}}$ &  $\underset{(-0.80)}{-0.34}$ \\ 
    \hline \hline
   \multicolumn{5}{p{10cm}}{ \footnotesize $^* \; p\; <\; 0.1$, $ ^{**} \; p\; <\; 0.05$, $ ^{***} \; p\; <\; 0.01$  } \\
   \multicolumn{5}{p{10cm}}{\footnotesize The tables shows the CAAR and associated t-value related to positive and negative events over an event window of 5, 10, and 21 days surrounding the event date along with the AAR on $t=0$. Positive and negative events consists of, respectively, 3564 and 1046 observations. } \\
   \hline
\end{tabular}
\label{tab: ST_pos_significance}
\end{table}

Likewise, the event study methodology works adequately at identifying negative events from spikes in news, as an evident investor reaction is recognized through the abnormal returns. However, the reaction materializes prior to the identification of events, which is an unfavorable feature of the event selection procedure. This occurs since the underlying data set contains the frequency of firms being mentioned in news articles globally - not the actual time of a specific event - upon which the identification procedure is based on detecting the largest events based on volume. As a results, the methodology may not capture the initial market response to "breaking" news, which is typically reflected in early investors reactions. Such early reactions are apparent in figure \ref{fig:ST_neg_news}, demonstrated by the large drawdowns between $t = -6$ and $t = 0$. \\
The incremental increases in the bars from $t=-6$ preceding an event indicates that the initial news story may be published approximately six business days on average before the outstanding medias pick up the story and the spike in news actually happens, which is harmonious with the observed investor reaction. Intuitively, investor reactions seems to increase along with media attention. \\

The benefits from the sensitivity analysis are two-fold. First, the results validate the robustness of the initial results. Second, it can broaden our knowledge on the drivers behind the results. \\

Section \ref{sec: sens_st_weights} demonstrates that applying more weight to smaller firms, through an equal-weighted portfolio, decrease the abnormal returns in absolute terms. The discrepancy originates in part from larger firms receiving more media attention along with in-depth news on corporate behavior. With greater news coverage the stock price reaction may be more severe due to 1) the news reach a larger potential investor base, and 2) investors receive comprehensive and more detailed information about the specific companies. The specific results are in contrast to \cite{tetlock_sentiment}, who finds a larger price impact on small stocks from negative sentiment due to the behavior of individual investors. The disparity between the results possibly emerge from the event selection methodology. My analysis includes companies based on the volume of firm-specific news, which may suppress small firms, while \citeauthor{tetlock_sentiment} calculates the overall sentiment from a \textit{Wall Street Journal} column and measures the effect on small stocks, where volatility to general sentiment plays a large role.  

Moreover section \ref{sec: sens_st_sd} establish how restraining the threshold of events increasing the abnormal returns. Intuitively, abnormal returns are expected to increase to the downside when the severity of negative events escalates. A more strict threshold directly affects which events get identified as important. Thus, the increased abnormal returns are completely in line with expectations.  




\subsubsection{Long term hypothesis}

The hypotheses \#2 a and b aims to answer whether SDG-related news are associated with changes in firms' market values on a long horizon. The empirical analysis is made with a focus on general news and with a partition on the level of ESG-risk a company faces. I assess the statistical significance of the alpha generated from portfolios holding firms which have experienced a positive or negative event within T months. With significant monthly alphas of -0.84\% and -0.36\% from negative news-portfolios with holding periods of 1 and 4 months, respectively, I reject hypothesis 2.a as negative news has a long term adverse impact on the average firm's market value.   

These results are in contrast with market efficiency, which proclaims the market should react instantly and adjust expectations to new information becoming available. Post-event long term return anomalies are generally an indication of reactions from an inefficient market \ref{sec: short_term_analysis}. However, the methodology in this study applies significant spikes in news articles as a proxy of events, indicating that a specific event date may not necessarily reveal the full context of a news story. Thus, post-event continuation of abnormal returns could be a reaction to more information about the specific event becoming available to the market. Such circumstances can result in overreaction as well as underreaction to the initial response, which is an argument from \cite{fama1998_events} to support the efficient market hypothesis.   

The portfolios based on positive events do not generate significant alpha across any of the holding periods. Hence, I fail to reject the hypothesis that positive news related to the SDGs does not have a long term impact on market values. 

The sensitivity analysis altered the original models by, separately, applying equal weights and by changing the event rule from one to two and three standard deviations, with all outcomes displayed in table \ref{tab: FF5_sensitivity}. Applying equal weights in portfolio allocation results in an adverse alpha for all holding periods compared to value weights. 

Relatively more weight toward small stocks takes part in driving portfolio alphas lower. One reason may be the apparent higher general volatility in the returns of small stocks \citep{Fama_french_3fac}. Such an effect becomes more feasible as I am isolating for firms involved in negative events, which may induce volatility further. Moreover, the issue may not necessarily arise as a discrepancy between the results from the short and long term results, but as a feature of the Fama-French regressions. Tables \ref{tab: summary_neg_1} and \ref{tab: summary_neg_EW} show the sensitivities of the value and equal weighted portfolios, with T = 1, to the risk factors in the regressions. 

Both portfolios have a positive and high sensitivity to the market excess return of around 1.2 and to high-minus-low (value) of 0.31 and 0.22 for value and equal weights, respectively. This comes as no surprise. Generally, the sensitivity to the market return is apparent, since the portfolio firms have simply one characteristic in common, which is that they have experienced an event, everything else should be random in a diversified portfolio. However, with more allocation to smaller stocks, the equal weighted portfolio has a significant beta coefficient of 0.51 to the small-minus-big (size) factor. Additionally, the beta to the robust-minus-weak (RMW) factor is 0.31. However, the value weighted portfolio has approximately zero sensitivity to these factors at -0.04 and -0.07, respectively. Although the r-squared is high at 0.89, it appears that the factor regressions has issues with explaining the returns of the value weighted portfolios, with most of the risk being explained by the market excess return and the value factor. With less sensitivity to the five risk factors the cross-section of returns becomes less explainable, hence the variation may end up in the residuals. This would result in a less significant regression intercept.  

Moreover, the intuition is countering the postulates from the former section on smaller stocks receiving less media attention. However, another possible explanation to dampen the alphas is that the reactions from small stocks appear to emerge with a lag compared to larger stocks, which helps explain the increased alpha from equal weighted portfolios with longer holding periods. 


Changing the relative amount of articles required for a spike in news to be classified as a significant event from one standard deviation to two and three does not change the sign of the abnormal returns. However, all alphas become insignificant. On a side note, the monthly average amount of firms in the portfolios decrease by roughly half as the requirements are tightened by one standard deviation. For example, with T = 1 the portfolio with SD = 1 consists of 18 firms on average, while SD = 2 decreases the amount to 9. Although presumed to be sorted on negative events, portfolios with less firms will inevitably be exposed to more idiosyncratic risks arising from the individual companies. Hence, the likelihood that the portfolio alpha is driven by specific company risks, rather than a response to negative news, increases. The analysis performed in this paper relies on diversification benefits from rebalancing random firms on a monthly basis as it allows to approximately isolate the effect of negative news on stock returns. Hence, reducing the amount of firms comes with disadvantages. The increased idiosyncrasies are revealed through decreasing r-squared values, e.g. from 0.89 to 0.81 when employing 1 and 3 standard errors. Hence, the abilities of the factor models to explain the returns through market risk factors decreases. 

Anyway, due to the sign and magnitude of abnormal returns of the altered portfolios, it seems the original results are representative of how the strategy performs. 


\subsection{News related to specific SDGs}

The question in focus is whether specific dimensions of corporate social responsibility are more important to shareholders. As figure \ref{fig:ST_neg_bar} and \ref{fig:ST_pos_bar} illustrates, negative news related to Prosperity, Peace, and Partnership are associated with significantly pessimistic market reactions, while no single Pillar for positive news is associated with significant abnormal returns. With such results and that the focus of most other research has been with negative events, I will place more emphasis to the effect from adverse news.

People is the only Pillar with a positive, although insignificant, CAAR (0.34\%) in relation to negative events. Social corporate behavior can be difficult to quantify as the theme is related to SDGs revolving no poverty, quality education and gender equality among others. Hence, immoral acts in this relation may be more visible through consumer actions rather than short term investor reactions. The positive investor reaction from negative news is contradictory compared to other studies. For example \cite{chen2001layoffs} find an abnormal return of -1.2\% from social interactions. \citep{farber2009changing} confirms the negative reaction and adds that the market reaction declined between the period 1970-1999. SDG 3 (Good Health and Well Being) has the second highest number of negative events among the SDGs, however no investor reaction is present. The volume of events indicates the media considers well-being of people as a important theme, but investors do not view it as essential from a corporate perspective.

The difference in our findings may be due to the methodology used. In the study by \citeauthor{chen2001layoffs}, they also conducted an event study, but their approach involved precise and defined events. Therefore, the large number of negative events associated with SDG 3 might indicate that the public focus on the topic is so significant that numerous events are identified, resulting in the negative impact on corporate performance gets suppressed by nonevents. The reaction from positive news is close to non-existing. 

The short term reaction from negative news on "Planet" is less severe than one may expect, given it incorporates themes such as clean water and climate action, which has received much public attention in the 21st century. \cite{karpoff2005reputational} finds that the average legal penalty for environmental violations is 2.26\% with the market penalty being approximately of the same magnitude. Moreover, \cite{capelle2010does} discovers an average penalty of -1.3\% of market value from industrial disasters. With a insignificant and low CAAR from negative news of less than 0.2\% a clear distinction to related research is present. Most of the identified events are related to SDGs 12 and 13, per figure \ref{fig:event_distribution_SDG}, which generate abnormal returns of approximately 0\% according to figure \ref{fig:ST_neg_bar_all}. Hence, climate action (12) and responsible consumption (13) draws a lot of attention from the media, but investors do not react to the news. The same story applies to the CAAR from positive news, which is approximately 0.1\% and insignificant. 

Negative news related to Prosperity generates a significant CAAR of -0.6\%, whereas for positive news it is insignificant with an value of around 0.1\%. The Prosperity Pillar is mostly driven by SDG 7 (Affordable and Clean Energy) and 8 (Decent work and Economic Growth) which account for 70\% of the volume in negative events. SDG 7 experiences the most events within all groups. It is particularly interesting for investors, as it stands out as the only SDG that yields significant abnormal returns from positive events and exhibits nearly significant negative returns from adverse events.\\
The necessary global transition to sustainable energy sources is in line with the Paris Agreement - making such investments legally binding - and accompanied with increased investor attention on climate risk as a systematic risk. In support of the positive investor reaction to good news, \cite{hart1996does} finds a positive relation between reducing emissions and financial performance for a sample of S\&P 500 firms. A newer study from \cite{paytobegreen} elaborates on the relation and shows that neither brown or green companies with higher environmental scores perform better financially. Hence, the investor decisions to positive and negative events are made purely from a social perspective. As such, companies that does not live up to climate goals are not necessarily penalized from a profit-maximization perspective, however they bear the risk of being black-listed by institutional investors with high ESG portfolio requirements, which as an isolated event is expected to increase divestments by other investors further \cite{dell2021norwegian}. 

The SDG Pillar on Peace encompass SDG 16 (Peace, Justice and Strong Institutions) solely. The keywords peace and strong institutions are rarely associated with important corporate events and financial performance. Justice on the other hand is important for both, as the keyword is related to all kinds of negative corporate events such as accounting fraud and market manipulation. Consequently, the SDG is identified with more than 1300 negative events - most of all SDGs, by far. With a CAAR of approximately -1.5\% over the full window, events related to the SDG are also associated with the largest average drop in market value across any of the sustainability goals. A short term event study from \cite{bauer2010misdeeds}, on alleged crimes of corporate misconduct involving governance-related offences, shows that investors react negatively to corporate filings before the events take place. They report an average -10\/+10 CAAR of -11.6\% from a Fama-French 3 factors model. With declines of that magnitude from actual filings, the investor reaction of -1.5\% from spikes in news relating to such events seem appropriate. 

The Partnership Pillar consists of SDG 17 (Partnerships for the Goals) alone. The SDG in itself is relatively vague, as the main message is to strengthen the implementation of global partnerships. However, it generates a CAAR of -1.1\%. The catalyst behind such events and significant returns lies in the discovery that some firms, proclaiming to be sustainable in diverse aspects, are revealed to be operating immorally. Such strategies, commonly referred to as 'green-washing,' are heavily disapproved of by activist groups and seemingly also penalized in financial markets, if discovered. News concerning green-washing is related to SDG 17, which is the expected main driver of the negative abnormal returns. \cite{Blancard_ESG_sentiment} backs up the argument of investors punishing such misconduct, but also note that successful green-washing can help mediate the financial penalties from adverse ESG events. Although green-washing can be net positive in general, the events I consider are meant to isolate the cases where such a misconduct was revealed, which fits well into the negative shareholder reaction.   

Whether specific SDGs are more important for shareholders is clearly dependent on the category of the news. Positive news related to the 5 P's do not generate abnormal returns in any instance, although Planet and Prosperity comes close. Hence, I conclude that the average impact from positive news is not statistically different from zero and similar across the 5 Pillars of SDGs. On the other hand, abnormal returns related to negative news are varying to a higher degree. For Prosperity, Peace and Partnership the returns are significant, hence it appears shareholders attach more importance to sustainability goals within these themes than People and Planet.    



\subsection{Impact of ESG risk reputation}

Generalizing changes in stock returns to overall and specific SDG news provide a good empirical understanding of how investors integrate corporate sustainability in their decision making. However, companies are facing different issues and risks toward ESG, hence external investor pressure might vary accordingly. 
The literature on the effect of reputation is inconclusive. On the one hand, some argue that firms with a  will face more severe investor reactions from failing to live up to their status. \cite{noNewsgoodnews} backs this claim from a media perspective, as they find that accidents related to companies with an admirable CSR records are far more likely to be reported in the media. Others, like \cite{flammer2013corporate} and \cite{godfrey2009relationship}, argue that a history of strong corporate responsibility will mitigate the social pressure from adverse events, meaning that firms with a good reputation experience less decrease of market value from negative ESG news. Moreover, \cite{Blancard_ESG_sentiment} argues that the sector's ESG reputation can mitigate the market value loss from negative events.  

Initially, we need to pay attention to the way ESG reputation is defined. \cite{rennings2007effect} point out that outcomes may differ depending on if reputation is calculated relative to industry peers or not. The ESG Risk Categories are absolute, hence a rating is comparable across all sub-industries. 

The illustrations from figure \ref{fig:ST_neg_ESG} are in contrast with such theories. The results from the graph are concretized in table \ref{tab: ST_neg_significance} along with significance levels of 10 and 21 day CAAR. Firms with low ESG risk are penalized more heavily after negative events compared to medium and high risk-firms. As a natural virtue of the rating, firms with low ESG risks comes with a price premium, as they are expected to avoid extreme negative events related to corporate sustainability. For example, a low-risk company may have a relatively high market valuation based on their current rating or history. If they disappoint on ESG-related topics, investors need to adjust their expectations about such events happening in the future and the corresponding change in market value they place with the firm's sustainability profile. The distinction between theory and results may appear due to the media effect as described by \cite{noNewsgoodnews}, with the argument that low risk firms are more exposed to the media in case of negative events. 

Moreover, firms with high ESG-risk are encountering no change in market value from negative news. If investors place a fairly high probability of negative events occurring for high-risk firms, then the risk of an actual events seems to be priced into the market value apriori. Medium ESG-risk firms are experiencing negative abnormal returns, however less negative than low-risk firms, which seems on par with an intermediate investor reaction.  

The reaction to positive news is less clear cut. Medium risk firms experience abnormal returns from positive events, as expected, while low risk firms bear negative abnormal returns. I find no theoretical underpinnings for the distinction in abnormal returns between the two groups in the literature. Although, \cite{flammer2013corporate} does report that positive market reactions to eco-friendly events is smaller for companies with low environmental risk, and that the increasing social pressure to become green has resulted in decreased reactions to eco-frinedly initiatives over time. However, as mentioned, positive ESG announcements have been found to be followed by negative abnormal returns \citep{fisher2011voluntary}.  

\cite{serafeim2022stock} finds that consensus ESG ratings predict future ESG news. Hence, a positive rating means investors expect mostly positive news in the future. Such a results can help explain why shareholders are not rewarding low-risk companies that experience a positive event, whereas medium risk firms are rewarded. Likewise, as high risk companies are expected to endure negative situations, they are not penalized to the same degree.  


Overall, I find a clear distinction between ESG-risk levels of companies and their abnormal returns from negative news related to the SDGs. 

\textbf{Long term:}

The gap between low and medium risk profiles does not persist on the long horizon. According to figure \ref{tab:FF5_neg_ESG} portfolios with low and medium risk generate abnormal returns of -0.64\% and -0.61\%, respectively, with a holding period of one month. The outcomes are similar for longer holding periods. 
Although the portfolios generate approximately equivalent alphas, the portfolio characteristics are completely different according to their sensitivities to the five factors, as illustrated in tables \ref{tab: summary_neg_ESG_L} and \ref{tab: summary_neg_ESG_M} in the appendix. For example, the low risk portfolio is negatively correlated with the size factors at -0.28, whereas the medium risk portfolio has a coefficient of 0.41. Hence, it appears that the loss in market value from negative events are similar across company characteristics and risk toward ESG on a longer horizon. The sensitivities are estimated from the portfolios with holding periods of one month, however they are mostly similar for longer periods, but these values are not reported. 

Furthermore, the pattern cannot be fully generalized to inferences on longer horizons, due to the complete turnaround of the sign on alphas on high risk firms. The high risk portfolio has imminent issues with idiosyncratic risks as it consists of only 81 events and holds only 18 distinct firms throughout the full period, which inevitably plays a sizeable factor in the remarkably negative and fluctuating alpha outcome. Ignoring the outcome of the high risk portfolio, the remaining results do to some degree indicate a pattern similar to that of the short term due negative abnormal returns.

However, as the low and medium risk portfolios generate negative alphas of approximately the same magnitude, the long term abnormal returns are assumed to be independent of the level of companies' ESG-risks. 


\textbf{A slight phrase pre-conclusion on the discoveries from the analysis.}