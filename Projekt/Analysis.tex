
The analysis is divided into three sections. First, in section \ref{sec: short_term_analysis} I focus on answering the hypotheses on short- and long-term performance from events related to the overall Social Development Goals, and to demonstrate the benefits of the sensitivity analysis. Then, the analysis shifts its focus towards building intuition and understanding regarding the sub-questions. In section \ref{sec: short_term_analysis_SDG}, I aim to determine whether investors place varying emphasis on different themes within sustainability. Finally, section \ref{ESG_reputation} delves into the relevance of firms' ESG risk characteristics when investors react to events.

\subsection{The impact of general SDG news on stock prices} \label{sec: short_term_analysis}
 
To answer the hypotheses, I assess the statistical significance and the intuition from the short- and long term results.  

\subsubsection{Short term hypothesis} 

The null hypothesis assumes that there is no market does reactions to news related to the SDGs within a short horizon. Any significant deviation from the expected returns suggest that sustainability-related news has notable influence on firm's market values.  \\
The significance of the results obtained from the Market Model is evaluated using a z-test. This test statistic is computed and compared to its assumed distribution under the null hypothesis that the abnormal returns are zero. The z-scores and corresponding significance levels, complementary to AAR on $t=0$ and CAAR 10 and 21 days surrounding the event, are presented in table \ref{tab: ST_neg_significance} for negative events and table \ref{tab: ST_pos_significance} for positive ones. The values presented in the tables are equivalent to the graphics in figures \ref{fig:ST_neg_news} and \ref{fig:ST_pos_news}. 

According to table \ref{tab: ST_neg_significance} negative SDG disclosures impact firms' market values significantly adverse not only on the event date but also over a 10-day and 21-day event window. In support of hypothesis 1.a the cumulative abnormal return is $-0.72\%$ throughout the 21-day event window. The observed impact of negative events is substantial and significantly below zero at the 1\% level of significance. As a result, I reject the hypothesis of no abnormal returns from negative events, and accept the alternative that negative events are associated with negative abnormal returns on average. 

Regarding positive news, the cumulative abnormal average change in market value around a 10- and 21-day event window is slightly negative. However, the results are insignificant, as indicated by the "Overall" column in table \ref{tab: ST_pos_significance}. With a CAAR of -0.02\% over the entire window, there is insufficient evidence to reject the null hypothesis. Hence, positive news related to the SDGs does not have a short term impact on firm's market values. Nonetheless, the AAR is significantly positive at the 1\% level with a value of 0.05\% on the event day. This suggest that a spike in positive news is instantly rewarded with a small increase in market value. However, this reaction alone is not sufficient to establish general intuition.

Accordingly, shareholders seem to penalize negative corporate social responsibility, while they do not really reward positive practices. 

\begin{table}[H]
\centering
\caption{Negative news: AAR and CAAR on overall and ESG risk level} 
\begin{tabular}{ccccc}
  \hline  \hline
  & \multicolumn{1}{c}{Overall} &  \multicolumn{1}{c}{Low} & \multicolumn{1}{c}{Medium} & \multicolumn{1}{c}{High}\\  
 \hline
$AAR_{t=0}$ &   $\underset{(-2.25)}{-0.17^{**}}$ &   $\underset{(-2.67)}{-0.22^{***}}$ &   $\underset{(-1.18)}{-0.13}$ &   $\underset{(-0.20}{-0.67 }$ \\

$CAAR_{[-5;+5]}$  &  $\underset{(-4.01)}{-0.65^{***}}$ &   $\underset{(-5.22)}{-1.39^{***}}$ &   $\underset{(-1.65)}{-0.36^{*}}$ &   $\underset{(-1.11)}{-0.52}$ \\ 

$CAAR_{[-10;+10]}$    & $\underset{(-3.21)}{-0.72^{***}}$ &   $\underset{(-4.49)}{-1.65^{***}}$ &   $\underset{(-1.80)}{-0.54^{**}}$ &   $\underset{(-0.38)}{-0.25}$ \\ 
   \hline \hline
   \multicolumn{5}{p{12cm}}{ \footnotesize $^* \; p\; <\; 0.1$, $ ^{**} \; p\; <\; 0.05$, $ ^{***} \; p\; <\; 0.01$  } \\
   \multicolumn{5}{p{13cm}}{\footnotesize The tables shows the CAAR and associated t-value related to positive and negative events over an event window of 10, and 21 days surrounding the event date along with the AAR on $t=0$. Negative events consists of 1046 observations. } \\
   \hline
\end{tabular}
\label{tab: ST_neg_significance}
\end{table}

The findings are in line with most of the previous literature. \cite{Blancard_ESG_sentiment} and \citep{kruger2015corporate} finds similar evidence of a short term pessimistic market reaction from negative news by applying the event study methodology and the Market Model as well. \citeauthor{Blancard_ESG_sentiment} identify slightly positive, but insignificant, abnormal return for positive events as well, whereas \citeauthor{kruger2015corporate} finds an ambivalent association that depends on the quality of the relation between the firms and their stakeholders. 

The short term shareholder reactions are in line with theories on efficient markets \citep{fama1969_EMH}, as the development in AAR is equivalent to shareholders promptly incorporate newly available information into to the market. While this is not a novel finding, the results of this paper provide evidence that shareholders do indeed integrate general information related to the Social Development Goals into their investment decisions. 

The Market Model appears to be a good choice for measuring abnormal returns in relation to negative news. In figure \ref{fig:ST_neg_news} the AAR is hovering around 0\% in the first four days of the window and again after the event has occurred at $t = 0$, implying that the expected returns are effective in reflecting the realized returns. The lack of a clear pattern in the reactions to positive news may be more indicative of the relative negligible impact of the news factor rather than criticism of the model.    

\begin{table}[H]
\centering
\caption{Positive news: AAR and CAAR (in \%) on overall and ESG risk level} 
\begin{tabular}{ccccc}
  \hline  \hline
  & \multicolumn{1}{c}{Overall} &  \multicolumn{1}{c}{Low} & \multicolumn{1}{c}{Medium} & \multicolumn{1}{c}{High}\\  
 \hline
$AAR_{t=0}$ &  $\underset{(3.22)}{0.05^{***}}$ & $\underset{(0.37)}{0.03}$ & $\underset{(1.05)}{0.08}$ &  $\underset{(0.20)}{0.16}$ \\ 
$CAAR_{[-5;+5]}$  & $\underset{(-1.18)}{-0.07}$ &  $\underset{(-0-55)}{-0.13 }$ &  $\underset{(0.22)}{0.05 }$ &  $\underset{(-0.91)}{-0.36}$ \\ 
$CAAR_{[-10;+10]}$    & $\underset{(-0.23)}{-0.02}$ &  $\underset{(-2.25)}{-0.23^{***}}$ &  $\underset{(1.86)}{0.24^{**}}$ &  $\underset{(-1.04)}{-0.34}$ \\ 
    \hline \hline
   \multicolumn{5}{p{12.5cm}}{ \footnotesize $^* \; p\; <\; 0.1$, $ ^{**} \; p\; <\; 0.05$, $ ^{***} \; p\; <\; 0.01$  } \\
   \multicolumn{5}{p{13cm}}{\footnotesize The tables shows the CAAR and associated t-value related to positive and negative events over an event window of 5, 10, and 21 days surrounding the event date along with the AAR on $t=0$. Positive and negative events consists of, respectively, 3564 and 1046 observations. } \\
   \hline
\end{tabular}
\label{tab: ST_pos_significance}
\end{table}

Likewise, the event study methodology works adequately at identifying negative events from spikes in news, as an evident investor reaction is recognized through the abnormal returns. However, the investor reaction materializes prior to the actual identification of events ($t=0$), which is a drawback of the event selection procedure. This issue arises since the underlying data set contains the frequency of firms being mentioned in news articles globally, rather than the precise timing of specific events. Consequently, the event identification procedure relies on detecting the largest events based on volume. As a results, the methodology may not fully capture the initial market response to "breaking" news, which is typically reflected in early investors reactions. These early reactions, characterized by significant drawdowns, are apparent in figure \ref{fig:ST_neg_news} between $t = -6$ and $t = 0$. \\
The incremental increases in the bars, from $t=-6$ preceding an event, indicates that, on average, the initial information regarding a particular event tends to be published approximately six business days prior to the event gaining significant attention from the outstanding media and experiences a spike in news coverage. During the period between $t=-6$ and $t=0$ investors gradually become aware of the event and react accordingly based on the information they acquire. 


\subsubsection{Long term hypothesis}

The hypotheses \#2 a and b aims to answer whether SDG-related news are associated with changes in firms' market values on long horizons. I assess the statistical significance of the alpha generated from portfolios that hold firms which have experienced a positive or negative event within T months. The regressions reveal significant monthly alphas of -0.84\% and -0.36\% from negative news portfolios with holding periods of one and four months, respectively. With these results I reject hypothesis 2.a, which implicate that negative news has a long term adverse impact, of at least four months, on the average firm's market value. The portfolios based on positive events do not generate significant alpha across any of the holding periods. Hence, I fail to reject the hypothesis that positive news related to the SDGs does not have a long term impact on market values.

However, the portfolio with a holding period of one month generates a negative alpha. This suggests that firms may be penalized when they are associated with positive interactions with the social development goals. Certain investors perceive interactions with SDGs and ESG as creating additional and unnecessary costs for the firm, which they believe may impact profitability negatively. For example, Fisher (2011) finds that companies making ESG-positive announcements are penalized by shareholders due to perceived conflicts with firm value maximization. 

\subsubsection{Threshold values}

The benefits from the sensitivity analysis are two-fold. First, the results validate the robustness of the initial findings. Second, it can broaden our knowledge on the drivers behind the results. 

Section \ref{sec: sens_st_sd} establish the impact of tightening the event threshold on the short-term abnormal returns. A more strict threshold directly affects which events get identified as important. Thus, the observed decrease in abnormal returns in figure \ref{fig:ST_neg_sensitivity} is not in line with expectations. There are no theoretical underpinnings of the result and it appears to be mostly an empirical issue in the sample. It appears that the events receiving the most public attention may not necessarily coincide with the events that investors value the most. 

The implications are similar for the long-term abnormal returns as demonstrated in table \ref{tab: FF5_sensitivity}. However, all the estimated alpha value becomes insignificant when tightening the threshold. 

On a side note, the monthly average amount of firms in the portfolios decrease by roughly half as the threshold is tightened by one standard deviation. For instance, with a holding period of one month, the portfolio with a threshold of one standard deviation consists of 18 firms on average, while tightening the threshold to two standard deviations reduces the number to 9. While the portfolios are sorted on negative events in order to capture the consequent factor, portfolios with fewer firms are inevitably exposed to higher levels of idiosyncratic risks arising from individual companies. Therefore, the likelihood, that the portfolio alpha is driven by specific company risks rather than a response to negative news, increases. The analysis conducted in this paper relies on diversification benefits achieved through rebalancing a random selection of stocks on a monthly basis. This approach allows to approximately isolate the effect of negative news on stock returns. Thus, reducing the number of firms, through e.g. tighter thresholds, has certain drawbacks, with one being that the portfolio returns may be exposed to unwanted factors besides that of the events. 


Anyway, based on the sign and magnitude of abnormal returns from the altered portfolio constructions, it appears that the original results accurately reflect the relation between negative events and investor reactions.  

\subsubsection{Changing portfolio weights}

Section \ref{sec: sens_st_weights} demonstrates that applying more relatively more weight to smaller firms, through an equal-weighted portfolio, increase the short-term abnormal returns through the period $t = -10$ to $t = 0$. On the one side, larger firms, on average, are anticipated to attract more media attention. Presumably, with greater news coverage, the stock price reaction is expected to be more severe due to 1) the news reach a larger potential investor base, and 2) investors receive comprehensive and more detailed information about the specific companies. On the other hand, the severe reaction among small stocks may be attributed to an overreaction when new information becomes available, which is plausible as the market value loss revert to approximately the level of the value weighted portfolio in the following days after more information becomes available. Anyhow, these explanations are not mutually exclusive. 

The findings from the long-term portfolios support this postulate, as relatively higher weight toward small stocks takes part in driving the alphas of the equal weighted portfolio lower for all holding periods.

A simple reason lies in the higher volatility of small stocks. The specific results from the short-term analysis are backed by the early research on media sentiment from \cite{tetlock_sentiment}. The author finds a larger price impact on small stocks relative to larger stocks from ordinary negative sentiment, due to their inherent larger volatility. Our methodologies are not alike, however. My analysis computes returns from companies based on the volume of firm-specific news, which may suppress small firms when applying value weights.  \citeauthor{tetlock_sentiment}, on the other hand, calculates the overall sentiment from a \textit{Wall Street Journal} column and measures the effect on a cross section of stocks, where volatility to general sentiment plays a large role for small stocks. 

The inherently higher volatility of small stocks play a similar role for equal weighted portfolio returns on the long run \citep{Fama_french_3fac}. 
Moreover, the discrepancy may be a feature of the Fama-French regressions. Tables \ref{tab: summary_neg_1} and \ref{tab: summary_neg_EW} in the appendix show the sensitivities to the risk factors of the value and equal weighted portfolios with a holding period of one month.  

Both portfolios have a positive beta to the market excess return in the range of 1.15-1.2 and to high-minus-low (value) of 0.31 and 0.22 for value and equal weights, respectively. With more allocation to smaller stocks, the equal weighted portfolio has a significant beta coefficient of 0.51 to the small-minus-big (size) factor. However, the value weighted portfolio has approximately zero sensitivity to the factor, with a value of -0.04. Although the r-squared is high at 0.89, it appears that the factor regression has issues with explaining the returns of the value weighted portfolios, with most of the variation in returns being explained by the market excess return and the value factor. The same issue does not pertain to the equal weighted portfolio, where the size factor explains a large part of the variations. 






\subsection{News impact from SDG Pillars} \label{sec: short_term_analysis_SDG}

The question in focus is whether specific dimensions of corporate social responsibility are more important to shareholders. The figures \ref{fig:ST_neg_bar} and \ref{fig:ST_pos_bar} illustrate that negative news concerning the SGD Pillars Prosperity, Peace, and Partnership leads to significantly pessimistic market reactions on average, whereas no Pillar for positive news is associated with significant abnormal returns. Given these initial results along with the predominant focus on negative events in existing research, this section will place greater emphasis on analyzing the impact of adverse news. 

Negative news related to Prosperity demonstrates a CAAR of -0.53\%, whereas positive news show no significant impact on returns with a value of approximately 0.1\%. The Prosperity Pillar is mostly driven by SDG 7 (Affordable and Clean Energy) and 8 (Decent work and Economic Growth) which account for 70\% of the volume in negative events. SDG 7 relates to the global transition to sustainable energy sources, in line with the objectives defined in the Paris Agreement, which describes climate risk as a systematic risk. While SDG 7 is associated with most positive events across SDGs, the corresponding investor reaction remains close to zero. This suggests that while there is a significant public attention, these events may be perceived as relatively unimportant from a corporate perspective. In contrast, although negative events also receive a lot of public attention, the events are to a higher degree punished by investors. This hypothesize a discrepancy in the magnitude of shareholder reactions to news concerning clean energy. 

However, \cite{hart1996does} finds a positive relation between reducing emissions and financial performance for a sample of S\&P 500 firms. A newer study from \cite{being_green} elaborates on the relation and shows that neither low-emission or high-emission companies with higher environmental scores perform better financially. According to the the article, companies that do not live up to environmental standards are not necessarily penalized from a profit-maximization perspective. On a side note, even though former research indicates that high-emission, or in other ways low ESG, companies are not penalized financially, they bear the risk of being black-listed by institutional investors with high ESG portfolio requirements, which as an isolated event is expected to increase divestments by other investors further \cite{dell2021norwegian}. 

The Peace Pillar encompass SDG 16 (Peace, Justice and Strong Institutions) solely. The keywords, peace and strong institutions, are not typically associated with important corporate events and financial performance. Justice, on the other hand, is crucial for both, as the keyword related to various negative corporate events like accounting fraud and market manipulation. Consequently, SDG 16 is linked to more than 1300 negative events, surpassing all other SDGs by a decent margin. Events tied to the SDG also exhibit the largest average drop in market value across all sustainability goals, with a CAAR of approximately -1.5\% throughout the entire window. A short term event study conducted by \cite{bauer2010misdeeds} examining alleged corporate misconduct involving governance-related offenses reveal that investors react negatively to corporate filings before any verdict is published. The study reports an average CAAR of -11.6\% based on a Fama-French 3 factors model over a 21-day window. Given the significant declines observed from actual filings, the investor reaction of -1.5\% from spikes in news related to similar events seem appropriate. As mentioned in the literature review in section \ref{lit_rev}, the disparity in methodologies between identifying events based on media activity (as in this case) and hand-picking specific events, as done in \citeauthor{bauer2010misdeeds}) results in discrepancies in results - in this case an average difference of more than 10\%-points. 

The Partnership Pillar focuses solely on SDG 17 (Partnerships for the Goals). While the SDG in itself is relatively vague and supposedly difficult to quantify, as the main message is to strengthen the implementation of global partnerships, it generates a CAAR of -1.1\%. The underlying catalyst for these events and the resulting significant returns are driven by some firms proclaiming to be sustainable across various aspects are revealed to be operating immorally. Such strategies, commonly referred to as 'green-washing,' are heavily disapproved of by activist groups and seemingly also penalized in financial markets, if discovered. News concerning green-washing is related to SDG 17, which is expected to be the main driver of the negative abnormal returns. \cite{Blancard_ESG_sentiment} backs up the argument of investors punishing such misconduct, but also note that successful green-washing can help mediate the financial penalties from adverse ESG events. While green-washing can be overall net positive if executed successfully, the events I focus on specifically aim to consider instances where such misconduct is uncovered. This aligns well with the negative reaction exhibited by shareholders. 

Among the Five Pillars, People stands out with a positive, although insignificant, CAAR of 0.34\% concerning negative events. Social corporate behavior can be difficult to quantify, as the theme encompasses development goals with a focus on poverty, quality education and gender equality among others. As the SDGs within this pillar are more related to supporting developing countries, it is uncommon for regular firms to take a large part, which in part is reflected through the relatively low amount of events for these SDGs, as per figure \ref{fig:event_distribution_SDG}. Thus, immoral acts in this context may be better reflected through long term rather than short term investor reactions. For instance, SDG 3 (Good Health and Well Being) has the second highest number of negative events among the SDGs, however the investor reaction is minimal. The volume of events indicates the media considers well-being of people as a important theme, but investors do not view it as essential from a corporate perspective. 

Contrary to expectations, the short term reaction to negative news concerning the Planet pillar is less severe than anticipated as it incorporates themes on climate action and responsible production, which has received much public attention in the 21st century.

Related research indicate that the average response is negative within these themes.  
from \cite{karpoff2005reputational}, show that, for example, the average legal penalty for environmental violations is 2.26\% with the market penalty being approximately of the same magnitude. Moreover, \cite{capelle2010does} discovers an average penalty of -1.3\% of market value from industrial disasters. With a insignificant and low CAAR from negative news of less than 0.2\% a clear distinction to related research is present. Most of the identified events are related to SDGs 12 and 13, per figure \ref{fig:event_distribution_SDG}, which generate abnormal returns of approximately 0\% according to figure \ref{fig:ST_neg_bar_all}. Hence, climate action (12) and responsible consumption (13) draws a lot of attention from the media, but investors do not react to the news. The same story applies to the CAAR from positive news, which is approximately 0.1\% and insignificant. 



Whether specific SDGs are more important for shareholders is clearly dependent on the category of the news. Positive news related to the 5 P's do not generate abnormal returns in any instance, although Planet and Prosperity comes close. Hence, I conclude that the average impact from positive news is not statistically different from zero and similar across the 5 Pillars of SDGs. On the other hand, abnormal returns related to negative news are varying to a higher degree. For Prosperity, Peace and Partnership the returns are significant, hence it appears shareholders attach more importance to sustainability goals within these themes than People and Planet.    

\subsection{Impact of ESG risk reputation} \label{ESG_reputation}

Generalizing changes in stock returns to overall and specific SDG news provide a good empirical understanding of how investors integrate corporate sustainability in their decision making. However, companies are facing different issues and risks toward ESG, hence external investor pressure might vary accordingly. 
The literature on the effect of reputation is inconclusive. On the one hand, some argue that firms with a good reputation will face more severe investor reactions from failing to live up to their status. \cite{noNewsgoodnews} backs this claim from a media perspective, as they find that accidents related to companies with an admirable CSR record are far more likely to be reported in the media. Others, like \cite{flammer2013corporate} and \cite{godfrey2009relationship}, argue that a history of strong corporate responsibility will mitigate the social pressure from adverse events, meaning that firms with a good reputation experience less decrease of market value from negative ESG news. Moreover, \cite{Blancard_ESG_sentiment} point out that a firm's sector ESG reputation can mitigate the market value loss from negative events.  

Initially, we need to pay attention to the way ESG reputation is defined. \cite{rennings2007effect} point out that outcomes may differ depending on whether reputation is calculated relative to industry peers or not. The ESG Risk Ratings applied in this study are absolute, hence a rating is comparable across all sub-industries. 

The illustrations from figure \ref{fig:ST_neg_ESG} are in contrast with the latter theories. The results from the graph are concretized in table \ref{tab: ST_neg_significance} along with significance levels of 10 and 21 day CAAR. Firms with low ESG risk are penalized more heavily after negative events compared to medium and high-risk firms. As a natural virtue of the rating, firms with low ESG-risks comes with a premium, as they are expected to avoid extreme negative events related to corporate sustainability. For example, a low-risk company may have a relatively high market valuation based on their current status. If they disappoint on ESG-related topics, investors need to adjust their expectations about such events happening in the future and the corresponding change in market value they place with the firm's sustainability profile. The distinction between theory and results may appear due to the media effect as described by \cite{noNewsgoodnews}, with the argument that low-risk firms are more exposed to the media in case of negative events. With that being said, I cannot prove whether the distinctions in market reaction is a consequence of the samples and time periods applied. 

Firms with high ESG-risk are encountering no change in market value from negative news. If investors place a fairly high probability of negative events occurring for high-risk firms, then the risk of an actual events seems to be priced into the market value apriori. Medium ESG-risk firms are experiencing negative abnormal returns, however less negative than low-risk firms, which seems on par with an intermediate investor reaction.  

The reaction to positive news is less clear cut. Medium-risk firms experience abnormal returns from positive events, as expected, while low-risk firms bear market value losses. I find no theoretical underpinnings for the distinction in abnormal returns between the two groups in the literature. Although, \cite{flammer2013corporate} does report that positive market reactions to eco-friendly events are smaller for companies with low environmental risk, and that the increasing social pressure to become green has resulted in decreased reactions to eco-friendly initiatives over time. Moreover, \cite{fisher2011voluntary} finds that positive ESG announcements have been found to be followed by negative abnormal returns, since these decisions often appear to be in contrast to profit-maximization.  
In addition,\cite{serafeim2022stock} finds that consensus ESG ratings predict future ESG news. Hence, a positive rating means investors expect mostly positive news in the future. Such a results can help explain why shareholders are not rewarding low-risk companies that experience a positive event, whereas medium risk firms are rewarded. Likewise, as high risk companies are expected to endure negative situations, they are not penalized to the same degree.  


Overall, I find a clear distinction between ESG-risk levels of companies and their abnormal returns from negative news related to the SDGs. 

\textbf{Long term:}

The gap between low and medium risk profiles does not persist on the long horizon. According to figure \ref{tab:FF5_neg_ESG} these portfolios generate abnormal returns of -0.64\% and -0.61\%, respectively, with a holding period of one month. The outcomes are similar for longer holding periods. 
Although the portfolios generate approximately equivalent alphas, the portfolio characteristics are completely different according to their sensitivities to the five factors, as illustrated in tables \ref{tab: summary_neg_ESG_L} and \ref{tab: summary_neg_ESG_M} in the appendix. For example, the low-risk portfolio is negatively correlated with the size factors at -0.28, whereas the medium risk portfolio has a coefficient of 0.41. Hence, it appears that the loss in market value from negative events are similar across company characteristics and risk toward ESG on a longer horizon. The sensitivities are estimated from the portfolios with holding periods of one month, however they are mostly similar for longer periods, but these values are not reported. 

Furthermore, the pattern cannot be fully generalized to inferences on longer horizons, due to the complete turnaround of the sign on alphas on high risk firms. The high risk portfolio has imminent issues with idiosyncratic risks as it consists of only 81 events and holds only 18 distinct firms throughout the full period, which inevitably plays a sizeable factor in the remarkably negative and fluctuating alpha outcome. Ignoring the outcome of the high risk portfolio, the remaining results do to some degree indicate a pattern similar to that of the short term due negative abnormal returns.

However, as the low and medium risk portfolios generate negative alphas of approximately the same magnitude, the long term abnormal returns are assumed to be independent of the level of companies' ESG-risks. 


\textbf{A slight phrase pre-conclusion on the discoveries from the analysis.}





