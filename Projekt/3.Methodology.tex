\subsection{The Event Study Methodology}

On the surface, measuring the effect of an economic event is often a difficult task for economists. However, with frequent and efficient data from the financial markets an event study can be constructed, which has the capabilities of measuring the impact of an isolated event on the value of the firm. 



The methodology is used to measure investors' reaction to unexpected news. The foundation for event studies emerge from the \textit{efficient market hypothesis} \citep{fama1969_EMH}, which states that security prices reflect all present information. Given rationality on the markets, the effects of an event will be reflected immediately in the price of a security. Consequently, it is possible to measure investors' reaction to specific information by measuring the ex-post realized return relative to the expected return. If investors react adversely to negative and positive events, it should to be reflected in the stock price. The advantage of this procedure is that it is expected to be efficient, since it is based on the overall assessment of many investors who treat the available information. 

Within finance research, event studies has been applied to firm specific events such as mergers and acquisitions and earnings announcements \citep{Event_studies}. The initial task of establishing an event study is to outline the event of interest and determine the window over which the security prices will be inspected. The latter is dependent on the former, as distinct events require different examination time. For example, the impact of an earnings release or a federal interest rate announcement will likely be priced in the market immediately and will thus need a narrow examination window. In contrast, the effects of public news, charges and accusations may require wider windows to reflect all information. 

After identifying the event and examination period, it is necessary to determine the selection criteria for which firms and events to include in the study, where it is useful to summarise some sample characteristics on the firm level. Furthermore, the analysis requires a measure of event impact on firm performance. \cite{Event_studies} defines the effect of an event expressed in terms of the "abnormal return", defined as the difference between the realized ex-post return and the expected return, with the latter characterized as the expected return conditional on the event not taking place: 
\begin{equation}
    AR_{i,t} = R_{i,t} - E[R_{i,t} \mid X_t ]
\end{equation}
where $AR_{i,t}$, $R_{i,t}$, and $E[R_{i,t}$ are the abnormal, actual and expected returns for firm \textit{i} at time \textit{t}, and $X_t$ is the conditional information for the expected returns model. 

The methodology is used to measure investors' reaction to unexpected news. The foundation for event studies emerge from the \textit{efficient market hypothesis} \citep{fama1969_EMH}, which states that security prices reflect all present information, and when new information becomes accessible it will be fully taken into account by investors. Consequently, it is possible to measure investors' reaction to specific information by measuring the ex-post realized return relative to the expected return. If investors react adversely to negative and positive events, it should to be reflected in the stock price. The advantage of this procedure is that 


This paper will apply the event study methodology examine the behavior of stock returns around corporate events, and is used to measure shareholders' reaction to unexpected news. 
- Something with Fama


\subsection{The Market Adjusted Model}

In the context of event studies, an expected return model is a hypothetical prediction of the stock return. The individual stock price return is measured against the market return, as a simple way to control for potential effects of the event on the general market. The model does not adjust for basic CAPM risk and thus abstracts from the individual firm's distinct risk profile. The argument for implementing a simple model is to give a general overview of long and short term return-effects from events, and not to provide statistical conclusions. 
The daily excess returns are measured as the difference between the stock return and the market return on a given day, with the market return assumed to be the MXWO\footnote{https://www.msci.com/World} index:
\begin{equation}
    \text{Abnormal Returns} \:  (AR_{i,t}) = R_{i,t} - R_{M,t} 
\end{equation}
Where, $R_{i,t}$ is the return of stock \textit{i} on day \textit{t}, and $R_{M,t}$ is the return of the market on day \textit{t}. 

The returns series for stock \textit{i} is indexed wrt. the event date (t = 0) by setting the value to 100 and expressing the cumulative return of the future and foregoing observations with respect to this value, to isolate the effect of the event from the discrete trading days. 

The argument is that on t = 0 the information becomes available to the market. However, the investor will not be able to trade on the information before the following trading day. The return index is calculated as the 
cumulative product of the daily excess return leading up to or following the event date multiplied by 100. 

\subsection{Short-term abnormal returns}

The inspection of short-term abnormal return behavior covers (2n+1) trading days surrounding a corporate event: n days before to catch possible insiders information, the day of the event, and n days after. The short term analysis will use N = 10, which leaves a total window of 21 days [-10,+10] to investigate effect from an event. The effect of an event is expressed in terms of the abnormal return, defined as the difference between the realized ex-post return and the expected return, with the latter characterized as the expected return conditional on the event not taking place \citep{Event_studies}. Several models have been applied in the literature to estimate the short-term abnormal returns in event studies, which includes the CAPM, the Market Model, and multi-factor models. According to \cite{holler2014event} the most widely used is the Market Model, which assumes a linear relationship between the individual stock return and the market return. 

\subsubsection{The Market Model}

The Market Model is a single factor statistical model, which relates the return of a given security to the return of the market portfolio. The linear specification follows from the assumed joint normality of asset returns. For any given stock the model is defined as:

\begin{equation} \label{market_model}
    R_{i,t} = \alpha_i + \beta_i * R_{m,t} + \epsilon_{i,t},
\end{equation}
 where $R_{i,t}$ is the return of the stock \textit{i} on day \textit{t}, $\alpha_i$ is the regression intercept\footnote{The excess return of a stock relative to the market, that is not explained by the $\beta$.} for stock \textit{i}, $R_{m,t}$ is the market return on day \textit{t}, and
 $\beta_i$ is the sensitivity of $R_{i,t}$ to the market return.  

 The regression coefficients $\alpha_i$ and $\beta_i$ are computed as the stock returns of a given frequency on the market returns. To analyze the market participants' short-term reaction before and after the event date, the event window is set to +-10 days around the event date, and the estimation window of the regression is set to 120 days prior to the event, as proposed by \cite{Event_studies}. The linear ordinary least squares (OLS) has been applied in the estimation. \cite{brown1985using} investigate the properties of applying daily stock returns in event studies with the market model, and find that the model is well-specified under a variety of conditions. 

 With the expected returns predicted from the Market Model, the abnormal returns are defined as the residual between the realized and the expected returns for each event:
 \begin{equation}
    \text{Abnormal returns} \: (AR_{i,t}) = R_{i,t} - (\alpha_i + \beta_i * R_{m,t})
 \end{equation}

 where $AR_{i,t}$ measures the shareholder's reaction to the event addressing firm \textit{i} at time \textit{t}  with $\alpha_i$ and $\beta_i$ defined as the OLS-parameter estimates from the regression in equation \ref{market_model}. The abnormal return can be transformed into cumulative abnormal returns (for n = (1, ..., 10) by: 
 \begin{equation}
     CAR_{i,t}[-n,+n] = \sum_{\tau = t-n} ^{t+n} AR_{i,t}
 \end{equation}
 To test the hypothesis we compute average abnormal returns ($AAR_t = \overline{AR}_{i,t})$ and cumulative average abnormal returns ($CAAR_{[-n,+n]} = \overline{CAR}_{[-n,+n]}$, and use statistical analysis to test the significance of the results. 

 

\subsection{Long-term abnormal returns}

\subsubsection{Portfolio sorts / Calendar Time Portfolio Approach}

\cite{event_studies_method}



