\subsection{The Event Study Methodology}

On the surface, measuring the effect of an economic event is often a difficult task for economists. However, with frequent and efficient data from the financial markets an event study can be constructed, which has the capabilities of measuring the impact of an isolated event on the value of the firm. 


The methodology is used to measure investors' reaction to unexpected news. The foundation for event studies emerge from the \textit{efficient market hypothesis} \citep{fama1969_EMH}, which states that security prices reflect all present information. Given rationality on the markets, the effects of an event will be reflected immediately in the price of a security. Consequently, it is possible to measure investors' reaction to specific information by measuring the ex-post realized return relative to the expected return. If investors react adversely to negative and positive events, it should to be reflected in the stock price. The advantage of this procedure is that it is expected to be efficient, since it is based on the overall assessment of many investors who treat the available information. 

Within finance research, event studies has been applied to firm specific events such as mergers and acquisitions and earnings announcements \citep{Event_studies}. The initial task of establishing an event study is to outline the event of interest and determine the window over which the security prices will be inspected. The latter is dependent on the former, as distinct events require different examination time. For example, the impact of an earnings release or a federal interest rate announcement will likely be priced in the market immediately and will thus need a narrow examination window. In contrast, the effects of public news, charges and accusations may require wider windows to reflect all information. 

After identifying the event and examination period, it is necessary to determine the selection criteria for which firms and events to include in the study, where it is useful to summarise some sample characteristics on the firm level. Furthermore, the analysis requires a measure of event impact on firm performance. \cite{Event_studies} defines the effect of an event expressed in terms of the "abnormal return", defined as the difference between the realized ex-post return and the expected return, with the latter characterized as the expected return conditional on the event not taking place: 
\begin{equation}
    AR_{i,t} = R_{i,t} - E[R_{i,t} \mid X_t ]
\end{equation}
where $AR_{i,t}$, $R_{i,t}$, and $E[R_{i,t}$ are the abnormal, actual and expected returns for firm \textit{i} at time \textit{t}, and $X_t$ is the conditional information for the expected returns model. 

The methodology is used to measure investors' reaction to unexpected news. The foundation for event studies emerge from the \textit{efficient market hypothesis} \citep{fama1969_EMH}, which states that security prices reflect all present information, and when new information becomes accessible it will be fully taken into account by investors. Consequently, it is possible to measure investors' reaction to specific information by measuring the ex-post realized return relative to the expected return. If investors react adversely to negative and positive events, it should to be reflected in the stock price. The advantage of this procedure is that 


This paper will apply the event study methodology examine the behavior of stock returns around corporate events, and is used to measure shareholders' reaction to unexpected news. 
- Something with Fama



\subsection{The Market Model and the short term}

The inspection of short-term abnormal return behavior covers (2n+1) trading days surrounding a corporate event: n days before to catch possible insiders information, the day of the event, and n days after. The short term analysis will use N = 10, which leaves a total window of 21 days [-10,+10] to investigate effect from an event. The effect of an event is expressed in terms of the abnormal return, defined as the difference between the realized ex-post return and the expected return, with the latter characterized as the expected return conditional on the event not taking place \citep{Event_studies}. Several models have been applied in the literature to estimate the short-term abnormal returns in event studies, which includes the CAPM, the Market Model, and multi-factor models. According to \cite{holler2014event} the most widely used is the Market Model, which assumes a linear relationship between the individual stock return and the market return. 

 \textcolor{red}{Tilføj noget om short term risk fra Kothari s. 22} 

The Market Model is a single factor statistical model, which relates the return of a given security to the return of a market portfolio. The linear specification follows from the assumed joint normality of asset returns. The statistical assumption that asset returns are jointly multivariate normal and independently and identically distributed through time is imposed. \cite{Event_studies} states that these assumptions are sufficient for the market model to be correctly specified, which in practice is empirically reasonable as inferences using normal return models tend to be robust to deviations from the assumptions.    For any given stock the model is defined as:

\begin{equation} \label{market_model}
    R_{i,t} = \hat{\alpha_i} + \hat{\beta_i} * R_{m,t} + \hat{\epsilon}_{i,t},
\end{equation}
 where $R_{i,t}$ is the return of the stock \textit{i} on day \textit{t}, $\alpha_i$ is the regression intercept\footnote{The excess return of a stock relative to the market, that is not explained by the $\beta$.} for stock \textit{i}, $R_{m,t}$ is the market return on day \textit{t}, and
 $\beta_i$ is the sensitivity of $R_{i,t}$ to the market return.  

 The regression coefficients $\alpha_i$ and $\beta_i$ are computed as the stock returns of a given frequency on the market returns. To analyze the market participants' short-term reaction before and after the event date, the event window is set to +-10 days around the event date. The estimation window of the regression is set to 120 days prior to the event, as proposed by \cite{Event_studies} to mitigate for sampling error and serial correlation in abnormal returns. The estimation window is practically $[-130;-11]$ prior to the event to avoid contamination from the days impacted by a potential event. Under the null hypothesis that a given event has no impact on returns, the distributional properties of the abnormal returns can be used to draw inferences in the event period. 
 The linear ordinary least squares (OLS) has been applied in the estimation. In addition to \cite{Event_studies},\cite{brown1985using} investigate the properties of applying daily stock returns in event studies with the market model, and find that methodologies based on the OLS market model using standard parametric tests are well-specified under a variety of conditions. Daily excess returns may suffer from non-normality, however the mean excess return in a cross-section of securities converges to normality as the amount of sample securities increases. For hypothesis tests over multi-day intervals the variance estimator is crucial, however it is only necessary to reflect autocorrelation in the estimated variance in special cases. An absence of clustering in events is sufficient for the requirement of a consistent variance estimator. According to \cite{kothari} cross-correlation in abnormal returns is irrelevant in short-window event studies when the event is not clustered in calendar time.

 With the expected returns predicted from the Market Model, the abnormal returns (AR) are defined as the residual between the realized and the expected returns for each event:
 \begin{equation}
    AR_{i,t} = R_{i,t} - (\hat{\alpha_i} + \hat{\beta_i} * R_{m,t})
 \end{equation}

 where $AR_{i,t}$ measures the shareholder's reaction to the event addressing firm \textit{i} at time \textit{t}  with $\alpha_i$ and $\beta_i$ defined as the OLS-parameter estimates from the regression in equation \ref{market_model}.
  
 The individual stock abnormal returns can be aggregated for each event as the average abnormal return (AAR), by:  
 \begin{equation}
 AAR_t = \frac{1}{N} \sum_{i=1} ^N AR_{i,t},
 \end{equation}
 \begin{equation}
 var(AAR_{t}) = \frac{1}{N^2} \sum_{i=1} ^N \sigma_{\epsilon_i} ^2.
 \end{equation}
 The AAR can then be aggregated from $t_1$ to $t_2$ to accommodate a multiple period event window as the cumulative AAR (CAAR), by:
 \begin{equation}
 CAAR_{[t_1,t_2]} = \sum_{t = t_1 } ^{t_2} AAR_{t},
 \end{equation}
 \begin{equation}
 var(CAAR_{[t_1,t_2]}) = \sum ^{t_2}_{t=t_1} var(AAR_t).
 \end{equation}
 
Inferences about the CAAR can be drawn by using: $CAAR_{[t_1,t_2]} \sim N[0,var(CAAR_{[t_1,t_2]})]$ to test the null hypothesis that the abnormal returns are zero. I use the sample variance from the market model regression as a estimator of $\sigma^2_{\epsilon_i}$. The distribution is asymptotic with the the number of stocks N and the length of the estimation window \cite{Event_studies}.  
 
 

\subsection{Long-term abnormal returns}

Evidence of long-term abnormal returns negate the Efficient Market Hypothesis, which states that security prices fully adjust within a narrow time period as a reaction to new information becoming available to the market \citep{fama1969_EMH}. 

When moving from short-term measurement horizons to long-term, two consideration are important to incorporate in the estimation of risk-adjusted returns; namely an appropriate adjustment for risk and the model for expected returns. First, in a multi-year test, even small errors in risk-adjustment can make significant discrepancies in measured abnormal performance, whereas the same issues become negligible for short-horizon studies. \\
Second, from the standpoint of event study research, the purpose is to isolate the event and measure the related abnormal returns. Further, the researcher needs to test the null hypothesis that the distribution of the long term abnormal returns concentrates around zero \citep{Ang_event_method}. the researcher needs to differentiate the performance associated with an event from other known determinants. The \cite{Fama_french_3fac} five-factor model is a popular choice. 

The measurement of long horizon abnormal returns are challenged by misspecification of test statistics. \cite{Lyon_1997_test_stats} studies the empirical power in long-term abnormal stock returns and identifies three sources for misspecification: 1) New listing bias\footnote{The new listing bias arises since sample firms in event studies often have a long posterior history of returns, while the constituents of the reference portfolio may include firms that began trading subsequent to the event period.}, 2) rebalancing bias\footnote{The rebalancing bias emerge since the compound returns of a reference portfolio, such as a weighed market index, are typically calculated assuming periodic rebalancing, while the returns of sample firms are compounded without rebalancing. }, and 3) skewness bias\footnote{ the skewness bias arises because long-run returns are typically positively skewed.}. Whether these biases give rise to misspecification depends on the method used in identifying abnormal returns. Additionally, approaches might also suffer from cross-correlation bias and the bad model bias (\citep{fama1998_events}; \citep{mitchell2000managerial}. The cross-correlation bias arises because matching on firm characteristics may fail to completely fail to remove the correlation between event firm's returns. The bad model bias occurs since no benchmark will provide a perfect estimate of the counter factual (i.e., if no event had happened) return of an event firm and benchmark errors accumulate in calculating long-term abnormal returns.

In their evaluation of the methodology behind long-horizon event studies,  examines the performance of various testing procedures.

In a study evaluating long-horizon event study methodologies, \cite{Ang_event_method} highlight two approaches which have been utilized substantially in the finance literature. The buy-and-hold approach calculates the abnormal returns as the difference between the return from the event firm and a benchmark. The calendar-time-portfolio (CTP) approach forms a portfolio each period consisting of firms that have experiences an event, and tests the null hypothesis that the intercept is zero in a regression against the factors in an asset-pricing model.

The buy-and-hold approach is sensitive to biases connected with cross-correlation, insufficient matching, new equity issues, periodic rebalancing and skewed distribution of long-term returns, while the CTP approach might encounter issues with the choice of asset pricing model and heteroskedasticity in portfolio returns \citep{Ang_event_method}.  


\subsubsection{Calendar Time Portfolio}

The calendar-time portfolio Approach (or Jensen-alpha) is used to calculate monthly portfolio returns for firms experiencing an event, and test whether they are abnormal in a factor regression model. The estimated intercept (alpha) from the regression is the post-event risk-adjusted abnormal return from the sample event firms. \\
Since the stock price impact of negative and positive events can be spread over a longer period, the method can assist in measuring the long term abnormal returns following an event. This paper seeks to investigate the impact on stock prices from events over a period of three months to one year $(T = 3, 6, 12 \text{ months})$ following an event for each sample firm. 

To implement the method a portfolio is constructed each calendar month consisting of all firms experiencing an event within the preceding \textit{T} months. As the number of firms encountering an event is not uniformly distributed through the sample period, the amount of firms in the portfolio is not constant. 

\cite{fama1998_events} outlines the advantages of the CTP approach as emerging from the use of monthly returns and rebalancing. He argues that monthly returns are less skewed, which makes the model less sensitive to the bad model problem. Further, the time-series variation of the rebalanced portfolio captures the effect of the cross-correlation between event stocks returns. 

Due to the changes in the composition of the portfolio, heteroskedasticity might be introduced in the error term. For example, \cite{ritter1995} argues that anomalies can be understated if events bunch in time due to e.g. timing of events. However, the issue of heteroskedasticity can be addressed by using the weighted least squares (WLS) technique with the monthly amount of firms as weights in the regression \citep{Ang_event_method}.

The calculation of monthly portfolio returns can be based on value or equal weights. \cite{fama1998_events} suggests that apparent anomalies might shrink when event firms are value-weighted, which possibly yield the right perspective because it more accurately captures the wealth-effect experienced by an investor. Moreover, an equal-weighted portfolio will put more weight on small stocks, which might increase the the bad-model bias, as asset-pricing models have systemic issues explaining the average return of small stocks. However, as the amount of firms in the portfolio may be relatively small, the portfolio return may be dominated by the outcome of large firms. As this is unfavorably when trying to estimate a relationship between specific events and stock returns, the equal-weighted portfolio returns will be applied in the analysis. 

The portfolio return is calculated as the average return of firms that experienced an event within the preceding T months:

\begin{equation}
    R_{p,t} = \sum_{t = 1} ^{T} \frac{1}{N_t} R_{i,t}.
\end{equation}

Where $R_{i,t}$ is the stock \textit{i} return in period \textit{t} and \textit{N} is the amount of stocks in the portfolio in period \textit{t}. Hence, firms experiencing more than one event in the preceding T months will only be included once, which assists in keeping the portfolio equal-weighted. 

The portfolio returns are regressed against the factors in the \cite{Fama_french_3fac} three-factor model to determine whether the portfolio has achieved abnormal returns when adjusting for the underlying risk. Under the assumption that the three-factor model present a full description of the expected stock returns, the intercept, $\alpha$, measures the risk-adjusted average abnormal return of the portfolio, which should be zero under the hypothesis of no abnormal returns. Thus, the accuracy of the description depends on the models ability to explain the risk factors in the market. The model aims to describe the risk of the market through three factors; 1) the market risk premium, 2) out-performance of small companies relative to large companies (small-minus-big), and 3) the out-performance of high book-to-market value firms  relative to low book-to-market value firms (high-minus-low). 

The representation of the Fama-French three-factor model is:

\begin{equation} \label{eq: FF5}
    R_{p,t} - R_{rf,t} = \alpha_p + \beta_1(R_{m,t} - R_{rf,t}) + \beta_2 SMB_t + \beta_3 HML_t + \epsilon_{p,t} 
\end{equation}

with $R_{rf,t}$ being the risk-free rate\footnote{US treasuries / EUR treasuries} at time \textit{t} and the $\beta$'s representing the sensitivity of the portfolio return towards the market risk premium, and the return of, respectively, small-minus-big (SMB) and high-minus-low (HML) portfolios for developed markets in Europe.   



\subsection{Data and general setup}

The potential amount of news related to sustainable development is enormous. Thousands are published every day around the globe. In order to assess what drives the financial markets and their impact on a broad portfolio, we need a large database with frequent updates. When conducting an event study, many researchers compose their database by either collecting events from archives or utilizing an existing database of comprehensive and carefully selected events. As both approaches involve using a database of hand-picked events, their observations updates infrequently, hence the data and the results are not useful in practice. \\
In this study I use a data set called SDG Signals provided by Matter\footnote{https://www.thisismatter.com/sdg-signals}. The company has developed a systematic collection of daily news articles from more than 150.000 media sources, which measure the amount of times a specific company has been mentioned positively, negatively, or neutrally in relation to a specific SDG or on a broad level in news articles worldwide.   
The data set is updated every day at midnight and uses artificial intelligence to relate every single news article to more than 75.000 publicly traded companies. 

Consequently, the relevant events analyzed in this paper are not specified directly in the data set. Instead, I apply a rule-based approach to detect large spikes in news articles for individual companies. The short term analysis applies daily observations, and identifies an event when the amount of news articles is greater than a pre-specified threshold. For the initial analysis, the threshold is set to a one standard deviation move from the mean of daily articles. Moreover, the threshold requires the amount of positive news to be twice as large as negative news, and vice versa, on a given day in order to avoid misleading events. News articles published on a Saturday or Sunday are mapped as occurring on a Monday, as the share price response will manifest here. I decontaminate the daily amount of news articles per company by subtracting the amount of irrelevant articles from the total amount. The methodology produces 1.046 negative events and 3.565 positive events from 232 distinct companies from January 2018 to January 2023. The long term analysis apply the same setup, however, the daily news articles are aggregated into monthly figures, upon which events are identified as spikes in the monthly sum. I identify 1.117 negative events and 2.054 positive events from 437 distinct companies. 

The events are distributed evenly throughout the 5 year period with the exception of some of the initial months of the COVID-19 pandemic as illustrated in figure \ref{fig:event_distribution} in the appendix. The distribution is sufficient evidence to confirm that events are not clustering in calendar time. Moreover, figure \ref{fig:event_distribution_SDG} in the appendix shows the events are not evenly distributed across SDGs as 3, 7, 9, 12, 13, 16, and 17 receive considerably more media attention than the remaining. 

In order to examine the permanency of the hypotheses I select all constituent from the \textit{STOXX Europe 600} index that traded as of 2018-01-01. The index covers 90\% of the market capitalization of the developed European equity market. After discarding companies that have not experienced an event, the total dataset contains N = 232 stocks. Daily and monthly market capitalization\footnote{I use the free-float market capitalization calculated in Euro to best reflect the market movements} and returns, adjusted for stock splits and dividends, are gathered from Bloomberg\footnote{https://www.bloomberg.com.}. Additionally, I extend the dataset by ESG Risk Ratings provided by Sustainalytics\footnote{https://www.sustainalytics.com/esg-ratings. The ESG Risk Ratings are the contemporaneous ratings and not a historical time series. As the analyzed period is only five years, I assume that the actual changes in ratings would not affect my results significantly.}, which identify individual firms' risk toward ESG as either negligible, low, medium, high, or severe. Since the groups negligible and severe contain few firms, I couple these to the groups low and high, respectively, in order to exploit the data fully.    
In order to investigate the relation between SDG-news and corporate performance, I identify significant events on a daily and monthly basis by the methods described above and compute the corresponding development in abnormal returns. 

The abnormal returns from the Market Model and the CTP are estimated using the STOXX Europe 600 Index and the Fama-French European 5 Factors \footnote{http://mba.tuck.dartmouth.edu/pages/faculty/ken.french/Data\_Library/f-f\_5developed.html}, respectively. All portfolio returns are weighted by market capitalization, unless stated otherwise. 

